\section{Searching for Iterative Trails and Estimation of Differentials and Linear Hulls\label{sec:method}}

\subsection{Generating a Graph Modelling Difference or Mask Propagations}

In a directed graph, each vertex can be associated with a difference or mask value. Each edge $u\rightarrow v$ represents a 1-round differential or linear trail. Given the round transformation $F$ of an iterative block cipher or permutation $\calE$, in the case of differential cryptanalysis, the weight of an edge is defined as:
\[
    w(u\rightarrow v)=\Prob^{F}(u,v).
\]
In the case of linear cryptanalysis, if $\calE$ is a block cipher, the weight of an edge is defined as:
\[
    w(u\rightarrow v)=(\Cor^{F}(u,v))^2;
\]
else if $\calE$ is a permutation, the weight is defined as:
\[
    w(u\rightarrow v)=\Cor^{F}(u,v).
\]

In this way, a weighted directed graph $G_{F}=(V_{F},E_{F})$ is generated to describe the difference or mask propagations given the round function $F$. A path in the graph represents a differential or linear trail. A hull with length $k$ in the graph represents a $k$-round differential or linear hull. 

However, the $G_F$ generated is too large for it has $2^n$ vertices, while an exploitable differential or linear trail usually has a large probability or correlation in each round. In order to limit the size of $V_F$, we set a parameter $max\_asn$ defined as the maximum number of active S-boxes that a vertex can has. Given $\text{Asn}(\cdot)$ as the function that returns the number of active S-boxes of a difference or mask, $G_{F,max\_asn}=(V_{F,max\_asn},E_{F,max\_asn})$ is an induced subgraph of $G_F$ where $V_{F,max\_asn}$ is given by
\[
    V_{F,max\_asn}=\{u|u\in V_F \text{ and } \text{Asn}(u)\leq max\_asn\}.
\]

\subsection{Finding the Best Iterative Trail}\label{sec:fbit}

\begin{definition}[Iterative Trails]\label{def:it}
	A differential or linear trail $(v^{(0)},\cdots,v^{(r)})$ is iterative if $v^{(0)}=v^{(r)}$.
\end{definition}

\begin{definition}[Elementry Iterative Trails]
    An iterative differential or linear trail $(v^{(0)},\cdots,v^{(r)}=v^{(0)})$ is elementary if $v^{(i)}\neq v^{(j)},\forall i,j\in [0,r-1]$.
\end{definition}

According to the definition, obiviously a (elementary) circuit in the graph represents an (elementary) iterative trail. Applying Johnson's algorithm in \cite{J75}, for a graph with $n$ vertices, $e$ edges and $c$ elementary circuits, we can enumerate all elementary circuits in $G_{F,max\_asn}$. The time complexity is $O((n + e)(c + 1))$ and memory complexity is $O(n + e)$.

Once a short-round elementary iterative trails is found, it can be concatenated to itself arbitary times and forms a long-round iterative trail. For a cipher that has iterative trails, the negative logarithm of the largest differential probability or linear correlation increases no faster than a linear function of the round number. 

For an $r$-round iterative trail with weight $w$, we regard its weight per round $\sqrt[r]{w}$ as an index describing the weight growth of the iterative trail. We refer the best iterative trail to the one with the largest weight per round. We claim that one of the best iterative trails must be an elementary one. Thus we can find out the best elementary iterative trail among the ones enumerated by Johnson's algorithm \cite{J75} and evaluate the best weight growth of one single iterative trail. The procedure is given in Algorithm \ref{algo:1}. %and the details of Johnson's algorithm are given in Appendix. 

\begin{algorithm}
	\caption{Finding the best iterative trail}
	\label{algo:1}
	\begin{algorithmic}[1]
		\Require $G_{F,max\_asn}$
		\Ensure the largest weight per round of an iterative trail
		\Procedure {}{}
		\State $wpr\leftarrow 0$
		\For{each elementary circuit $p'$ in $G_{F,max\_asn}$ enumerated by Johnson's algorithm}
		\If{$\sqrt[l(p')]{w(p')}>wpr$}
		\State $wpr\leftarrow \sqrt[l(p')]{w(p')},p\leftarrow p'$
		\EndIf
		\EndFor
		\State \Return{$wpr$}
		\EndProcedure
	\end{algorithmic}
\end{algorithm}

\subsection{Finding the Best Iterative Hull}\label{sec:fbih}

\begin{example}\label{eg:1}
	Suppose that we find two elementary iterative trails $p_0=(u_0,u_1,u_0$ and $p_1=(u_0,u_2 ,u_0$. $p_0)$ itself can form a 4-round iterative trail $(u_0,u_1,u_0,u_1,u_0)$ with weight $w^2(p_0)$. Additionally combined with $p_1$, 4 4-round iterative trails can be formed which are
	\begin{align*}
		&(u_0,u_1,u_0,u_1,u_0)\\
		&(u_0,u_1,u_0,u_2,u_0)\\
		&(u_0,u_2,u_0,u_1,u_0)\\
		&(u_0,u_2,u_0,u_2,u_0).
	\end{align*}
	Thus a 4-round iterative hull containing 4 trails with weight $w^2(p_0)+w^2(p_1)+2w(p_0)w(p_1)$ can be constructed.
\end{example}

From Example \ref{eg:1}, we find that it's not enough to evaluate the best weight growth of one single iterative trail. If two elementary circuits have common vertices, then the combination of elementary iterative trails may result in iterative hulls with better weight growths.  

To evaluate the weight growth of iterative hulls, we define subgraph \textit{iterative structure}. 

\begin{definition}[Iterative Structure]
	An induced subgraph $G'=(V',E')$ is a (maximal) iterative structure of $G$ if for all $u\in V'$, there exist at least one edge originating from it and one edge terminating at it. And this property holds for subgraph of $G$ induced by a vertex set $\overline{V'}$ such that $V' \subset \overline{V'} \subseteq V$.
\end{definition}

We claim that for an arbitary vertex in the iterative structure, it whether belongs to at least one elementary circuit or belongs to at least one path linking two elementary circuits. 

To compute the iterative structure $G^{IS}_{F,max\_asn}$ of $G_{F,max\_asn}$, we eliminate the vertices in having no incoming or outgoing edges until the size of the graph becomes stable. 

Given the round number $r$, we can compute the iterative hull $w(h_{u,u}^r)$ for all $u$ in $V^{IS}_{F,max\_asn}$. An $r$-round iterative hull $h_{u,u}^r$ is a hull containing all iterative trails with the same input and output difference of mask value. We regard $\max\limits_{u}\sqrt[r]{w(h_{u,u}^r)}$ as an index describing the best weight growth among the iterative hulls. The procedure of obtaining the best weight per round of the iterative hulls is given in Algorithm \ref{algo:2}.


\begin{algorithm}
	\caption{Finding the best iterative hull}
	\label{algo:2}
	\begin{algorithmic}[1]
		\Require $G_{F,max\_asn}$, the number of rounds $r$
		\Ensure the largest weight of an $r$-round iterative hull
		\Procedure {}{}
		\State Eliminate all vertices having no incoming or outgoing edges repeatedly until no more vertices can be eliminated and restore the subgraph into $G^{IS}_{F,max\_asn}$
		\State Let $\mathcal{H}$ be a hash table. Set $\mathcal{H}(u,v)\leftarrow 0$ for all $u,v\in V^{IS}_{F,max\_asn}$
		\For{$i=1:r$}
		\State Create a new hash table $\mathcal{H}'$. 
		\For{each pair of keys $(u,v)$ of $\mathcal{H}$}
		\For{each edge $v\rightarrow x$ originating from $v$ in $G^{IS}_{F,max\_asn}$}
		\State $\mathcal{H}'(u,x)\leftarrow \mathcal{H}'(u,x)+\mathcal{H}(u,v)\cdot w(v\rightarrow x)$
		\EndFor
		\EndFor
		\State $\mathcal{H}\leftarrow \mathcal{H}'$
		\EndFor
		\State \Return{$\sqrt[r]{\max\limits_{u=v}\mathcal{H}(u,v)}$}
		\EndProcedure
	\end{algorithmic}
\end{algorithm}

\subsection{Finding the Best Differential or Linear Hull}\label{sec:fbh}

In \cite{W08}, a 4-round elementary iterative trail is found and is used to construct a 12-round iterative trail. By further extending the trail forward and backward by one round, a 14-round trail is obtained. Following this idea, we hope to collect trails constructed by iterative trails as many as possible. Such a trail is treated as three parts: the core middle part which is a path in $G^{IS}_{F,max\_asn}$, the extension forward part and the extension backward part. 

In \cite{HV18}, a multistage graph is generated and pruned in order to find good differentials and linear hulls. Here given the number of rounds $r$, we generate a multistage graph $MSG$ with $r+1$ stages from $G^{IS}_{F,max\_asn}$. Firstly, we append every edge in $E^{IS}_{F,max\_asn}$ between every two adjacent stages in $MSG$. For each vertex in $V^{IS}_{F,max\_asn}$, we extend a trail both forward and backward from it. The number of rounds of the extended trails is limited within a parameter $k$. The weight of the extended trails is limited by a parameter $wb$. Thus for an extended trail $p=(v_0,v_1,\cdots,v_l)$, it has 
\begin{enumerate}
	\item $v_0\in V^{IS}_{F,max\_asn}$ or $v_l\in V^{IS}_{F,max\_asn}$,
	\item $l\leq k$,
	\item $w(p)\times (mw)^{k-l} \geq wb$ where $mw$ is the largest weight that a 1-round trail can have.
\end{enumerate}
The first restriction ensures the extended trails are attached to $G^{IS}_{F,max\_asn}$. The next two restrictions are used to keep the extended trails with large weights while keeping the collection of extended trails efficient. We restore these extended trails into $MSG$. The generating procedure is given in Algorithm \ref{algo:3}. 

Once $MSG$ generated, we can compute the largest weight that a hull in $MSG$ can has. The computation is conducted stage by stage. The procedure is given in Algorithm \ref{algo:4}.

\begin{algorithm}
	\caption{Generating multistage graph $MSG$}
	\label{algo:3}
	\begin{algorithmic}[1]
		\Require $G^{IS}_{F,max\_asn}$, the number of rounds $r$, the number of round to be extended $k$, a weight bound $wlb$ constraining extended trails
		\Ensure multistage graph $MSG$
		\Procedure {}{}
		\State Build a multistage graph $MSG$ with $r+1$ stages $S_0,\cdots,S_r$
		\For{$i=0:r+1$}
		\State $S_i\leftarrow V^{IS}_{F,max\_asn}$
		\EndFor
		\For{each edge $u\rightarrow v$ in $E^{IS}_{F,max\_asn}$}
		\State Append $u\rightarrow v, u\in S_i,v\in S_{i+1}, \forall i$ to $MSG$
		\EndFor
		\State $min\_w\_r\leftarrow$ the largest weight of one single S-box
		\For{each trail $p=(v_0,v_1,\cdots,v_l)$ extended from a vertex in $V^{IS}_{F,max\_asn}$ with $l\le k$ and $w(p)\times (min\_w\_r)^{k-l} \ge wlb$}
		\If{$p$ is a trail extended forward from $v_0\in V^{IS}_{F,max\_asn}$}
		\For{$i\leftarrow 1:l$}
		\State Append $v_{i-1}\rightarrow v_i$ between $S_{r-l+i-1}$ and $S_{r-l+i}$
		\EndFor
		\ElsIf{$p$ is a trail extended backward from $v_l\in V^{IS}_{F,max\_asn}$}
		\For{$i\leftarrow 1:l$}
		\State Append $v_{i-1}\rightarrow v_i$ between $S_{i-1}$ and $S_{i}$
		\EndFor
		\EndIf
		\EndFor
		\EndProcedure
	\end{algorithmic}
\end{algorithm}

\begin{algorithm}
	\caption{Finding the best differential or linear hull}
	\label{algo:4}
	\begin{algorithmic}[1]
		\Require $MSG$
		\Ensure the largest weight of an $r$-round hull
		\Procedure {}{}
		\State Let $\mathcal{H}$ be a hash table. 
		\For{each $u\in S_0$}
		\State $\mathcal{H}(u)\leftarrow 0$
		\For{$i=1:r$}
		\State Create a new hash table $\mathcal{H}'$. 
		\For{each key $v$ of $\mathcal{H}$}
		\For{each edge $v\rightarrow x, v\in S_{i-1},x\in S_i$}
		\State $\mathcal{H}'(x)\leftarrow \mathcal{H}'(x)+\mathcal{H}(v)\cdot w(v\rightarrow x)$
		\EndFor
		\EndFor
		\State $\mathcal{H}\leftarrow \mathcal{H}'$
		\EndFor
		\If{$\max\limits_{v}\mathcal{H}(v)>wt$}
		\State $wt\leftarrow \max\limits_{v}\mathcal{H}(v)$
		\EndIf
		\EndFor
		\State \Return{$wt$}
		\EndProcedure
	\end{algorithmic}
\end{algorithm}


%For any trail based on iterative trails, it can be treated as three parts: the extension backward, the iterative trail and the extension forward. The 14-round differential trail of PRESENT found in \cite{W08} is shown in Table \ref{tab:trail-present}. It is constructed by concatenating a 4-round iterative trail to itself two times and extending both forward and backward by 1 round. Thus the subtrail from round 0 to round 1 is the extension backward part, the one from round 1 to round 13 is the iterative trail part and the one from round 13 to round 14 is the extension forward part. In the following, we try to compute the largest probability a trail of such type can has based on $G^{IT}_{F,max\_asn}$. 

%In another way, a weighted 2-stage graph $G^1_F=(V^1_F,E^1_F)$ can also be generated to describe the 1-round differentials or linear hulls. $G^1_F$ has 2 stages $S_0$ and $S_1$ each being $\mathbb{F}_2^n$. For $u\in S_0$ and $v\in S_1$, the weight of an edge $w^1(u\rightarrow v)$ is defined the same way as $l(u\rightarrow v)$ in $G_F$ is.

%If $S_0=S_1$, then $G^1_F$ can be iteratively concatenated to itself. 


\section{Improvements}\label{sec:improvements}

\subsection{Expanding $G_{F,max\_asn}$}\label{sec:expansion}

When generating graph $G_{F,max\_asn}$, we only consider 1-round trails with the maximum number of S-boxes of the input and output difference or mask values no larger than $max\_asn$. However, a difference or mask value of which the number of S-boxes is larger than $max\_asn$ may also participates in the differential or linear propagations. In the following, we define a set of trails:

\begin{definition}
	A trail $p=(u^{(0)},u^{(1)},\cdots,u^{(r)})$ belongs to the set of trails $T_{a,b,c}$ if (1) $\Asn(u^{(0)})\leq a,\Asn(u^{(r)})\leq a$, (2) $a< \Asn(u^{(i)})\leq b,\forall 0<i<r$ and (3) $r\leq c$.
\end{definition}

According to the definition above, the generated graph $G_{F,max\_asn}$ only considers the set of trails $T_{max\_asn,*,1}$. We denote a graph generated by the collected trails $T_{a,b,c}$ by $G_{F,T_{a,b,c}}$. With each parameter of $T_{a,b,c}$ larger, the size of $G_{F,T_{a,b,c}}$ becomes larger and thus the results obtained afterwards will be more accurate because the search space is larger. The parameters $a,b,c$ should be set as large as possible while keeping the collection of trails in $T_{a,b,c}$ efficient. The choice of $a,b,c$ is a trade-off between the accuracy of results and the efficiency of the algorithm.

\subsection{Modification in case of Rotation Equivalence}

According to \cite{ZBL15}, RECTANGLE has the property of rotational symmetry, that is, all its operations have rotational symmetry. 

\begin{definition}[Rotational Symmetry and Rotation Equivalence]
	First suppose that a trail $p=(u^{(0)},u^{(1)},\cdots,u^{(r)})$ is found valid. If the cipher is based on SPN structure, the difference or mask value $u^{(i)}$ can be further described by S-boxes as $u^{(i)}=(u_1^{(i)},u_2^{(i)},\cdots,u_k^{(i)})$ where $k$ is the number of S-boxes of each round. If the cipher has rotation equivalence, then another trail $p'=(v^{(0)},v^{(1)},\cdots,v^{(r)})$ is also valid, if there exists $j$ such that $(v_1^{(i)},v_2^{(i)},\cdots,v_k^{(i)})=(u_{j+1}^{(i)},u_{j+2}^{(i)},\cdots,u_{j}^{(i)}),\forall i$. And we call two values $u=(u_1,u_2,\cdots,u_k)$ and $v=(v_1,v_2,\cdots,v_k)$ are rotation equivalent if there exists $j$ such that $(u_1,u_2,\cdots,u_k)=(v_{j+1},v_{j+2},\cdots,v_{j})$. 
\end{definition}

\begin{example}\label{exp:itre}
	Suppose that a cipher $\mathcal{E}$ has rotational symmetry and a 1-round trail $0x3000f000\rightarrow 0x03000f00$ is found. According to Definition \ref{def:it}, it's not an iterative trail. However, it can also be concatenated to itself arbitary times.
\end{example}

From Example \ref{exp:itre}, we can see that if a cipher has rotational symmetry, the definition of iterative trails should be modified as follows:  

\begin{definition}[Iterative Trails under Rotation Equivalence]
	A differential or linear trail $(v^{(0)},\cdots,v^{(r)})$ is iterative if $v^{(0)}$ and $v^{(r)}$ are rotation equivalent.
\end{definition}

The rotation equivalence then defines a rotation equivalent class. In the euivalent class, we define a representive element as follows:

\begin{definition}[Rotation Equivalent Class and Rotation Representive Element]
	A rotation equivalent class is a difference or mask value maximal set that for every two elements $u$ and $v$ in it, we have $u$ and $v$ are ratation equivalent. The representive element of the rotation equivalent class is the largest value in it. The rotation is operated on S-boxes. For any element $u$ in a rotation equivalent class, we call the distance between $u$ and its representive element $u'$ the left rotation number from it to its representive elelent and denote the distance as $d(u)$. 
\end{definition}

We denote the transform from $u=(u_1,u_2,\cdots,u_k)$ to $u'=(u_{j+1},u_{j+2},\cdots,u_{j+k})$ where $u_i$ is the value of the i-th S-box as $\rotl_j(\cdot)$. In the same way, the inverse transfrom is $\rotr_j(\cdot)$. We denote the transform from $u$ to its representive element as $r(\cdot)$. We define $d(u\rightarrow v)=(d(v)-d(u))\mod sn$ where $sn$ is the number of S-boxes. 

In the graph generation of a cipher having rotational symmetry, a vertex representives a rotation equivalent class instead of a value. However, This results in multiple costs of one edge. Or it can be seen as multiple edges between two vertices. 

Taking advantage of the property of rotational symmetry, which RECTANGLE, KNOT permutations and ASCON permutation has, we can avoid restoring duplicate differential and linear propagations and thus accelerate the computation process. %The modified versions adapting to rotational symmetry of algorithms in Section \ref{sec:method} are given in Appendix.

\begin{example}
	For a cipher having rotational symmetry, suppose that we find two 1-round trails $0x3000f000\rightarrow 0x0e000020$ with weight $2^5$ and $0xf0003000\rightarrow 0xe0000200$ with weight $2^6$, then edge $u=0xf0003000\rightarrow v=0xe0000200$ is appended to the graph. The edge has two costs: $w(u\rightarrow v)=2^5$ when $d(u\rightarrow v)=5$ and $w(u\rightarrow v)=2^6$ when $d(u\rightarrow v)=0$. 
\end{example}
