\section{Searching for Iterative Trails and Estimation of Differentials and Linear Hulls\label{sec:method}}

\subsection{Generating a Graph Modelling Difference or Mask Propagations}

In a directed graph, each vertex can be associated with a difference or mask value. Each edge $u\rightarrow v$ represents a 1-round differential or linear trail. Given the round transformation $F$ of an iterative block cipher or permutation $\calE$, in the case of differential cryptanalysis, the weight of an edge is defined as:
\[
    w(u\rightarrow v)=\Prob^{F}(u,v).
\]
In the case of linear cryptanalysis, if $\calE$ is a block cipher, the weight of an edge is defined as:
\[
    w(u\rightarrow v)=(\Cor^{F}(u,v))^2;
\]
else if $\calE$ is a permutation, the weight is defined as:
\[
    w(u\rightarrow v)=\Cor^{F}(u,v).
\]

In this way, a weighted directed graph $G_{F}=(V_{F},E_{F})$ is generated to describe the difference or mask propagations given $F$. A path in the graph represents a differential or linear trail. A hull with length $k$ in the graph represents a differential or linear hull. 

However, the $G_F$ generated is too large for it has $2^n$ vertices. Therefore we need to choose a interesting graph according to computation power. 

\subsection{Choosing a Subgraph}

In order to limit the size of $V_F$, we set a parameter $max_asn$ defined as the maximum number of active S-boxes that a vertex can has. Then $G_{F,max\_asn}=(V_{F,max\_asn},E_{F,max\_asn})$ is an induced subgraph of $G_F$. $V_{F,max\_asn}$ is given by
\[
    V_{F,max\_asn}=\{u|\text{Asn}(u)\leq max\_asn \text{ and } u\in V_F\}
\]
where $\text{Asn}(\cdot)$ is a function returns the number of active S-boxes of a difference or mask.

\subsection{Finding the Best Iterative Trail}

\begin{definition}[Iterative Trails]
	A differential or linear trail $(v^{(0)},\cdots,v^{(r)})$ is iterative if $v^{(0)}=v^{(r)}$.
\end{definition}

\begin{definition}[Elementry Iterative Trails]
    An iterative differential or linear trail $(v^{(0)},\cdots,v^{(r)}=v^{(0)})$ is elementary if $v^{(i)}\neq v^{(j)},\forall i,j\in [0,r-1]$.
\end{definition}

According to the definition, Obiviously a (elementary) circuit in the graph represents an (elementary) iterative trail. Applying Johnson's algorithm in \cite{J75}, we can enumerate all elementary circuits in $G_{F,max\_asn}$. 

Once a short-round iterative trails is found, it can be concatenated to itself arbitary times and forms a long-round trail. For an $r$-round iterative trail with $rd$ and weight $wt$, the weight per round $wt/rd$ can be regarded as an index describing the weight growth of the iterative trail. Then the best iterative trail is the one with the largest weight per round. We claim that the best iterative trail must be an elementary one. 

\subsection{Finding the Best Iterative Cluster}

\begin{example}
	Suppose that we find two elementary iterative trails $p_0=u_0\rightarrow u_1\rightarrow u_0$ and $p_1=u_0\rightarrow u_2 \rightarrow u_0$, then four iterative trails with lenght 4 can be constructed.
\end{example}


Further extending the trail forward and backward, we can an exploitable trail. The method is used in \cite{BS91,BS92,W08} to find exploitable differential trails for DES and PRESENT. 

Our work is motivated by iterative trails. An observation is that the best long-round differential or linear trail contains iterative sub-trails for including but not limited to DES, PRESENT, RECTANGLE and GIFT. We 

Given $G_F$ generated in the previous subsection, theoretically we can list all elementary iterative trails applying Johnson's algorithm in \cite{J75}. 

\subsection{Searching for Iterative Trails\label{subsec:iterative-trails}}

According to the definition of circuits and iterative trails, the elementary iterative trails of $\calE$ can be viewed as elementary circuits in $G_F$. Applying Johnson's algorithm in \cite{J75}, we can list all the elementary circuits in $G_F$. However, if $V_F=\mathbb{F}_2^n$, the size of $V_F$ is $2^n$ which is too large. In order to limit the size of $V_F$, we set a parameter $max\_asn$ which is defined as the maximum active S-boxes that a vertex can has. That is, given an SPN round transformation $F$ and the parameter $max\_asn$, $G_{F,max\_asn}=(V_{F,max\_asn},E_{F,max\_asn})$ is a subgraph of $G_F$. The set of vertices is given by
\[
    V_{F,max\_asn}=\{u|\text{Asn}(u)\leq max\_asn \text{ and } u\in V_F\}
\]
where $\text{Asn}(\cdot)$ is a function returns the number of active S-boxes of its input. The set $E_{F,max\_asn}$ of weighted edges between any two vertices in $V_{F,max\_asn}$ is given according to the method in the last subsection. 

Applying Johnson's algorithm to $G_{F,max\_asn}$, we can obtain a set of elementary circuits
\[
    \{p_{uv}|u=v \text{ and } v^{(i)}\neq v^{(j)},\forall i,j\in[0,r-1]\},
\]
in which each element represent an elementary iterative trails for $F$. 

We extract every vertex that lies in at least one elementary circuit and denote the set of vertices as $V^{IT}_{F,max\_asn}$. $V^{IT}_{F,max\_asn}$ is a subset of $V_{F,max\_asn}$ and it forms a subgraph $G^{IT}_{F,max\_asn}=(V^{IT}_{F,max\_asn},E^{IT}_{F,max\_asn})$ of $G_{F,max\_asn}$. 

\subsection{Finding Differential Trails and Linear Trails\label{subsec:find-trails}}

For any trail based on iterative trails, it can be treated as three parts: the extension backward, the iterative trail and the extension forward. The 14-round differential trail of PRESENT found in \cite{W08} is shown in Table \ref{tab:trail-present}. It is constructed by concatenating a 4-round iterative trail to itself two times and extending both forward and backward by 1 round. Thus the subtrail from round 0 to round 1 is the extension backward part, the one from round 1 to round 13 is the iterative trail part and the one from round 13 to round 14 is the extension forward part. In the following, we try to compute the largest probability a trail of such type can has based on $G^{IT}_{F,max\_asn}$. 

\begin{table}
	\caption{A 14-round differential trail of PRESENT}\label{tab:trail-present}
	\centering
	\begin{tabular}{|c|c|c|}
		\hline
		Round &  Diffference & Prob.\\
		\hline
		0 &  $x_2=7,x_{14}=7$ & \\
		1 &  $x_0=4,x_3=4$ & $2^{-4}$\\
		2 &  $x_0=9,x_8=9$ & $2^{-4}$\\
		3 &  $x_8=1,x_{10}=1$ & $2^{-4}$\\
		4 &  $x_2=5,x_{14}=5$ & $2^{-4}$\\
		5 &  $x_0=4,x_3=4$ & $2^{-6}$\\
		6 &  $x_0=9,x_8=9$ & $2^{-4}$\\
		7 &  $x_8=1,x_{10}=1$ & $2^{-4}$\\
		8 &  $x_2=5,x_{14}=5$ & $2^{-4}$\\
		9 &  $x_0=4,x_3=4$ & $2^{-6}$\\
		10 &  $x_0=9,x_8=9$ & $2^{-4}$\\
		11 &  $x_8=1,x_{10}=1$ & $2^{-4}$\\
		12 &  $x_2=5,x_{14}=5$ & $2^{-4}$\\
		13 &  $x_0=4,x_3=4$ & $2^{-6}$\\
		14 &  $x_0=9,x_8=9$ & $2^{-4}$\\
		\hline
	\end{tabular}
\end{table}

%example

Let $B^F_{u,i}$ be the largest probability (correlation) that an $i$-round differential (linear) trail starting from $u$ can has, $u\in V^{IT}_{F,max\_asn}$. Let $B^{F^{-1}}_{u,i}$ be the largest probability (correlation) that an $i$-round differential (linear) trail ending with $u$ can has, $u\in V^{IT}_{F,max\_asn}$. Given parameters $r^F$ and $r^{F^{-1}}$ which represent the number of rounds to be extended forward and backward, we can obtain $B^F_{u,i},i\in [0,r^F]$ and $B^{F^{-1}}_{u,j},j\in [0,r^{F^{-1}}]$ for each $u\in V^{IT}_{F,max\_asn}$ using Matsui's branch-and-bound depth-first search algorithm.  Based on iterative trails, the largest probability (correlation) $B_r$ that a single $r$-round trail can has is
\[
    B_r=\max\limits_{\substack{p_{u,v}\in E^{IT}_{F,max\_asn}\\r_1+l(p_{u,v})+r_2=r\\0\leq r_1\leq r^{F^{-1}},0\leq r_2\leq r^F}} B^{F^{-1}}_{u,r_1}\times w(p_{u,v})\times B^F_{v,r_2}.
\]
To obtain $B_r$, instead of traversing all $p_{u,v}$, we use dynamic programming. See Algorithm \ref{algo1}. 

\begin{algorithm}
	\caption{Compute $B_r$}
	\label{algo1}
	\begin{algorithmic}[1] %每�?�显示�?�号
		\Require $r$, parameters $r^F\leq r,r^{F^{-1}}\leq r$, the round function $F$, $G^{IT}_{F,max\_asn}$
		\Ensure $B_r$
		\Procedure {ComputeB}{}
		\State /*Phase 1: Compute $B^F$ and $B^{F^{-1}}$*/
		\For{each $u\in V^{IT}_{F,max\_asn}$}
		\State $B^F_{u,0}\leftarrow 1,B^{F^{-1}}_{u,0}\leftarrow 1$
		\EndFor
		\For{each $u\in V^{IT}_{F,max\_asn}$ and $i\leftarrow 1:r^F$}
		\State $BW\leftarrow B^F_{u,i-1}\times\max\limits_{a,b} \Prob^F(a,b)$
		\While{not \Call{Search}{$u,i,0,F,BW$}}
		\State $BW\leftarrow BW\times 2^{-1}$
		\EndWhile
		\State $B^f_{u,i}\leftarrow BW$
		\EndFor
		\For{each $u\in V^{IT}_{F,max\_asn}$ and $i\leftarrow 1:r^{F^{-1}}$}
		\State $BW\leftarrow B^{F^{-1}}_{u,i-1}\times\max\limits_{a,b} \Prob^{F^{-1}}(a,b)$
		\While{not \Call{Search}{$u,i,0,F^{-1},BW$}}
		\State $BW\leftarrow BW\times 2^{-1}$
		\EndWhile
		\State $B^{F^{-1}}_{u,i}\leftarrow BW$
		\EndFor
		\State /*Phase 2: Computation using dynamic programming*/
		\For{each $u\in V^{IT}_{F,max\_asn}$ and $i\in [0,r^F]$}
		\State $\overline{B^F_{u,i}}\leftarrow B^f_{u,i}$
		\EndFor
		\For{each $u\in V^{IT}_{F,max\_asn}$ and $i\leftarrow (r^F+1):r$}
		\State $\overline{B^F_{u,i}}\leftarrow\max\limits_{v\in V^{IT}_{F,max\_asn}} w(u\rightarrow v)\times\overline{B^F_{v,i-1}}$
		\EndFor
		\State /*Phase 3: Compute $B_r$*/
		\State $B_r\leftarrow\max\limits_{\substack{i+j=r\\u\in V^{IT}_{F,max\_asn}}} B^{F^{-1}}_{u,i}\times \overline{B^F_{u,j}}$
		\EndProcedure
		
		\Function{Search}{$u,j,w,rf,BW$}
		\State $found\leftarrow\false$
		\For{each $v$ such that $\Prob^{rf}(u,v)\geq BW\div w\div j\times\max\limits_{a,b}\Prob^{rf}(a,b)$}
		\State $w'\leftarrow w\times \Prob^{rf}(u,v)$
		\If{$j=0$}
		\If{$w'>=BW$}
		\State$BW\leftarrow w'$, $found\leftarrow \true$
		\EndIf
		\Else
		\State $found\leftarrow found$ or \Call{Search}{$v,j-1,w',rf,BW$}
		\EndIf
		\EndFor
		\State \Return{$found$}
		\EndFunction
	\end{algorithmic}
\end{algorithm}

\subsection{Finding Differentials and Linear Hulls\label{subsec:find-clusters}}

Let $w^F_{u,v,i}$ be the probability (correlation) of the $i$-round differential (linear hull) $(u,v)$ where $u\in V^{IT}_{F,max\_asn}$. Let $w^{F^{-1}}_{u,v,i}$ be the probability (correlation) of the $i$-round differential (linear hull) $(v,u)$ where $u\in V^{IT}_{F,max\_asn}$. Given parameters $r^F,r^{F^{-1}}$ which represent the maximum number of rounds to be extended forward and backward and parameters $w^F,w^{F^{-1}}$ which heuristically bounds the probability (correlation) of the extension subtrails. To compute $w^F_{u,v,i}$ and $w^{F^{-1}}_{u,v,i}$, We collect as many extension subtrails as possible using Matsui's branch-and-bound depth-first algorithm. Note that during traversing extension subtrails, we abandon any subtrail that contains any value in $V^{IT}_{F,max\_asn}$ to avoid duplicate trails in the next step.

In a graph, a hull $h_{u,v}$ is the set of all paths from $u$ to $v$. Here, we define a hull $h_{u,v,r}$ as the set of all paths $p_{u,v}$ with $l(p_{u,v})=r$. Then its weight is
\[
	w(h_{u,v,r})=\sum\limits_{l(p_{u,v})=r} w(p_{u,v}).
\]
$w(h_{u,v,r})$ can be computed using dynamic programming.

Based on iterative trails, the largest probability (correlation) $BC_r$ that a $r$-round differential (linear hull) can has is
\[
	BC_r=\max\limits_{\substack{x,y\in \mathbb{F}_2^n\\u,v\in V^{IT}_{F,max\_asn}\\r_1+r_2+r_3=r\\0\leq r_1\leq r^{F^{-1}},0\leq r_3\leq r^F}} w^{F^{-1}}_{x,u,r_1}\times w(h_{u,v,r_2})\times w^F_{v,y,r_3}.
\]
See Algorithm \ref{algo2}. 

\begin{algorithm}
	\caption{Compute $BC_r$}
	\label{algo2}
	\begin{algorithmic}[1] %每�?�显示�?�号
		\Require $r$, parameters $r^F\leq r,r^{F^{-1}}\leq r$, parameters $wb^F,wb^{F^{-1}}$, the round function $F$, $G^{IT}_{F,max\_asn}$
		\Ensure $BC_r$
		\Procedure {ComputeBC}{}
		\State /*Phase 1: Compute $w^F$ and $w^{F^{-1}}$*/
		\For{each $u\in V^{IT}_{F,max\_asn}$}
		\State $w^F_{u,u,0}\leftarrow 1,w^b_{u,u,0}\leftarrow 1$
		\EndFor
		\For{each $u\in V^{IT}_{F,max\_asn}$ and $i\leftarrow 1:r^F$}
		\State \Call{Collect}{$u,u,r^F,1,F$}
		\EndFor
		\For{each $u\in V^{IT}_{F,max\_asn}$ and $i\leftarrow 1:r^{F^{-1}}$}
		\State \Call{Collect}{$u,u,r^{F^{-1}},1,F^{-1}$}
		\EndFor
		\State /*Phase 1: Compute $w(h_{u,v,i})$ using dynamic programming*/
		\For{each $u,v\in V^{IT}_{F,max\_asn}$}
		\State $w(h_{u,v,0})\leftarrow 1$
		\EndFor
		\For{$i\leftarrow 1:r$}
		\For{each $u,v\in V^{IT}_{F,max\_asn}$}
		\State $w(h_{u,v,i})\leftarrow \sum\limits_x w(h_{u,x,i-1})\times w(x\rightarrow v)$
		\EndFor
		\EndFor
		\State /*Phase 1: Compute $BC_r$*/
		\For{each possible first subscript index $x$ of $w^{F^{-1}}$}
		\For{each possible second subscript index $y$ of $w^F$}
		\State $BC_{y,r}\leftarrow \sum\limits_{\substack{r_1+r_2+r_3=r\\u,v\in V^{IT}_{F,max\_asn}}}w^{F^{-1}}_{x,u,r_1}\times w(h_{u,v,r_2})\times w^F_{v,y,r_3}$
		\If{$BC_{y,r}>BC_r$}
		\State $BC_r\leftarrow BC_{y,r}$
		\EndIf
		\EndFor
		\EndFor
		\EndProcedure
		
		\Procedure{Collect}{$s,x,j,w,rf$}
		\For{each $y$ such that $\Prob^{rf}(x,y)\geq wb^{rf}\div w\div (j\times\max\limits_{a,b}\Prob^{rf}(a,b))$}
		\State $w'\leftarrow w+\Prob^{rf}(x,y)$
		\If{$w^{rf}_{s,y,r_f-j}$ exists}
		\State $w^{rf}_{s,y,r_{rf}-j}\leftarrow w^{rf}_{s,y,r^{r43}-j}\times w'$
		\Else
		\State $w^{rf}_{s,y,r_{rf}-j}\leftarrow w'$
		\EndIf
		\If{$j\neq 0$}
		\State \Call{Collect}{$s,y,j-1,w',rf$}
		\EndIf
		\EndFor

		\EndProcedure
	\end{algorithmic}
\end{algorithm}
%In another way, a weighted 2-stage graph $G^1_F=(V^1_F,E^1_F)$ can also be generated to describe the 1-round differentials or linear hulls. $G^1_F$ has 2 stages $S_0$ and $S_1$ each being $\mathbb{F}_2^n$. For $u\in S_0$ and $v\in S_1$, the weight of an edge $w^1(u\rightarrow v)$ is defined the same way as $l(u\rightarrow v)$ in $G_F$ is.

%If $S_0=S_1$, then $G^1_F$ can be iteratively concatenated to itself. 
