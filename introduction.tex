\section{Introduction}

Differential cryptanalysis \cite{biham1991differential} and linear cryptanalysis \cite{matsui1993linear} are two of the most powerful attacks against modern symmetric-key primitives. Both kinds of cryptanalysis managed to break the full-round DES at that time. With the increasing number of symmetric-key primitives, every well-designed symmetric-key primitive must resist against differential and linear cryptanalysis in the first place. The development of cryptanalysis also drives the design of symmetric-key primitives, e.g. the wide trail design strategy provides provable security of AES \cite{daemen2002design} against differential and linear cryptanalysis. 

To conduct a differential or linear attack, an adversary expects to find exploitable differential or linear distinguishers. Instead of deducing  exploitable differential and linear distinguishers manually, cryptanalysts and cipher designers develop computer-aided methods of searching for them which are also called automatic cryptanalysis. The two main kinds of automatic search tools are dedicated search algorithms \cite{matsui1994correlation,ohta1995improving,aoki1997best,abdelraheem2012estimating,bao2014speeding,dobraunig2015heuristic} and methods based on mathematical solvers such as MILP solvers \cite{mouha2011differential,sun2013automatic,sun2014automatic,sun2014towards,zhou2019improving}. To evaluate the advantage of a differential and linear distinguisher more precisely, the clustering effect needs to be taken into consideration. One approach is to enumerate differential (linear) characteristics with the same input and output differences (masks) and sum their differential probabilities (correlations). This approach is essentially a depth-first search. Another approach is using transition matrices \cite{abdelraheem2012estimating} or a multistage graph \cite{EPRINT:HalVej18}. These two methods are essentially breadth-first search. 

When investigating the long-round best differential and linear characteristics of RECTANGLE, we find that they always contain iterative sub-characteristics. In this paper, we focus on finding differential and linear iterative distinguishers. Once a short-round iterative distinguisher with large differential probability or correlation amplitude is obtained, it can be concatenated to itself arbitrary times, forming an exploitable long-round distinguisher. In \cite{biham1991differential,biham1992differential,knudsen1992iterative} the authors managed to break the full-round DES using short-round differential iterative characteristics. In \cite{wang2008differential}, for PRESENT, the author gave a 3-round iterative diffrential characteristic and built a 14-round differential characteristic based on it. In \cite{liu2019iterative}, the author attacked TRIFLE using a 1-round differential iterative characteristic obtained using the MILP method. 

In this papar, we develop a new method searching for iterative distinguishers using a graph approach. We require that our method can find multi-round iterative distinguishers. And we require that our method can find iterative distinguishers for a symmetric-key primitive with operation components having a symmetry, e.g. for RECTANGLE. 
%To solve whether there is better iterative distinguishers, we try to develop a new method searching iterative distinguishers. 

\subsubsection{Our Contribution}
\begin{enumerate}
    \item For SPN symmetric-key primitives, we propose our method of finding the best differential or linear iterative characteristic using a graph approach. For primitives with rotational symmetry like RECTANGLE and KNOT, we instead search for \textit{rotational} iterative characteristics. We further propose our method of find the best iterative differential or linear hull. It's shown that better iterative distinguishers can be found by considering the clustering effect. 
    %the two results, we find that for some primitives, their best single iterative characteristics are not best iterative distinguishers. 
    
    %, by which we can find better iterative distinguishers. For most ciphers investigated, the running time of the above two algorithms is a few minutes on a stardard PC. 
    \item We visualize the generated graph containing iterative characteristics. We find that the clustering effect appears if different differential (linear) iterative characteristics have any equivalent difference (mask). Through the visualization, we hope to reveal some insights for cipher designers. 
    \item We propose our method of finding the best differential or linear hull containing iterative sub-characteristics. We apply our methods to 6 SPN symmetric-key primitives including KNOT, RECTANGLE, GIFT, PRESENT, PUFFIN and TRIFLE. For KNOT, PUFFIN and TRIFLE, we find better differential and linear distinguishers. 
    \item For KNOT, a round 2 candidate of the NIST lightweight cryptography standardization process, we mount different differential or linear attacks according to the different phases targeted and show that KNOT has enough security margin against differential and linear cryptanalysis considering the clustering effect. 
\end{enumerate}

%In 1990, Biham and Shamir \cite{BS92} introduced differential cryptanalysis and successfully attacked the full-round DES. In 1991, they \cite{BS92} improved the attack with $2^{47}$ chosen plaintexts. In 1993, Matsui \cite{M93} introduced linear cryptanalysis and succeeded in breaking DES with $2^{47}$ known plaintexts. In 1994, Matsui \cite{M94_1} improved the data complexity to $2^{43}$. Cryptanalysis also drives the design of ciphers in return. In 2001, Rijmen and Daemon \cite{DR01} proposed the wide trail design strategy, providing provable security against DC and LC for AES winner Rijndael \cite{DR98}. With increasing number of symmetric cryptographic primitives emerging, every well-designed block cipher must resist against DC and LC in the first place. To conduct the differential or linear attack, an adversary expects to find exploitable differential or linear distinguishers. Usually, the probability of the best differential trail and the correlation (or bias) of the best linear trail are respectively used as the indices to the resistance against DC and LC. The two main kinds of the automatic search tools for the best differential and linear trails are dedicated tree search algorithms \cite{M94_2,OMA95,AKM97,BZL14} and mathematical-solver-based methods \cite{MWG12,SHW14-1,SHW14-2,ZZDX19}. In this article, we focus on the dedicated search algorithms. 

%In 1994, Matsui proposed a branch-and-bound depth-first tree search algorithm for searching the best differential or linear trail of DES \cite{M94_2}. In 1995, Moriai et al. \cite{OMA95} introduced the concept of search pattern to reduce unnecessary search candidates, which improves the performance of searching the best trail of FEAL. In 1997, Aoki et al. \cite{AKM97} further improved the performance by using a pre-search for impossible search patterns. In 2014, Bao et al. \cite{BZL14} proposed new strategies including starting from the narrowest point, concretizing and grouping search patterns and trailing in minimal changes order, achieving significant efficiency improvement on NOEKEON and Spongent. Dobraunig et al. \cite{DEM15} proposed a stack-based depth-first search algorithm characterizing in guessing sbox by sbox or bit by bit instead of round by round in Matsui's algorithm. Hall-Andersen et al. \cite{HV18} modeled the trail search problem as a graph problem and managed to obtain results on clustering effect for many ciphers. 

%Another line of research is modelling the differential or linear propagation and solve the model using mathematical tools. In 2011, Wu et al. modeled block ciphers using integer programming\cite{WW11}. Works that modelling using mixed integer linear programming (MILP) include but not limited to \cite{MWG12}. Works that modelling using SAT/SMT include but not limited to []. Works that modelling using constraint programming (CP) include but not limited to [].

%Besides automatically searching the best differential or linear trail, iterative trails are used to construct long-round significant trails in order to efficiently obtain exploitable trails for cryptanalysis. Iterative trails refer to trails that have the same input and output difference (or mask) and thus they can concatenate to themselves. Biham and Shamir \cite{BS91,BS92} used iterative differential characteristics to cryptanalyze DES with an arbitrary number of rounds. Knudsen \cite{K92} examined the 2 iterative characteristics found in \cite{BS91,BS92} and found additional 3 iterative characteristics for DES. Wang et al. \cite{W08} found a 4-round iterative differential characteristic for PRESENT by which a 14-round significant differential characteristic is constructed. 

%If the iterative structure is not too complicated, our method is able to find all iterative trails in seconds. Using the iterative structure, We can further determine the weight of the best $n$-round trails constructed by iterative trails. 

%\begin{table}
%	\caption{results on weights of best trails and clusters}\label{tab:sum}
%	\centering
%	\begin{tabular}{|c|c|c|c|c|c|}
%        \hline
%        cipher&target&round&best weight&method&time\\
%        \hline
%        \multirow{9}{*}{KNOT-perm-256}
%        &differential trail   &14     &71         &\cite{ZDY19}           &-\\
%        &differential cluster  &14     &$\leq$ 65.4493    &Sec. \ref{sec:para3}   &511s\\
%        &differential trail   &48     &252         &\cite{BZL14}           &2.87h\\
%        &differential trail   &48     &$\leq$ 252        &Sec. \ref{sec:para2}   &0.1s\\
%        \cline{2-6}
%        &linear trail         &10     &23         &\cite{ZDY19}            &-\\
%        &linear cluster        &10     &$\leq$ 21.8301    &Sec. \ref{sec:para3}   &7509s\\
%        &linear trail         &39     &110         &\cite{BZL14}            &344.01h\\
%        &linear trail         &39     &$\leq$ 110        &Sec. \ref{sec:para2}   &0.6s\\
%        &linear trail         &44     &$\leq$ 125        &Sec. \ref{sec:para2}   &0.6s\\
%        \hline
%        \multirow{4}{*}{PRESENT}
%        &differential trail   &14     &62         &\cite{ZZDX19}          &-\\
%        &differential trail   &14     &$\leq$ 62         &Sec. \ref{sec:para2}   &0.6s\\
%        \cline{2-6}
%        &linear trail     &16     &30         &\cite{ZZDX19}          &-\\
%        &linear trail      &16     &$\leq$ 30         &Sec. \ref{sec:para2}   &0.1s\\
%        \hline
%        \multirow{5}{*}{GIFT-64}
%        &differential trail   &13     &62         &\cite{ZZDX19}          &-\\
%        &differential trail   &13     &$\leq$ 62         &Sec. \ref{sec:para2}   &0.3s\\
%        &differential cluster   &13     &$\leq$ 60.415     &Sec. \ref{sec:para3}   &414s\\
%        \cline{2-6}
%        &linear trail         &12     &31         &\cite{ZZDX19}          &-\\
%        &linear trail         &12     &$\leq$ 32         &Sec. \ref{sec:para2}   &0.2s\\
%        \hline
%        \multirow{6}{*}{RECTANGLE}
%        &differential trail   &14     &61         &\cite{ZBL15}           &-\\
%        &differential trail   &14     &$\leq$ 61         &Sec. \ref{sec:para2}   &1.2s\\
%        &differential cluster   &15     &$\leq$ 63.7857    &Sec. \ref{sec:para3}   &1329s\\
%        \cline{2-6}
%        &linear trail         &13     &31         &\cite{ZBL15}            &-\\
%        &linear trail         &13     &$\leq$ 31         &Sec. \ref{sec:para2}   &0.2s\\
%        &linear cluster ($c^2$)  &13     &$\leq$ 59.3561    &Sec. \ref{sec:para3}   &2481s\\
%        \hline
%	\end{tabular}
%\end{table}

%\subsubsection{Our Contribution}
%\begin{enumerate}
%    \item We propose a new automatic tool searching for iterative trails applying to permutations or block ciphers based on the SPN structure. Restricting the number of active S-boxes of difference or mask values, we model the 1-round differential or linear propagations as a weighted directed graph. Using an algorithm finding all elementary circuits in a directed graph \cite{J75}, we can enumerate all iterative trails and thus obtain the best one. The found iterative trails form a subgraph and can be visualized. We further take the effect of the combination of iterative trails into consideration and propose a method for finding iterative hulls which are clusters of iterative trails. 
%    \item Using the subgraph containing iterative trails and by extending trails from the vertices in the subgraph, we propose an algorithm to estimate the probabilities of differentials and correlations of linear hulls. 
%    \item For PRESENT, GIFT-64 and RECTANGLE, the results of EDP and ELP are close to the results in \cite{HV18} while our method costs much less time. The results implies that (1) the combination of iterative trails enhances differential and linear propagations and (2) the good differentials and linear hulls are dominated by iterative trails for these ciphers.
%    \item The inner permutations of KNOT, which is a round 2 candidate of the NIST lightweight cryptography standardization process, are inheritors of RECTANGLE. For 256-bit KNOT permutation, we can find good differentials up to 52 rounds and good linear hulls up to 51 rounds, the number of rounds increasing by 4 rounds and 12 rounds respectively compared to the result obtained by only considering single trails in the original specification document. 
%    \item For arbitrary-round the 256-bit KNOT permutation, PRESENT, GIFT-64 and RECTANGLE, we obtain the weight of the best trails constructed by iterative trails, which is an upper bound of the weight of the best trails. The upper bound is tight when the round number is large. Our method finishes in several seconds. Comparing our results for the 256-bit KNOT permutation with the results obtained using the method proposed by Bao et al. \cite{BZL14}, we find that the upper bound is exactly the best weight and the time our method consumes is negligible. See Table \ref{tab:sum}. 
%    \item For 256-bit KNOT permutation, GIFT-64 and RECTANGLE, we obtain the weight of the best differential or linear propagation composed of significant trails constructed by iterative trails. Because GIFT-64 and RECTANGLE are keyed block ciphers, the correlation potential ($\sum c^2$) is computed for a cluster of linear trails, while because KNOT permutations are permutations without key xor, the correlation contribution ($\sum c$) is computed for a cluster of linear trails. Except for linear cryptanalysis of GIFT-64, we obtain better weight of the best differential or linear propagation than the weight of the best single trail. See Table \ref{tab:sum}. 
%    \item For different versions of KNOT-AEAD, we obtain the upper bound of the weight of the best trails with input and output differences or masks constrained depending on different attack scenarios. The differential and linear attacks proposed are universally applicable for any AEAD scheme based on the Duplex mode of operations. 
%\end{enumerate}

\subsubsection{Organization}
The paper is organized as follows. In Section \ref{sec:pre}, we introduce notations and concepts. In Section \ref{sec:method_ite}, we present our method of finding differential or linear iterative distinguishers using a graph approach. Further we present our method of finding the best differential or linear hull containing iterative sub-characteristics. In Section \ref{sec:result}, results for 6 SPN symmetric-key primitives are shown. In Section \ref{sec:conclusion}, we conclude our paper.
%Section \ref{sec:method} gives the method modelling the problem of searching for iterative trails to a graph problem and the algorithm estimating the probability (correlation) of differentials (linear hulls). Section \ref{sec:improvements} proposes several improvement techniques. Section \ref{sec:experiment} shows experimental results. In Section \ref{sec:conclusion}, we conlude our work.
