% This is samplepaper.tex, a sample chapter demonstrating the
% LLNCS macro package for Springer Computer Science proceedings;
% Version 2.20 of 2017/10/04
%
\documentclass[runningheads]{llncs}
%\pagestyle{plain}
%
\usepackage{graphicx}
% Used for displaying a sample figure. If possible, figure files should
% be included in EPS format.
%
% If you use the hyperref package, please uncomment the following line
% to display URLs in blue roman font according to Springer's eBook style:
% \renewcommand\UrlFont{\color{blue}\rmfamily}

\usepackage{amssymb}
\usepackage[super]{nth}

\usepackage{algorithm}
\usepackage{algorithmicx}
\usepackage{algpseudocode}
\usepackage{multirow}
\usepackage{amsmath}
\usepackage{extarrows}
\usepackage{pgf}

\newif\ifsans
\newif\iftext
\newif\ifdetails
\usepackage{tikz}
\tikzset{sparsam/.style={inner sep=1pt}}
\tikzset{bitwidth/.style={above=-1pt, font=\tiny}}
\tikzset{next/.style={->, >=latex}}
\usetikzlibrary{crypto.symbols,decorations.pathreplacing}
%\usetikzlibrary{arrows,shapes,snakes,automata,backgrounds,petri}

%usepackage{algpseudocode}
%\algdef{SE}[DOWHILE]{Do}{doWhile}{\algorithmicdo}[1]{\algorithmicwhile\ #1}%

\newcommand{\ith}{i\textsuperscript{th} }
\newcommand{\Prob}{\text{Prob}}
\newcommand{\Cor}{\text{Cor}}
\newcommand{\EDP}{\text{EDP}}
\newcommand{\ELP}{\text{ELP}}
\newcommand{\Exp}{\text{Exp}}
\newcommand{\calE}{\mathcal{E}}
\newcommand{\calP}{\mathcal{P}}
\newcommand{\calH}{\mathcal{H}}
\newcommand{\calR}{\mathcal{R}}
\newcommand{\calA}{\mathcal{A}}
\newcommand{\calS}{\mathcal{S}}
\newcommand{\calL}{\mathcal{L}}
\newcommand{\bbF}{\mathbb{F}}
\newcommand{\bbP}{\mathbb{P}}
\newcommand{\true}{\text{true}}
\newcommand{\false}{\text{false}}
\newcommand{\rotr}{\text{rotr}}
\newcommand{\rotl}{\text{rotl}}
\newcommand{\Asn}{\text{Asn}}
\makeatletter
\newenvironment{breakablealgorithm}
  {% \begin{breakablealgorithm}
   \begin{center}
     \refstepcounter{algorithm}% New algorithm
     \hrule height.8pt depth0pt \kern2pt% \@fs@pre for \@fs@ruled
     \renewcommand{\caption}[2][\relax]{% Make a new \caption
       {\raggedright\textbf{\ALG@name~\thealgorithm} ##2\par}%
       \ifx\relax##1\relax % #1 is \relax
         \addcontentsline{loa}{algorithm}{\protect\numberline{\thealgorithm}##2}%
       \else % #1 is not \relax
         \addcontentsline{loa}{algorithm}{\protect\numberline{\thealgorithm}##1}%
       \fi
       \kern2pt\hrule\kern2pt
     }
  }{% \end{breakablealgorithm}
     \kern2pt\hrule\relax% \@fs@post for \@fs@ruled
   \end{center}
  }
\makeatother

\begin{document}
%
\title{An Automatic Search Tool for Iterative Trails and its Application to estimation of differentials and linear hulls 
%\thanks{Supported by organization x.}
}
%
\titlerunning{An Automatic Search Tool for Iterative Trails}
% If the paper title is too long for the running head, you can set
% an abbreviated paper title here
%
\author{Tianyou Ding\inst{1,2}
%\orcidID{0000-1111-2222-3333} 
\and Wentao Zhang\inst{1,2}
\and Chunning Zhou\inst{1,2}
\and Fulei Ji\inst{1,2}
}
%
\authorrunning{Tianyou Ding et al.}
% First names are abbreviated in the running head.
% If there are more than two authors, 'et al.' is used.
%
\institute{State Key Laboratory of Information Security, Institute of Information Engineering, Chinese Academy of Sciences, Beijing, China\\
\email{\{dingtianyou, zhangwentao, zhouchunning, jifulei\}@iie.ac.cn}
\and School of Cyber Security, University of Chinese Academy of Sciences, Beijing, China
}
%
\maketitle              % typeset the header of the contribution
%
\begin{abstract}

%The design and cryptanalysis are the both sides from which we look at symmetric-key primitives. If a symmetric-key primitive is broken by a kind of cryptanalysis, it's definitely insecure. If a designer claims a symmetric-key primitive to be secure, one should demonstrate that the primitive resists against all known attacks. 

%%Differential and linear cryptanalysis are two of the most important kinds of cryptanalysis for symmetric-key primitives. To conduct a successful differential (linear) cryptanalysis, a differential (linear) distinguisher with significant differential probability (linear correlation) is needed. 

%%For some lightweight symmetric-key primitives, their best trails contain iterative trails. In this work, We present an algorithm searching for iterative trails by modelling it into a graph problem. We find that the clustering effect of iterative trails may enhance differential and linear propagations. If iterative trails found, we propose a method to estimate the probability (correlation) of differentials (linear hulls). 

%%We apply our methods to 256-bit KNOT permutation, ASCON permutation, PRESENT, GIFT-64 and RECTANGLE. The found iterative trails form a directed graph and are visualized. If iterative trails are found, our method can efficiently find good differentials and linear hulls. Our results imply that for the primitives we experiment on with bit permutations as their linear layers, the best differentials and linear hulls are dominated by iterative trails. 

%%For KNOT, which is a round 2 candidate of the NIST lightweight cryptography standardization process, we find differential and linear distinguishers with certain restrictions and truncations on the input and output differences or masks, which adapt the distinguishers to certain differential or linear attacks on the AEAD and Hash schemes. 

%An upper bound of the weight of the best trail with arbitrary rounds is obtained in several seconds. When the number of rounds is large, it's illustrated that the upper bound coincide with the best weight or is very tight. An upper bound of the weight of the best differential or linear propagation is also obtained by considering the clustering effect. For 14-round 256-bit KNOT permutation, 13-round GIFT-64 and 15-round RECTANGLE, better differential distinguishers than the best differential trails are obtained. For 10-round 256-bit KNOT permutation and 13-round RECTANGLE, better linear distinguishers than the best linear trails are obtained. Additionally, by constraining the input and output differences and masks, results for KNOT-AEAD fitting certain differential and linear attack scenarios in Duplex mode of operations are obtained. 

\keywords{Differential cryptanalysis \and Linear cryptanalysis \and differentials \and Linear hulls \and Iterative trails \and Graph theory}
\end{abstract}

%\section{Introduction}

Differential cryptanalysis \cite{biham1991differential} and linear cryptanalysis \cite{matsui1993linear} are two of the most powerful attacks against modern symmetric-key primitives. Both kinds of cryptanalysis managed to break the full-round DES at that time. With the increasing number of symmetric-key primitives, every well-designed symmetric-key primitive must resist against differential and linear cryptanalysis in the first place. The development of cryptanalysis also drives the design of symmetric-key primitives, e.g. the wide trail design strategy provides provable security of AES \cite{daemen2002design} against differential and linear cryptanalysis. 

To conduct a differential or linear attack, an adversary expects to find exploitable differential or linear distinguishers. Instead of deducing  exploitable differential and linear distinguishers manually, cryptanalysts and cipher designers develop computer-aided methods of searching for them which are also called automatic cryptanalysis. The two main kinds of automatic search tools are dedicated search algorithms \cite{matsui1994correlation,ohta1995improving,aoki1997best,abdelraheem2012estimating,bao2014speeding,dobraunig2015heuristic} and methods based on mathematical solvers such as MILP solvers \cite{mouha2011differential,sun2013automatic,sun2014automatic,sun2014towards,zhou2019improving}. To evaluate the advantage of a differential and linear distinguisher more precisely, the clustering effect needs to be taken into consideration. One approach is to enumerate differential (linear) characteristics with the same input and output differences (masks) and sum their differential probabilities (correlations). This approach is essentially a depth-first search. Another approach is using transition matrices \cite{abdelraheem2012estimating} or a multistage graph \cite{EPRINT:HalVej18}. These two methods are essentially breadth-first search. 

When investigating the long-round best differential and linear characteristics of RECTANGLE, we find that they always contain iterative sub-characteristics. In this paper, we focus on finding differential and linear iterative distinguishers. Once a short-round iterative distinguisher with large differential probability or correlation amplitude is obtained, it can be concatenated to itself arbitrary times, forming an exploitable long-round distinguisher. In \cite{biham1991differential,biham1992differential,knudsen1992iterative} the authors managed to break the full-round DES using short-round differential iterative characteristics. In \cite{wang2008differential}, for PRESENT, the author gave a 3-round iterative diffrential characteristic and built a 14-round differential characteristic based on it. In \cite{liu2019iterative}, the author attacked TRIFLE using a 1-round differential iterative characteristic obtained using the MILP method. 

In this papar, we develop a new method searching for iterative distinguishers using a graph approach. We require that our method can find multi-round iterative distinguishers. And we require that our method can find iterative distinguishers for a symmetric-key primitive with operation components having a symmetry, e.g. for RECTANGLE. 
%To solve whether there is better iterative distinguishers, we try to develop a new method searching iterative distinguishers. 

\subsubsection{Our Contribution}
\begin{enumerate}
    \item For SPN symmetric-key primitives, we propose our method of finding the best differential or linear iterative characteristic using a graph approach. For primitives with rotational symmetry like RECTANGLE and KNOT, we instead search for \textit{rotational} iterative characteristics. We further propose our method of find the best iterative differential or linear hull. It's shown that better iterative distinguishers can be found by considering the clustering effect. 
    %the two results, we find that for some primitives, their best single iterative characteristics are not best iterative distinguishers. 
    
    %, by which we can find better iterative distinguishers. For most ciphers investigated, the running time of the above two algorithms is a few minutes on a stardard PC. 
    \item We visualize the generated graph containing iterative characteristics. We find that the clustering effect appears if different differential (linear) iterative characteristics have any equivalent difference (mask). Through the visualization, we hope to reveal some insights for cipher designers. 
    \item We propose our method of finding the best differential or linear hull containing iterative sub-characteristics. We apply our methods to 6 SPN symmetric-key primitives including KNOT, RECTANGLE, GIFT, PRESENT, PUFFIN and TRIFLE. For KNOT, PUFFIN and TRIFLE, we find better differential and linear distinguishers. 
    \item For KNOT, a round 2 candidate of the NIST lightweight cryptography standardization process, we mount different differential or linear attacks according to the different phases targeted and show that KNOT has enough security margin against differential and linear cryptanalysis considering the clustering effect. 
\end{enumerate}

%In 1990, Biham and Shamir \cite{BS92} introduced differential cryptanalysis and successfully attacked the full-round DES. In 1991, they \cite{BS92} improved the attack with $2^{47}$ chosen plaintexts. In 1993, Matsui \cite{M93} introduced linear cryptanalysis and succeeded in breaking DES with $2^{47}$ known plaintexts. In 1994, Matsui \cite{M94_1} improved the data complexity to $2^{43}$. Cryptanalysis also drives the design of ciphers in return. In 2001, Rijmen and Daemon \cite{DR01} proposed the wide trail design strategy, providing provable security against DC and LC for AES winner Rijndael \cite{DR98}. With increasing number of symmetric cryptographic primitives emerging, every well-designed block cipher must resist against DC and LC in the first place. To conduct the differential or linear attack, an adversary expects to find exploitable differential or linear distinguishers. Usually, the probability of the best differential trail and the correlation (or bias) of the best linear trail are respectively used as the indices to the resistance against DC and LC. The two main kinds of the automatic search tools for the best differential and linear trails are dedicated tree search algorithms \cite{M94_2,OMA95,AKM97,BZL14} and mathematical-solver-based methods \cite{MWG12,SHW14-1,SHW14-2,ZZDX19}. In this article, we focus on the dedicated search algorithms. 

%In 1994, Matsui proposed a branch-and-bound depth-first tree search algorithm for searching the best differential or linear trail of DES \cite{M94_2}. In 1995, Moriai et al. \cite{OMA95} introduced the concept of search pattern to reduce unnecessary search candidates, which improves the performance of searching the best trail of FEAL. In 1997, Aoki et al. \cite{AKM97} further improved the performance by using a pre-search for impossible search patterns. In 2014, Bao et al. \cite{BZL14} proposed new strategies including starting from the narrowest point, concretizing and grouping search patterns and trailing in minimal changes order, achieving significant efficiency improvement on NOEKEON and Spongent. Dobraunig et al. \cite{DEM15} proposed a stack-based depth-first search algorithm characterizing in guessing sbox by sbox or bit by bit instead of round by round in Matsui's algorithm. Hall-Andersen et al. \cite{HV18} modeled the trail search problem as a graph problem and managed to obtain results on clustering effect for many ciphers. 

%Another line of research is modelling the differential or linear propagation and solve the model using mathematical tools. In 2011, Wu et al. modeled block ciphers using integer programming\cite{WW11}. Works that modelling using mixed integer linear programming (MILP) include but not limited to \cite{MWG12}. Works that modelling using SAT/SMT include but not limited to []. Works that modelling using constraint programming (CP) include but not limited to [].

%Besides automatically searching the best differential or linear trail, iterative trails are used to construct long-round significant trails in order to efficiently obtain exploitable trails for cryptanalysis. Iterative trails refer to trails that have the same input and output difference (or mask) and thus they can concatenate to themselves. Biham and Shamir \cite{BS91,BS92} used iterative differential characteristics to cryptanalyze DES with an arbitrary number of rounds. Knudsen \cite{K92} examined the 2 iterative characteristics found in \cite{BS91,BS92} and found additional 3 iterative characteristics for DES. Wang et al. \cite{W08} found a 4-round iterative differential characteristic for PRESENT by which a 14-round significant differential characteristic is constructed. 

%If the iterative structure is not too complicated, our method is able to find all iterative trails in seconds. Using the iterative structure, We can further determine the weight of the best $n$-round trails constructed by iterative trails. 

%\begin{table}
%	\caption{results on weights of best trails and clusters}\label{tab:sum}
%	\centering
%	\begin{tabular}{|c|c|c|c|c|c|}
%        \hline
%        cipher&target&round&best weight&method&time\\
%        \hline
%        \multirow{9}{*}{KNOT-perm-256}
%        &differential trail   &14     &71         &\cite{ZDY19}           &-\\
%        &differential cluster  &14     &$\leq$ 65.4493    &Sec. \ref{sec:para3}   &511s\\
%        &differential trail   &48     &252         &\cite{BZL14}           &2.87h\\
%        &differential trail   &48     &$\leq$ 252        &Sec. \ref{sec:para2}   &0.1s\\
%        \cline{2-6}
%        &linear trail         &10     &23         &\cite{ZDY19}            &-\\
%        &linear cluster        &10     &$\leq$ 21.8301    &Sec. \ref{sec:para3}   &7509s\\
%        &linear trail         &39     &110         &\cite{BZL14}            &344.01h\\
%        &linear trail         &39     &$\leq$ 110        &Sec. \ref{sec:para2}   &0.6s\\
%        &linear trail         &44     &$\leq$ 125        &Sec. \ref{sec:para2}   &0.6s\\
%        \hline
%        \multirow{4}{*}{PRESENT}
%        &differential trail   &14     &62         &\cite{ZZDX19}          &-\\
%        &differential trail   &14     &$\leq$ 62         &Sec. \ref{sec:para2}   &0.6s\\
%        \cline{2-6}
%        &linear trail     &16     &30         &\cite{ZZDX19}          &-\\
%        &linear trail      &16     &$\leq$ 30         &Sec. \ref{sec:para2}   &0.1s\\
%        \hline
%        \multirow{5}{*}{GIFT-64}
%        &differential trail   &13     &62         &\cite{ZZDX19}          &-\\
%        &differential trail   &13     &$\leq$ 62         &Sec. \ref{sec:para2}   &0.3s\\
%        &differential cluster   &13     &$\leq$ 60.415     &Sec. \ref{sec:para3}   &414s\\
%        \cline{2-6}
%        &linear trail         &12     &31         &\cite{ZZDX19}          &-\\
%        &linear trail         &12     &$\leq$ 32         &Sec. \ref{sec:para2}   &0.2s\\
%        \hline
%        \multirow{6}{*}{RECTANGLE}
%        &differential trail   &14     &61         &\cite{ZBL15}           &-\\
%        &differential trail   &14     &$\leq$ 61         &Sec. \ref{sec:para2}   &1.2s\\
%        &differential cluster   &15     &$\leq$ 63.7857    &Sec. \ref{sec:para3}   &1329s\\
%        \cline{2-6}
%        &linear trail         &13     &31         &\cite{ZBL15}            &-\\
%        &linear trail         &13     &$\leq$ 31         &Sec. \ref{sec:para2}   &0.2s\\
%        &linear cluster ($c^2$)  &13     &$\leq$ 59.3561    &Sec. \ref{sec:para3}   &2481s\\
%        \hline
%	\end{tabular}
%\end{table}

%\subsubsection{Our Contribution}
%\begin{enumerate}
%    \item We propose a new automatic tool searching for iterative trails applying to permutations or block ciphers based on the SPN structure. Restricting the number of active S-boxes of difference or mask values, we model the 1-round differential or linear propagations as a weighted directed graph. Using an algorithm finding all elementary circuits in a directed graph \cite{J75}, we can enumerate all iterative trails and thus obtain the best one. The found iterative trails form a subgraph and can be visualized. We further take the effect of the combination of iterative trails into consideration and propose a method for finding iterative hulls which are clusters of iterative trails. 
%    \item Using the subgraph containing iterative trails and by extending trails from the vertices in the subgraph, we propose an algorithm to estimate the probabilities of differentials and correlations of linear hulls. 
%    \item For PRESENT, GIFT-64 and RECTANGLE, the results of EDP and ELP are close to the results in \cite{HV18} while our method costs much less time. The results implies that (1) the combination of iterative trails enhances differential and linear propagations and (2) the good differentials and linear hulls are dominated by iterative trails for these ciphers.
%    \item The inner permutations of KNOT, which is a round 2 candidate of the NIST lightweight cryptography standardization process, are inheritors of RECTANGLE. For 256-bit KNOT permutation, we can find good differentials up to 52 rounds and good linear hulls up to 51 rounds, the number of rounds increasing by 4 rounds and 12 rounds respectively compared to the result obtained by only considering single trails in the original specification document. 
%    \item For arbitrary-round the 256-bit KNOT permutation, PRESENT, GIFT-64 and RECTANGLE, we obtain the weight of the best trails constructed by iterative trails, which is an upper bound of the weight of the best trails. The upper bound is tight when the round number is large. Our method finishes in several seconds. Comparing our results for the 256-bit KNOT permutation with the results obtained using the method proposed by Bao et al. \cite{BZL14}, we find that the upper bound is exactly the best weight and the time our method consumes is negligible. See Table \ref{tab:sum}. 
%    \item For 256-bit KNOT permutation, GIFT-64 and RECTANGLE, we obtain the weight of the best differential or linear propagation composed of significant trails constructed by iterative trails. Because GIFT-64 and RECTANGLE are keyed block ciphers, the correlation potential ($\sum c^2$) is computed for a cluster of linear trails, while because KNOT permutations are permutations without key xor, the correlation contribution ($\sum c$) is computed for a cluster of linear trails. Except for linear cryptanalysis of GIFT-64, we obtain better weight of the best differential or linear propagation than the weight of the best single trail. See Table \ref{tab:sum}. 
%    \item For different versions of KNOT-AEAD, we obtain the upper bound of the weight of the best trails with input and output differences or masks constrained depending on different attack scenarios. The differential and linear attacks proposed are universally applicable for any AEAD scheme based on the Duplex mode of operations. 
%\end{enumerate}

\subsubsection{Organization}
The paper is organized as follows. In Section \ref{sec:pre}, we introduce notations and concepts. In Section \ref{sec:method_ite}, we present our method of finding differential or linear iterative distinguishers using a graph approach. Further we present our method of finding the best differential or linear hull containing iterative sub-characteristics. In Section \ref{sec:result}, results for 6 SPN symmetric-key primitives are shown. In Section \ref{sec:conclusion}, we conclude our paper.
%Section \ref{sec:method} gives the method modelling the problem of searching for iterative trails to a graph problem and the algorithm estimating the probability (correlation) of differentials (linear hulls). Section \ref{sec:improvements} proposes several improvement techniques. Section \ref{sec:experiment} shows experimental results. In Section \ref{sec:conclusion}, we conlude our work.

\section{Preliminaries}\label{sec:pre}

A block cipher is a function $\calE:\bbF_2^k \times \bbF_2^n \rightarrow \bbF_2^n$ with $C=\calE(K,P)$ where $K$, $P$ and $C$ are the $k$-bit master key, $n$-bit plaintext and $n$-bit ciphertext. $k$ is the key size and $n$ is the block size. Embedded into an mode of operation, a block cipher can be used to encrypt a message with arbitrary length. 

A permutation is a function $\calP:\bbF_2^n \rightarrow \bbF_2^n$ with $SO=\calP(SI)$ where $SI$ and $SO$ are the $n$-bit input and output state. The sponge/duplex construction where a permutation is embedded can build various primitives such as a hash function, a stream cipher, a MAC or an authenticated encryption scheme \cite{bertoni2007sponge}. Note that a block cipher with key fixed $\calE_K=\calE(K,\cdot)$ can be seen as a permutation.

In this paper, we focus on symmetric-key primitives including iterated key-alternating SPN block ciphers and primitives based on SPN permutations. The state of such a primitive can be seperated into $m$ words of $s$ bits and it holds that $n=s\times m$. The round function of the $i$-th round consists of three layers and is denoted by $\calR_i=\calL\circ\calS\circ\calA_{W_i}$ where the three layers are:
\begin{itemize}
    \item Addition layer $\calA_{W_i}$: xor the $i$-th $n$-bit round key or constant $W_i$ to the state;
    \item Non-linear layer $\calS$: apply $m$ parallel $s$-bit S-boxes to the words, i.e.
    \begin{align*}
        \calS=\calS_0||\cdots||\calS_{m-1};
    \end{align*}
    \item Linear layer $\calL$: multiply an $n\times n$ bijective binary matrix to the state. 
\end{itemize}
\begin{figure}[H]
    \centering
\begin{tikzpicture}
    \tikzstyle{every node}=[transform shape];
    \tikzstyle{every node}=[node distance=1.2cm];
    \tikzstyle{every node}=[font=\footnotesize,scale=0.9]
    \node (P) [] {$P(SI)$};
    \node (XOR-0) [right of=P,XOR] {};
    \node (S-0) [right of=XOR-0,draw,rectangle] {$\calS$};
    \node (L-0) [right of=S-0,draw,rectangle] {$\calL$};
    \node (XOR-1) [right of=L-0,XOR] {};
    \node (S-1) [right of=XOR-1,draw,rectangle] {$\calS$};
    \node (L-1) [right of=S-1,draw,rectangle] {$\calL$};
    \node (dots) [right of=L-1,] {$\dots$};
    \node (XOR-r-1) [right of=dots,XOR] {};
    \node (S-r-1) [right of=XOR-r-1,draw,rectangle] {$\calS$};
    \node (L-r-1) [right of=S-r-1,draw,rectangle] {$\calL$};
    \node (XOR-r) [right of=L-r-1,XOR] {};
    \node (C) [right of=XOR-r] {$C(SO)$};
    
    \path[line] (P) edge (XOR-0);
    \path[line] (XOR-0) edge node[above] {$X_0$} (S-0);
    \path[line] (S-0) edge node[above] {$Y_0$} (L-0);
    \path[line] (L-0) edge node[above] {$Z_0$} (XOR-1);
    \path[line] (XOR-1) edge node[above] {$X_1$} (S-1);
    \path[line] (S-1) edge node[above] {$Y_1$} (L-1);
    \path[line] (L-1) edge node[above] {$Z_1$} (dots);
    \path[line] (dots) edge node[above] {$Z_{r-2}$} (XOR-r-1);
    \path[line] (XOR-r-1) edge node[above] {$X_{r-1}$} (S-r-1);
    \path[line] (S-r-1) edge node[above] {$Y_{r-1}$} (L-r-1);
    \path[line] (L-r-1) edge node[above] {$Z_{r-1}$} (XOR-r);
    \path[line] (XOR-r) edge (C);

    \node (W-0) [above of=XOR-0] {$W_0$};
    \node (W-1) [above of=XOR-1] {$W_1$};
    \node (W-r-1) [above of=XOR-r-1] {$W_{r-1}$};
    \node (W-r) [above of=XOR-r] {$W_r$};

    \path[line] (W-0) edge (XOR-0);
    \path[line] (W-1) edge (XOR-1);
    \path[line] (W-r-1) edge (XOR-r-1);
    \path[line] (W-r) edge (XOR-r);

    \draw [decorate,decoration={brace,amplitude=5pt,mirror},xshift=-0.5cm,yshift=0pt] (1.2,-0.5) -- ++(2.3,0) node [black,midway,yshift=-0.5cm] {$\calR_0$};
    \draw [decorate,decoration={brace,amplitude=5pt,mirror},xshift=-0.5cm,yshift=0pt] (3.9,-0.5) -- ++(2.3,0) node [black,midway,yshift=-0.5cm] {$\calR_1$};
    \draw [decorate,decoration={brace,amplitude=5pt,mirror},xshift=-0.5cm,yshift=0pt] (7.5,-0.5) -- ++(2.3,0) node [black,midway,yshift=-0.5cm] {$\calR_{r-1}$};

    \draw [decorate,decoration={brace,amplitude=5pt},xshift=-0.5cm,yshift=0pt] (2.1,0.5) -- ++(1.3,0) node [black,midway,yshift=0.5cm] {$\calR_0^*$};
    \draw [decorate,decoration={brace,amplitude=5pt},xshift=-0.5cm,yshift=0pt] (4.8,0.5) -- ++(1.3,0) node [black,midway,yshift=0.5cm] {$\calR_1^*$};
    \draw [decorate,decoration={brace,amplitude=5pt},xshift=-0.5cm,yshift=0pt] (8.4,0.5) -- ++(1.3,0) node [black,midway,yshift=0.5cm] {$\calR_{r-1}^*$};

    %\draw [decorate,decoration={brace,amplitude=10pt,mirror},xshift=-0.5cm,yshift=0pt] (XOR-0.south west) -- node [black,midway,yshift=-0.7cm] {$\calR_0$} (L-0.south east);
    %\draw [decorate,decoration={brace,amplitude=10pt,mirror},xshift=-0.5cm,yshift=0pt] (XOR-1.south west) -- node [black,midway,yshift=-0.7cm] {$\calR_1$} (L-1.south east);
    %\draw [decorate,decoration={brace,amplitude=10pt,mirror},xshift=-0.5cm,yshift=0pt] (XOR-r-1.south west) -- node [black,midway,yshift=-0.7cm] {$\calR_{r-1}$} (L-r-1.south east);

    %\draw [decorate,decoration={brace,amplitude=10pt},xshift=-0.5cm,yshift=0pt] (S-0.north west) -- node [black,midway,yshift=0.7cm] {$\calR_0^*$} (L-0.north east);
    %\draw [decorate,decoration={brace,amplitude=10pt},xshift=-0.5cm,yshift=0pt] (S-1.north west) -- node [black,midway,yshift=0.7cm] {$\calR_1^*$} (L-1.north east);
    %\draw [decorate,decoration={brace,amplitude=10pt},xshift=-0.5cm,yshift=0pt] (S-r-1.north west) -- node [black,midway,yshift=0.7cm] {$\calR_{r-1}^*$} (L-r-1.north east);
\end{tikzpicture}
\caption{Structure of an SPN block cipher or permutation}
\label{fig:SPN}
\end{figure}
We use $W_i$ to denote the $i$-th round key for a block cipher or round constant for a permutation. The primitive iterates the round function $r$ times (See Fig. \ref{fig:SPN}). We denote the $i$-th round function excluding the addition layer by $\calR_i^*=\calL\circ\calS$. We denote the states before the non-linear layer, before the linear layer and after the linear layer of the $i$-th round function by $X_i$, $Y_i$ and $Z_i$. $X_i[j]$ denotes the $j$-th word of $X_i$, i.e. the input value of the $j$-th S-box. %For an $X_i$ with $k$ active S-boxes whose index set is $\mathfrak{K}=\{j_0,\cdots,j_{k-1}\}$, we denote it by $X_i[j_0]=x_0,\cdots,X_i[j_{k-1}]=x_{k-1}$ where $x_t\neq 0,\forall 0\leq t<k$ and $X_i[t]=0$ if $t\notin \mathfrak{K}$.

\subsection{Differential Cryptanalysis}

In differential cryptanalysis, the attacker tries to find an exploitable \textit{differential propagation}, which is a difference pair, with high probability to distinguish the target symmetric-key primitive from a random permutation. 
%That is, considering a couple of $n$-bit $a$ and $a'$ with $\alpha=a\oplus a'$, we let $\beta=h(a)\oplus h(a')$.

\begin{definition}[Differential Probability for $\calP$]
    For a permutation $\calP:\bbF_2^n\rightarrow \bbF_2^n$, the probability of a difference propagation is defined as:
    \begin{align*}
        \bbP(\alpha\xrightarrow{\calP}\beta)=2^{-n}\cdot\Big|\{x\in\bbF_2^n|\calP(x)\oplus\calP(x\oplus\alpha)=\beta\}\Big|.
    \end{align*}
\end{definition}

For a pair chosen uniformly from the set of all pairs $(a,a')$ where $a\oplus a'=\alpha$,
$\bbP(\alpha\xrightarrow{\calP}\beta)$ is the probability that $\calP(a)\oplus \calP(a')=\beta$, ranging from 0 to 1.

Since the fixed-key block cipher $\calE_K$ is a permutation, a differential propagation probability for $\calE_K$ is also defined as above. However, the key $K$ of $\calE_K$ is unknown for the attacker. Thus the \textit{Expected Differential Probability} (EDP) over all keys for $\calE$ is defined as follows:

\begin{definition}[EDP for $\calE$]
    For a block cipher $\calE:\bbF_2^n\rightarrow \bbF_2^k\times\bbF_2^n$, given $\alpha,\beta\in \bbF_2^n$, the EDP of the difference propagation $(\alpha,\beta)$ over a uniformly distributed random key $K\in \bbF_2^k$ is defined as:
    \begin{align*}
        \EDP(\alpha\xrightarrow{\calE}\beta)=2^{-k}\cdot\sum\limits_{K\in \bbF_2^k}\bbP(\alpha\xrightarrow{\calE_K}\beta)
    \end{align*}
\end{definition}

\begin{definition}[Differential Characteristic]
    An $r$-round differential characteristic is a sequence of $r+1$ differences $(\alpha_0,\cdots,\alpha_r)\in(\bbF_2^n)^{r+1}$. Its probability is
    \begin{align*}
        &\bbP(\alpha_0\xrightarrow{\calR_0}\cdots\xrightarrow{\calR_{r-1}}\alpha_r)\\
        =&2^{-n}\cdot\Big|\{x\in\bbF_2^n|\forall i, \calR_i\circ\cdots\circ\calR_0(x)\oplus\calR_i\circ\cdots\circ\calR_0(x\oplus\alpha_0)=\alpha_{i+1}\}\Big|\\
        \approx&\prod\limits_{i=0}^{r-1}\bbP(\alpha_i\xrightarrow{\calR_i^*}\alpha_{i+1})
        =\prod\limits_{i=0}^{r-1}\prod\limits_{j=0}^{m-1}\bbP(\alpha_i[j]\xrightarrow{\calS_i}\beta_i[j])
    \end{align*}
    where $\alpha_{i+1}=\calL(\beta_i),0\leq i<r$. 
\end{definition}

Under the assumption of independent random round keys and the hypothesis of stochastic equivalence \cite{lai1991markov}, the probability of a differential characteristic is calculated round by round and S-box by S-box.

\begin{definition}[Differential]
    An $r$-round differential $(\alpha,\beta)$ is a set of differential characteristics sharing the same input difference $\alpha$ and output difference $\beta$. 
\end{definition}

The probability of a difference propagation is estimated by summing the probabilities of all differential characteristics sharing the same input and output differences, i.e. the probability of the differential:
\begin{align*}
    \bbP(\alpha\xrightarrow{\calP}\beta) \text{ or } \EDP(\alpha\xrightarrow{\calE}\beta)=&\sum\limits_{\substack{\alpha_0=\alpha,\alpha_r=\beta\\\alpha_1,\cdots,\alpha_{r-1}}}\bbP(\alpha_0\xrightarrow{\calR_0}\cdots\xrightarrow{\calR_{r-1}}\alpha_r)\\
    \approx&\sum\limits_{\substack{\alpha_0=\alpha,\alpha_r=\beta\\\alpha_1,\cdots,\alpha_{r-1}}}\prod\limits_{i=0}^{r-1}\prod\limits_{j=0}^{m-1}\bbP(\alpha_i[j]\xrightarrow{\calS_i}\beta_i[j])
\end{align*}

\subsubsection{Truncated Differential \cite{daemen2002design}}
Let $\lambda$ be a linear function corresponding to an $n\times l$ binary matrix $M$. The probability of a \textit{truncated} differential of $\lambda\circ\calP$ is given by:
\[
    \bbP(\alpha\xrightarrow{\lambda\circ\calP}\beta)=\sum\limits_{\omega|\beta=M\omega}\bbP(\alpha\xrightarrow{\calP}\omega).
\]

\subsection{Linear Cryptanalysis}

In linear cryptanalysis, the attacker tries to find an exploitable \textit{linear propagation}, which is a mask pair, revealing an approximate linear relation between the input value and the output value of the target symmetric-key primitive, and thus we can distinguish the target primitive from a random permutation. 

\begin{definition}[Correlation]
    The correlation of a Boolean function $f:\bbF_2^n\rightarrow\bbF_2$ is
    \begin{align*}
        c_f=2^{-n}\cdot\Big(\Big| \{x\in\bbF_2^n|f(x)=0\} \Big|-\Big| \{x\in\bbF_2^n|f(x)=1\} \Big|\Big)
    \end{align*}
\end{definition}

\begin{definition}[Correlation for $\calP$]
    For a permutation $\calP:\bbF_2^n\rightarrow \bbF_2^n$, the correlation of a linear propagation is defined as:
    \begin{align*}
        \Cor(\alpha\xrightarrow{\calP}\beta)=c_{\alpha\cdot x\oplus \beta\cdot\calP(x)}
    \end{align*}
    %$\alpha\cdot x\oplus \beta\cdot\calP(x)$ is a linear approximation of $\calP$ and we denote $c_{\alpha\cdot x\oplus \beta\cdot\calP(x)}$ by $\Cor(\alpha\xrightarrow{\calP}\beta)$.
\end{definition}

The fixed-key block cipher $\calE_K$ is a permutation. However, the key $K$ of $\calE_K$ is unknown for the attacker. Thus the \textit{Expected Linear Potential} (ELP) over all keys for $\calE$ is defined as follows:

\begin{definition}[ELP for $\calE$]
    For a block cipher $\calE:\bbF_2^n\rightarrow \bbF_2^k\times\bbF_2^n$, given $\alpha,\beta\in \bbF_2^n$, the ELP of the linear propagation $(\alpha,\beta)$ over a uniformly distributed random key $K\in \bbF_2^k$ is defined as:
    \begin{align*}
        \ELP(\alpha\xrightarrow{\calE}\beta)=2^{-k}\cdot\sum\limits_{K\in \bbF_2^k}\Cor^2(\alpha\xrightarrow{\calE_K}\beta)
    \end{align*}
\end{definition}

\begin{definition}[Linear Characteristic]
    An $r$-round linear characteristic is a sequence of $r+1$ masks $(\alpha_0,\cdots,\alpha_r)\in(\bbF_2^n)^{r+1}$. Its correlation is calculated by
    \begin{align*}
        \Cor(\alpha_0\xrightarrow{\calR_0}\cdots\xrightarrow{\calR_{r-1}}\alpha_r)
        =(-1)^{\oplus_{i=0}^r \alpha_i\cdot W_i}\cdot\prod\limits_{i=0}^{r-1}\Cor(\alpha_i\xrightarrow{\calR_i^*}\alpha_{i+1})
    \end{align*}
\end{definition}

\begin{definition}[Linear Hull]
    An $r$-round linear hull $(\alpha,\beta)$ is a set of linear characteristics sharing the same input mask $\alpha$ and output mask $\beta$. 
\end{definition}

For a permutation, the correlation of a linear propagation is calculated by the signed sum of correlations all linear characteristics sharing the same input and output masks:
\begin{align*}
    \Cor(\alpha\xrightarrow{\calP}\beta)=&\sum\limits_{\alpha_1,\cdots,\alpha_{r-1}}\Cor(\alpha_0\xrightarrow{\calR_0}\cdots\xrightarrow{\calR_{r-1}}\alpha_r)\\
    =&\sum\limits_{\alpha_1,\cdots,\alpha_{r-1}}\prod\limits_{i=0}^{r-1} (-1)^{\oplus_{i=0}^r \alpha_i\cdot W_i} \prod\limits_{j=0}^{m-1}\Cor(\alpha_i[j]\xrightarrow{\calS_i}\beta_i[j]).
\end{align*}

For a key-alternating block cipher, the ELP of a linear propagation is calculated by summing the correlation squares of all linear characteristics sharing the same input and output masks according to Theorem 7.9.1 in \cite{daemen2002design}:
\begin{align*}
    \ELP(\alpha\xrightarrow{\calE}\beta)=&\sum\limits_{\alpha_1,\cdots,\alpha_{r-1}}\Cor^2(\alpha_0\xrightarrow{\calR_0}\cdots\xrightarrow{\calR_{r-1}}\alpha_r)\\
    =&\sum\limits_{\alpha_1,\cdots,\alpha_{r-1}}\Cor^2(\alpha_0\xrightarrow{\calR_0^*}\cdots\xrightarrow{\calR_{r-1}^*}\alpha_r)\\
    =&\sum\limits_{\alpha_1,\cdots,\alpha_{r-1}}\prod\limits_{i=0}^{r-1}\prod\limits_{j=0}^{m-1}\Cor^2(\alpha_i[j]\xrightarrow{\calS_i}\beta_i[j])
\end{align*}
where $\beta_i=\calL^T(\alpha_{i+1}),0\leq i<r$.



\subsection{Concepts in Graph Theory} \label{sec:graph-concept}

A \textit{directed graph} $G(V, E)$ consists of a nonempty and finite set of \textit{vertices} $V$ and a set $E$ of ordered pairs of distinct vertices called \textit{edges}. We denote a directed edge from a vertex $u\in V$ to a vertex $v\in V$ by $u\rightarrow v$. $u$ is called the head of the edge and $v$ is called the tail of the edge. Each edge $u\rightarrow v$ can be associated with a \textit{cost} and we denote it by $c(u\rightarrow v)$. A \textit{path} $p_{u,v}$ is a sequence of vertices $(u=v_0,v_1,\cdots,v_{k-1},v=v_k)$ such that $v_i\rightarrow v_{i+1}\in E, 0\leq i<k$. The \textit{length} of the path is
\[
    l(p_{u,v})=k,
\]
the \textit{cost} of the path is
\[
    c(p_{u,v})=\prod\limits_{i=1}^{k}c(v_{i-1}\rightarrow v_i).
\]
A \textit{hull} of $(u,v)$ is defined as the set of all paths $p_{u,v}$ leading from $u$ to $v$. More specifically, we define a $k$-length hull of $(u,v)$, denoted by $h^k_{u,v}$, as the set of all paths $p_{u,v}$ satisfying $l(p_{u,v})=k$. The cost of $h^k_{u,v}$ is
\[
    c(h^k_{u,v})=\sum\limits_{l(p_{u,v})=k} c(p_{u,v}).
\]

A path $p_{u,u}$ whose head is equal to its tail is called a \textit{circuit}. A circuit is \textit{elementary} if no vertex but the first and last appears twice. Two circuits are distinct if one is not a cyclic permutation of the other. An induced subgraph $G'=(V',E')$ is a (maximal) \textit{strong component} of $G$ if for all $u, v\in V'$ there exist paths $p_{uv}$ and $p_{vu}$ and this property holds for no subgraph of $G$ induced by a vertex set $\overline{V'}$ such that $V' \subset \overline{V'} \subseteq V$. A vertex in a strong component must belong to a circuit. And a vertex in a circuit must belong to a strong component. 

Let $G$ be a directed graph with vertices $V$ and edges $E$. If the vertices in $V$ are partitioned into $l$ subsets $S_0,\cdots,S_{l-1}$, called \textit{stages}, such that any edge in $E$ has the form $u\rightarrow v$ with $u\in S_i$ and $v\in S_{i+1}$, $0\leq i<l$. We call the graph a \textit{multistage graph}.

\subsubsection{Associating Characteristics with a Directed Graph}

For a directed graph $G$ with $V$ as its vertex set and $E$ as its edge set, we associate every vertex to a difference or mask. An edge $u\rightarrow v$ exists if the corresponding 1-round characteristic is valid. The cost of the edge can be associated with the differential probability, correlation or correlation square of the 1-round characteristic. A path with length $k$ is a $k$-round characteristic. A circuit is an iterative characteristic. A $k$-length hull is a $k$-round differential or linear hull. 

%\subsubsection{Notation} In this paper, we will not distinguish a vertex from a difference or mask value, an edge from a 1-round trail, a path from a trail, a weight from a differential probability or linear correlation. Note that the term "weight" has several meanings in other literatures, like the hamming weight of a difference or mask value, the negative logarithm of a differential probability or linear correlation. 


\section{Our Method of Finding Iterative Distinguishers}\label{sec:method_ite}

In this section, we present our method of searching for iterative distinguishers using a graph approach and further finding the best differential or linear hull containing iterative sub-characteristics. 

In Section \ref{sec:def-it}, we give the definition of iterative characteristics and further define \textit{rotational iterative characteristics} for primitives having the rotational symmetry. 

In Section \ref{sec:gen_G}, we present how we generate an interesting graph given the maximum number of active S-boxes to be considered of each round. 

In Section \ref{sec:find_ite_c}, we define its \textit{average weight growth} as the index quantizing the optimality of an iterative characteristic. Then we present our method of finding the best iterative characteristic in the generated graph. 

In Section \ref{sec:find_ite_h}, we propose our method of finding the best iterative differential or linear hull in the generated graph in order to investigate the clustering effect of iterative characteristics. 

In Section \ref{sec:method-edp-elp}, we propose our method of finding the best differential or linear hull containing iterative sub-characteristics. 

\subsection{Definitions of Iterative Characteristics}\label{sec:def-it}

For an SPN symmetric-key primitive, we first state the definition of an iterative characteristic. 
%that is used implicitly in \cite{wang2008differential,liu2019iterative} as follows:
\begin{definition}[Iterative Characteristic]\label{def:it}
	A differential (linear) characteristic with difference (mask) sequence $(\alpha_0,\cdots,\alpha_r)$ is iterative if $\alpha_0=\alpha_r$.
\end{definition}

%In this definition, the equals sign is taken as the equivalence in numerical value by default. 
However, such a definition cannot be applied to primitives like RECTANGLE \cite{zhang2015rectangle}, KNOT \cite{zhang2019knot} and ASCON \cite{dobraunig2016ascon} directly. For the substitution and linear operations in them have the rotational symmetry. In Example \ref{exp:ic_rect}, we give a 1-round differential characteristic of RECTANGLE. It's not an iterative characteristic according to Definition \ref{def:it}, but it can be concatenated to itself arbitrary times. 

\begin{example}\label{exp:ic_rect}
    a 1-round differential characteristic of RECTANGLE \cite{zhang2015rectangle} is
    \[
        \alpha_0:0x 0200 0060 0000 0000\xrightarrow{\bbP=2^{-5}}\alpha_1:0x 2000 0600 0000 0000.
    \]
    We can construct a differential characteristic as:
    \[
        0x 0002 0000 6000 0000 \rightarrow 0x 0020 0006 0000 0000 \rightarrow 0x 0200 0060 0000 0000 \cdots
    \]
\end{example}

Thus we generalize Definition \ref{def:it} by considering the equal sign as a equivalance relation. To adapt Example \ref{exp:ic_rect} to Definition \ref{def:it}, we define the rotationl equivalance relation as follows:

%is not adequate for some primitives like RECTANGLE \cite{zhang2015rectangle}, KNOT \cite{zhang2019knot}, ASCON \cite{dobraunig2016ascon}. Since all operations in them have rotational symmetry, every characteristic has rotational equivalent variants. 

%In Example \ref{exp:ic_rect}, the 1-round differential rotational iterative characteristic can be concatenated to itself arbitrary times, though $\alpha_0$ and $\alpha_1$ are not equal in numerical value. Thus we generalize Definition \ref{def:it} by considering the equal sign as a equivalance relation. We define the rotationl equivalance relation as follows:

\begin{definition}[Rotational Equivalence Relation]
	For $\alpha=\alpha[0]||\cdots||\alpha[m-1]$ and $\beta=\beta[0]||\cdots||\beta[m-1]$, $\alpha$ and $\beta$ are rotational equivalent, if there exists a $j\in[0,m-1]$ such that $\beta=\rotl_j(\alpha)$, where $\rotl_j(\alpha)=\alpha[j]||\alpha[j+1]||\cdots||\alpha[m-1]||\alpha[0]||\cdots||\alpha[j-1]$. 
\end{definition}

%\begin{definition}[Generalized Iterative %Characteristic]\label{def:git}
%	Given an equivalence relation $=_E$, a characteristic $(\alpha_0,\cdots,\alpha_r)$ is iterative if $\alpha_0=_E\alpha_r$.
%\end{definition}

%For two values which are equal under the equivalence relation, they belong to the same equivalence class. In one equivalence class, one of its element is chosen as its representative. In the case of RECTANGLE, we define the \textit{rotational equivalence relation} as:

We define a rotational iterative characteristic based on the rotational equivalance relation. In \cite{cui2019tangram}, the authors also defined the rotational iterative characteristic, but not using an equivalance relation. Note that for any other symmetry, the equivalence relation can also be customized. For example, it's mentioned in \cite{bertoni2013keccak} that all steps of the Keccak-f round function are translation invariant in the direction of the z axis, indicating a symmetry along the z axis. Thus our definition is general. However, in this papar, we only consider primitives having no symmetry and the rotatianal symmetry. 

\begin{definition}[Rotational Iterative Characteristic]\label{def:rit}
    A differential (linear) characteristic with difference (mask) sequence $(\alpha_0,\cdots,\alpha_r)$ is rotational iterative if $\alpha_0$ and $\alpha_r$ are rotational equivalent.
\end{definition}

In Example \ref{exp:ic_rect}, under the rotational equivalence relation, $\alpha_0$ and $\alpha_1$ are equivalent. Thus the 1-round characteristic is a rotational iterative characteristic. 

%Thus the one-round characteristic is iterative according to Definition \ref{def:git}. The representative of the class is $0x 6000 0000 0002 0000$. $d_R(\alpha_0)=6$ and $d_R(\alpha_1)=5$.

In the following parts in this section, we present our algorithms targetting primitives having no symmetry. The algorithms targeting primitives having the rotational symmetry are presented in Appendix \ref{sec:equiv}. 



%The rotational equivalence class of $\alpha$ is the set $[\alpha]_R=\{x|x=_R\alpha\}=\{\rotl_j(\alpha),0\leq j\leq m-1\}$. We define the representative of $[\alpha]_R$ as the largest value in $[\alpha]_R$ and we denote it by $r_R(\alpha)$. We further define the distance between $\alpha$ and its representative under $=_R$ as the $j\in[0,m-1]$ such that $r_R(\alpha)=\rotl_j(\alpha)$ and denote it by $d_R(\alpha)$. 



%For completeness, we denote the numerical equivalence relation by $=_{N}$. The numerical equivalence class of $\alpha$ is the set $[\alpha]_N$ containing only $\alpha$ itself. The representative of $[\alpha]_N$ is $\alpha$ itself. The distance between $\alpha$ and its representative under $=_N$ is always 0, i.e. $d_N(\alpha)\equiv 0$. In this paper, we only consider these two equivalence relations. 

\subsection{Generating an Interesting Graph}\label{sec:gen_G}

By viewing the problem of finding iterative characteristics as finding circuits in a graph, we can utilize currently available graph algorithms to solve it. The first step is to generate an interesting graph, since the graph will be exceedingly huge if we consider all $2^n$ differences or masks given the block size $n$. Thus we only consider differences or masks with no more than $A$ active S-boxes. We generate the interesting graph $G$ given $A$ as shown in Algorithm \ref{algo:gen-g}. We explain the algorithm as follows: 
\begin{itemize}
    \item In Line 2, we traverse every possible input difference (mask) $u$ of a 1-round characteristic. We keep $u$ with no more than $A$ active S-boxes. 
    \item In Line 3, we filter out the input difference (mask) $u$ not satisfying $\Asn(\calL^{-1}(u))\le A$. A 1-round characteristic $u\xrightarrow{\calL\circ\calS}v$ satisfying $\Asn(\calL^{-1}(u))>A$ won't occur in any circuit because in the traversal of output differences (masks), we keep the output differences (masks) with no more than $A$ active S-boxes. When generating $G$ to find linear distinguishers, we modify $\Asn(\calL^{-1}(u))\le A$ to $\Asn(\calL^{T}(u))\le A$.
    \item In Line 4, we traverse every possible difference (mask) $v$ satisfying $Asn(v)\le A$ and the 1-round characteristic $u\xrightarrow{\calL\circ\calS}v$ with input difference (mask) $u$ and output difference (mask) $v$ is valid. 
    \item In Line 5-6, we add edge $u\leftarrow v$ to graph $G$. The cost of the edge is the differential probability or correlation square of the 1-round characteristic. When generating $G$ to find linear distinguishers, we modify $\bbP(u\xrightarrow{\calL\circ\calS}v)$ to $\Cor^2(u\xrightarrow{\calL\circ\calS}v)$. 
    \item In Line 10, we reduce $G$ such that $G$ only contains circuits and paths linking two different vertices belonging to two different strong components. 
\end{itemize}
%for a primitive with block size $n$, the total number of vertices in $G$ is $2^n$, making the graph to be generated exceedingly huge. 

%In order to limit the size of the graph, we only consider difference or mask values with no more than $A$ active S-boxes. We define the function extracting the number of active S-boxes of $\alpha$ as $\Asn(\alpha)=\#\{j|\alpha[j]\neq 0\}$. 

%In Section \ref{sec:def-it}, we've introduced the customization of equivalence relation to generalize the definition of iterative characteristics. To adapt to the generalized circumstance, while generating $G$, we regard a vertex as the value of the representative of an equivalence class. As a result, between two vertices, there may be multiple edges corresponding to independent costs. Thus we add an extra dimension, the distance difference between the two original values, to index the edges. 

%We generate $G$ as Algorithm \ref{algo:gen_G} does in the case of diffrential cryptanalysis. While in the case of linear cryptanalysis, we substitute $\calL^{T}(u)$ for $\calL^{-1}(u)$ in Line 3 and $\Cor^2(u\xrightarrow{\calL\circ\calS}v)$ for $\bbP(u\xrightarrow{\calL\circ\calS}v)$ in Line 6. 

\begin{algorithm}
	\caption{Generate $G$}
	\label{algo:gen-g}
	\begin{algorithmic}[1]
		\Require upper limit of active S-box number $A$
        \Ensure graph $G$
		\Procedure {}{}
        \State Create an empty graph $G$
		\For{each $u\in\bbF_2^n$ satisfying $\Asn(u)\leq A$}
		\If{$\Asn(\calL^{-1}(u))\leq A$}
		\For{each $v\in\bbF_2^n$ satisfying $\Asn(v)\leq A$ and the 1-round characteristic $u\xrightarrow{\calL\circ\calS} v$ is valid}
		\State Add edge $u\rightarrow v$ to $G$
        \State $c(u\rightarrow v)\leftarrow \bbP(u\xrightarrow{\calL\circ\calS}v)$
		\EndFor
		\EndIf
		\EndFor
        \State Reduce $G$ by removing any vertex if it doesn't have at least one incoming and one outgoing edge until no more vertices can be removed. 
        \State \Return{$G$}
		\EndProcedure
	\end{algorithmic}
\end{algorithm}



\subsection{Finding Iterative Characteristics}\label{sec:find_ite_c}

%Note that one characteristic is iterative or not is irrelavant to its diffrential probability or correlation. Thus 
%We can reduce $G$ to its variant $G'$ where multiple edges between two vertices in $G$ degrade to one edge in $G'$. 
%Applying Johnson's algorithm \cite{johnson1975finding} to $G'$, we can enumerate all elementary circuits in $G'$, i.e. all elementary iterative characteristics described via $G'$. The next problem is how to define the optimality of iterative characteristics and whether there is a difference between the optimality of iterative characteristics and the optimality of elementary ones. 

Applying Johnson's algorithm \cite{johnson1975finding} to $G$, we can enumerate all elementary circuits in $G$. The next two problems are (1) how to define the optimality of iterative characteristics and (2) whether there is a difference between the optimality of iterative characteristics and the optimality of elementary iterative characteristics. 

For a characteristic with differential probability or correlation square $2^{-w}$, we call $w$ the weight of the characteristic. For an $r$-round iterative characteristic with weight $w$, we regard its weight per round $w/r$ as an index describing the \textit{average weight growth} for each round of the iterative characteristic. We refer the best iterative characteristic to the one with the smallest average weight growth. 
%According to this optimality of iterative characteristics, we assign the edge with the largest cost in $G$ to $G'$, i.e. $c'(u\rightarrow v)$ is assigned with $\max_d c(u\rightarrow_d v)$ where $c'(\cdot\rightarrow\cdot)$ denotes the cost of an edge in $G'$. 

To obtain the best average weight growth of a circuit in $G$, we obtain the best average weight growth of an elementary circuit in $G$ instead. We state that
\begin{theorem}
    One of the best iterative characteristics must be an elementary one.
\end{theorem}
\begin{proof}
    Suppose that none of the best iterative characteristics are elementary ones. Then for any of the best iterative characteristics $ic$, we divide it into two iterative characteristics $ic_1$ and $ic_2$ since it's not elementary. The weight per round of $ic_1$, $ic_2$ and $ic$ is respectively $wpr_1=-\log_2|c(ic_1)|/l(ic_1)$, $wpr_2=-\log_2|c(ic_2)|/l(ic_2)$ and $wpr=-\log_2|c(ic_1)c(ic_2)|/(l(ic_1)+l(ic_2))$. If $wpr_1< wpr_2$, then $wpr_1<wpr<wpr_2$, which contradicts to that $ic$ is a best iterative characteristic. Else if $wpr_1=wpr_2$, then $ic_1$ and $ic_2$ are both not elementary, since they are also best iterative characteristics. Thus the above steps can be continuously conducted on $ic_1$ and $ic_2$. However, the division can't be conducted infinite times. One of $ic_1$ and $ic_2$ will be elementary at some time and then we will get a contradiction.
\end{proof}

Therefore, it's enough to investigate only on elementary iterative characteristics and reasonable to apply Johnson's algorithm \cite{johnson1975finding}. Through enumerating elementary circuits, we can evaluate the average weight growth for each of them and thus find the best one. The procedure is given as shown in Algorithm \ref{algo:find-ite-c}. We explain the algorithm as follows: 

\begin{itemize}
    \item In Line 2, we set $bwpr$, which means the best weight per round, to infinity. 
    \item In Line 3, we enumerate all elementary circuits in $G$ applying Johnson's algorithm \cite{johnson1975finding}. In practice, we can limit the length of the circuits within 10 when the size of the generated graph is very large. Otherwise the algorithm will stick with enumerating elementary iterative characteristics with too many rounds. 
    \item In Line 4, $bwpr$ restores the smallest average weight growth of an elementary circuit. 
\end{itemize}

\begin{algorithm}
	\caption{Find the best iterative characteristic in $G$}
	\label{algo:find-ite-c}
	\begin{algorithmic}[1]
        \Require $G$
        \Ensure The best average weight growth of an iterative characteristic
        \Procedure {}{}
        \State $bwpr\leftarrow\infty$
        \For{each elementary circuit $p=(\alpha_0,\cdots,\alpha_r)$ in $G$}
        \State $bwpr\leftarrow\min\{(-\log_2\Big|\prod\limits_{i=0}^{r-1}c'(\alpha_i,\alpha_{i+1})\Big|)/r,bwpr\}$        
        \EndFor
        \State \Return{$bwpr$}
		\EndProcedure
	\end{algorithmic}
\end{algorithm}

\subsection{Finding the Best Iterative Differential or Linear Hull}\label{sec:find_ite_h}

In order to investigate the clustering effect, we try to find the best iterative differential or linear hull in $G$, that is, to compute $\min\limits_{\substack{r\\u\in G}}(-\log_2c(h^r_{u,u}))/r$. 

%A strong component $G'=(V',E')$ is a subgraph of $G$ where for all $u,v\in V'$ there exist paths $p_{u,v}$ and $p_{v,u}$. For all $u\in V'$, all circuits in $G$ containing $u$ are included within $G'$. Applying Tarjan's algorithm in \cite{tarjan1972depth} to $G$, we can obtain the vertex sets of all the distinct strong components of $G$. Then each strong component $G'$ can be induced from its vertex set $V'$. 

If two elementary circuits have any common vertex, then a better iterative distinguisher can be found. We use Example \ref{exp:ih} to illustrate. 

\begin{example}\label{exp:ih}
    Suppose that we find two elementary iterative characteristic $p_0=(\alpha_0,\alpha_1,\alpha_0)$ with weight per round $wpr_0$ and $p_1=(\alpha_0,\alpha_1,\alpha_2,\alpha_0)$ with weight per round $wpr_1$. See Figure \ref{fig:eg-hull}. Then a 6-length hull $h_{\alpha_0,\alpha_0}^6$ with weight per round $-\log_2(2^{-wpr_0}+2^{-wpr_1})$ can be constructed. Such an iterative hull is better than any of the two single iterative characteristics. 
\end{example}

\begin{figure}[H]
    \centering
\begin{tikzpicture}
    \tikzstyle{every node}=[transform shape];
    \tikzstyle{every node}=[node distance=1.2cm];
    \tikzstyle{every node}=[font=\footnotesize,scale=0.9]
    \node (a00) [] {$\alpha_0$};
    \node (a10) [right of=a00] {$\alpha_1$};
    \node (a20) [right of=a10] {$\alpha_2$};
    \node (a21) [below of=a20] {$\alpha_0$};
    \node (a30) [right of=a20] {$\alpha_0$};
    \node (a31) [below of=a30] {$\alpha_1$};
    \node (a40) [right of=a30] {$\alpha_1$};
    \node (a41) [below of=a40] {$\alpha_2$};
    \node (a50) [right of=a40] {$\alpha_0$};
    
    \path[line] (a00) edge node[above] {} (a10);
    \path[line] (a10) edge node[above] {} (a20);\path[line] (a10) edge node[above] {} (a21);
    \path[line] (a20) edge node[above] {} (a30);
    \path[line] (a21) edge node[above] {} (a31);
    \path[line] (a30) edge node[above] {} (a40);
    \path[line] (a31) edge node[above] {} (a41);\path[line] (a40) edge node[above] {} (a50);
    \path[line] (a41) edge node[above] {} (a50);
\end{tikzpicture}
\caption{An iterative distinguisher constructed by two iterative characteristics}
\label{fig:eg-hull}
\end{figure}

Given the interesting graph $G$, we find the best iterative differential or linear hull with rounds no more than $rd$ as shown in Algorithm \ref{algo:find-ite-h}. We explain the algorithm as follows: 


%Given $r$, to find 

%According to Example \ref{exp:ih}, we may find better iterative distinguishers by considering the clustering effect. To calculate the largest value of $c(h_{u,u}^r),u\in V'$ given the round number $r$, we perform a breadth first traversal of the graph for each vertex in $V'$. The procedure is given in Algorithm \ref{algo:find_ite_h_G}.

%With the number of rounds increasing, the clustering effect of iterative characteristics having common vertices will be stronger. Note that, iterative distinguishers are used to concatenate to itself to construct a long distinguisher. However, an iterative distinguisher is unexploitable if it's too long. 

\begin{itemize}
    \item In Line 2, $bwpr$, restoring the best average weight growth, is set to infinity. 
    \item In Line 3, we traverse $r$ from $1$ to $rd$. Given the fixed $r$, to find the minimun $(-\log_2c(h^r_{u,u}))/r$ is to find the maximum $c(h^r_{u,u})$.
    \item In Line 4, we traverse the strong components in $G$ applying Tarjan's algorithm in \cite{tarjan1972depth}. all the vertices of one circuit must belong to the same strong component. 
    \item In Line 5-18, we compute $c(h^r_{u,v})$ for all $u,v\in V_{sc}$ throung a round-by-round way implemented by hash tables, where $V_{sc}$ is the vertex set of a strong component. Then we find the maximum $c(h^r_{u,u})$ in the hash table. $bwpr$ restores the best average weight growth.
\end{itemize}

\begin{algorithm}
	\caption{Find the best iterative differential or linear hull in $G$}
	\label{algo:find-ite-h}
	\begin{algorithmic}[1]
        \Require $G$, upper limit of the round number $rd$
        \Ensure the best average weight growth of an iterative distinguisher considering the clustering effect with the number of rounds no more than $rd$
        \Procedure {}{}
        \State $bwpr\leftarrow\infty$
        \For{$r=1:rd$}
        \For{each distinct strong component of $G$ with vertex set $V_{sc}$}
        \State Let $\calH(\cdot,\cdot)$ be an empty hash table. 
        \State $\calH(u,v)\leftarrow c(u\rightarrow v),\forall u,v\in V_{sc}$
        \For{$i\leftarrow 2:r$}
        \State Create an empty hash table $\calH'(\cdot,\cdot)$.
        \For{each key $(u,z)$ of $\calH(\cdot,\cdot)$ and each edge $z\rightarrow v$}
        \If{$\calH'(u,v)$ exists}
        \State $\calH'(u,v)\leftarrow\calH'(u,v)+\calH(u,z)\cdot c(z\rightarrow v)$
        \Else
        \State $\calH'(u,v)\leftarrow\calH(u,z)\cdot c(z\rightarrow v)$
        \EndIf
        \EndFor
        \State $\calH\leftarrow \calH'$
        \EndFor
        \State $bwpr\leftarrow \min\{-\log_2\max\limits_{u}|\calH(u,u)|/r,bwpr\}$
        \EndFor
        \EndFor
        \State \Return{$bwpr$}
        \EndProcedure
	\end{algorithmic}
\end{algorithm}

\subsection{Finding the Best Differential or Linear Hull containing iterative sub-characteristics}\label{sec:method-edp-elp}

An iterative characteristic is exploited in two phases. Firstly, it's concatenated to itself several times, forming a longer characteristic. Secondly, the resulting characteristic is extended both forward and backward by several rounds. Following this idea, in our method, we pick a path in $G$ and it's extended backward from the first vertex and forward from the last vertex, forming an exploitable characteristic. To consider as many such characteristics as we can, for each vertex in $G$, we extend characteristics from it by several rounds and restore them. given the round number $r$, We create a multistage graph $MSG$ with $r+1$ stages $S_0,\cdots,S_r$. Then we add vertices and edges to $MSG$ according to $G$ and those extended characteristics. Once $MSG$ determined, we compute $-\log_2\max\limits_{u\in S_0,v\in S_r}c(h^r_{u,v})$. 

%According to the previous sections, once the interesting graph generated, we can obtain the strong components and enumerate the elementary circuits using existing graph algorithms. 

%When exploiting the iterative characteristics, a path linking two circuits in different strong components is also valuable to be taken into consideration. We call a subgraph $G_{IS}$ of $G$ its \textit{iterative structure} if $G_{IS}$ contains all strong components and path linking different strong components. To reduce $G$ to $G_{IS}$, we eliminate any vertex if it does not have at least one incoming and one outgoing edge until no more vertices can be eliminated. 

%Conventionally, an iterative characteristic is exploited in two phases. Firstly, it's concatenated to itself. Secondly, the resulting characteristic is extended both forward and backward by several rounds. Following this idea, a path is picked in $G_{IS}$ and extended to obtain an exploitable characteristic. To achieve this, for each vertex in $G_{IS}$, we extend a characteristic from it by several rounds and restore them.

%To estimate the EDPs and ELPs, we utilize the framework proposed in \cite{EPRINT:HalVej18} as we demonstrate in Section \ref{sec:frame-msg}. Once the interesting multistage graph is built, we can obtain the best hull within the graph. We present the total procedure in Algorithm \ref{algo:msg}.

Given the interesting graph $G$, the number of rounds $r$ and $(rb,wb)$ limiting the scope of characteristics to be considered, we find the best differential or linear hull containing iterative sub-characteristics as shown in Algorithm \ref{algo:msg}. We explain the algorithm as follows: 

%When exploiting the iterative characteristics, conventionally it's concatenated to itself. If there is a path linking two iterative characteristics, then

\begin{itemize}
    \item In Line 2, we create an emtpy multistage graph $MSG$. 
    \item In Line 3, we add edges to $MSG$ according to $G$. 
    \item In Line 4-9, we add edges to $MSG$ according to characteristics extended backward or forward from vertices in $G$. We keep that the number of rounds of an extended characteristic doesn't exceed $rb$ and the differential probability or correlation square of an extended characteristic is at least $2^{-wb}$. 
    \item In Line 10-26, once $MSG$ determined, we compute $\max\limits_{u\in S_0,v\in S_r} c(h^r_{u,v})$ round by round. $bw$ restores the candidate best weight of a differential or linear hull. In Line 25, after the round-by-round computation, $\calH(v)$ restores the probability of the differential $(u,v)$ or the correlation square of the linear hull $(u,v)$. The time complexity can be reduced if we consider $u\in S_0$ simultaneously. However this will increase the memory complexity from $\calO(|S_r|)$ to $\calO(|S_0|\cdot|S_r|)$ \cite{EPRINT:HalVej18}. In our method, $S_0$ and $S_r$ contains the most vertices. 
\end{itemize}

\begin{algorithm}
	\caption{Find the best differential or linear hull containing iterative sub-characteristics}
	\label{algo:msg}
	\begin{algorithmic}[1]
        \Require $G$, round number $r$, maximum round number to be extended $rb$, maximum weight a extended characteristic can has $wb$
        \Ensure The weight of the best differential or linear hull containing iterative sub-characteristics
        \Procedure {}{}
        
        \State Create an empty multistage graph $MSG$ with $r+1$ stages $S_0,\cdots,S_r$.
        \State Add each edge in $G$ between $S_i$ and $S_{i+1}$ for all $i\in[0,r-1]$
        \For{each characteristic $p=(u=u_0,u_1,\cdots,u_k)$ extended forward from $u\in G$ satisfying $k\leq rb$ and $c(p)\geq 2^{-wb}$} 
        \State Add edge $u_j\rightarrow u_{j+1}$ between $S_{r-k+j}$ and $S_{r-k+j+1}$
        \EndFor
        \For{each characteristic $(u_0,u_1,\cdots,u_k=u)$ extended backward from $u\in G$ satisfying $k\leq rb$ and $c(p)\geq 2^{-wb}$}
        \State Add edge $u_j\rightarrow u_{j+1}$ between $S_{k-1+j}$ and $S_{k+j}$
        \EndFor

        \State $bw\leftarrow\infty$
        \For{each $u\in S_0$}
        \State Let $\calH(\cdot)$ be an empty hash table. 
        \State $\calH(v)\leftarrow c(u\rightarrow v),\forall v\in S_1$
        \For{$i\leftarrow 2:r$}
        \State Create an empty hash table $\calH'(\cdot)$.
        \For{each key $z$ of $\calH(\cdot)$ and each edge $z\rightarrow v,z\in S_{i-1},v\in S_i$}
        \If{$\calH'(v)$ exists}
        \State $\calH'(v)\leftarrow\calH'(v)+\calH(z)\cdot c(z\rightarrow v)$
        \Else
        \State $\calH'(v)\leftarrow\calH(z)\cdot c(z\rightarrow v)$
        \EndIf
        \EndFor
        \State $\calH\leftarrow \calH'$
        \EndFor
        \State $bw\leftarrow\min\{-\log_2\max\limits_{v}\calH(v),bw\}$
        \EndFor
        \State \Return{$bw$}
        \EndProcedure
	\end{algorithmic}
\end{algorithm}

\section{Results}\label{sec:result}

Given a SPN primitive and $A$, the maximum S-boxes to be considered, we first generate graph $G$ according to Section \ref{algo:gen_G}. Applying our method in Section \ref{algo:find_ite_c_G}, we compute the weight growth of the best iterative characteristic. Applying our method in Section \ref{algo:find_ite_h_G}, we compute the weight growth of the best the best iterative hull. Comparing the two results, we show the strength of the clustering effect through a bar chart. 

\subsection{Results on Iterative Characteristics and Hulls}

We apply our algorithms in Section \ref{sec:find_ite_c} and \ref{sec:find_ite_h} to SPN ciphers including RECTANGLE, PRESENT, GIFT and so on. Note that a iterative distinguisher is used to build a longer distinguisher, thus we keep the iterative distinguishers we consider with small number of rounds. In our experiments, we set the biggest number of rounds to be 10. We show results on differential cryptanalysis in Figure \ref{fig:bar_ddt} and results linear cryptanalysis in Figure \ref{fig:bar_lat2}. The blue bars are obtained applying our method in Section \ref{sec:find_ite_c} searching the best elementary iterative characteristics with no more than 10 rounds. The orange bars are obtained by computing the minimum value applying our method in Section \ref{sec:find_ite_h} given $r=1:10$. Comparing the two figures, it is shown that linear characteristics are more likely to cluster than diffrential ones. 

\begin{figure}\label{fig:bar_ddt}
    \centering
    \caption{Results on iterative differential characteristics for some SPN block ciphers including RECTANGLE, PRESENT, GIFT64, PUFFIN and TRIFLE. The y-axis is the negative logarithm of differential probabilities. The blue bars are the weight growth of the best iterative characteristics. The orange bars are the weight growth of the best iterative hulls.}
	\includegraphics[width=1\textwidth]{fig/bar_ddt.png}
\end{figure}

\begin{figure}\label{fig:bar_lat2}
    \centering
    \caption{Results on iterative linear characteristics for some SPN block ciphers including RECTANGLE, PRESENT, GIFT64, PUFFIN, EPCBC and TRIFLE. The y-axis is the negative logarithm of linear correlation squares. The blue bars are the weight growth of the best iterative characteristics. The orange bars are the weight growth of the best iterative hulls.}
	\includegraphics[width=1\textwidth]{fig/bar_lat2.png}
\end{figure}

In the following, we list the experimental settings for each of the primitive. 

\subsubsection{RECTANGLE} All operations of RECTANGLE has the rotational symmetry and thus we set the equivalance relation to be $=_R$. We set $A=3$ in both cases of differential and linear cryptanalysis. 

\subsubsection{PRESENT} We set $A=3$ in the differential case and $A=1$ in the linear case. 

\subsubsection{GIFT} We set $A=3$ in both differential and linear cases. 

\subsubsection{PUFFIN} We set $A=1$ in both differential and linear cases.

\subsubsection{EPCBC} We set $A=1$ in the linear case. We can't find any iterative characteristic though $A$ is set up to 3. 

\subsubsection{TRIFLE-BC} We set $A=1$ in both diffrential and linear cases. 

%\begin{figure}\label{fig:plot_ite}
%    \centering
%    \caption{Results on some SPN block ciphers, which are RECTANGLE, PRESENT, GIFT64, PUFFIN, EPCBC and TRIFLE from top to bottom resp. For each subfigure, the x-axis is the number of rounds. For a subfigure in the left column, the y-axis is the weight of differential probability, while for a subfigure in the right column, the y-axis is the weight of correlation square. The red spot denotes the best single iterative characteristic with fewest rounds obtained by Algo. \ref{algo:find_ite_c_G}. The yellow (green, blue) line denotes the best iterative hull with $A=$1 (2, 3).} 
%	\includegraphics[width=1\textwidth]{fig/iterative.png}
%\end{figure}

\subsection{Results on Estimating EDPs and ELPs}

We apply our algorithm in Section \ref{sec:method-edp-elp} to some SPN primitives. The results are given in Table \ref{tab:EDP} and Table \ref{tab:ELP}. 

\begin{table}
	\caption{Results on estimating EDPs}\label{tab:EDP}
	\centering
	\begin{tabular}{|c|c|c|c|c|c|c|}
		\hline
		cipher & $r$ & $A$ & $(k,-\log_2wlb)$ & $-\log_2\Prob$ & $-\log_2\EDP$ & $T_g+T_s$ \\
		\hline
		PRESENT & 16 & 2 & (3,10) & 70 & 61.84 & 26s+165s\\
		\hline
		GIFT-64 & 13 & 2 & (3,13) & 62 & 60.42 & 24s+76s\\
		\hline 
		RECTANGLE & 14 & 2 & (6,25) & 61 & 60.65 & 4s+133.1h \\
		\hline
		KNOT-perm-256 & 52 & (2,*,1) & (3,12) & 274 & 251.831 & 0s+10s\\
		\hline
	\end{tabular}
\end{table}

\begin{table}
	\caption{Results on estimating ELPs}\label{tab:ELP}
	\centering
	\begin{tabular}{|c|c|c|c|c|c|c|}
		\hline
		cipher & $r$ & $A$ & $(k,-\log_2wlb)$ & $-2\log_2\Cor$ & $-\log_2\ELP$ & $T_g+T_s$ \\
		\hline
		PRESENT & 24 & 1 & (3,8) & 92 & 63.61 & 0s+14s\\
		\hline
		GIFT-64 & 13 & 2 & (3,10) & 64 & 64 & 1s+1s\\
		\hline 
		RECTANGLE & 14 & 3 & (3,14) & 68 & 62.31 & 17s+1.5h \\
		\hline
		KNOT-perm-256 & 56 & 3 & (2,6) & 161 & 125.262 & 27s+6.0h\\
		\hline
	\end{tabular}
\end{table}

\subsubsection{Comparison with the results in \cite{EPRINT:HalVej18}}



\subsection{Results on differential and linear attacks against KNOT-AEAD and KNOT-Hash}

\begin{table}
	\caption{Results for the primary version of KNOT}\label{tab:knot}
	\centering
	\begin{tabular}{|c|c|c|c|c|c|c|c|}
		\hline
		Attack Model & $r$ & $A$ & $(k,-\log_2wlb)$ & $-\log_2$(EDP or ELP)\\
		\hline
		AM1 & 14 & 2 & (5,20) & 62.2 \\
		AM2 & 13 & 3 & (3,8) & 30.7*2=61.4 \\
		AM3 & 12 & 3 & (3,10) & 30.2*2=60.4 \\
		AM4 & 12 & 2 & (5,20) & 62.4 \\
		AM5 & 13 & 2 & (5,20) & 61.4 \\
		AM6 & 12 & 2 & (5,20) & 62.7 \\
		AM7 & 13 & 2 & (5,20) & 61.4 \\
		\hline
	\end{tabular}
\end{table}

\subsubsection{Comparison with the method of enumerating characteristics}

\subsection{Visualization of the Iterative Structure}

\begin{figure}\label{fig:dis-rect}
    \centering
    \caption{Visualization of the differential iterative structure of RECTANGLE when $A=2$. The red part is the strong components. }
	\includegraphics[width=0.5\textwidth]{fig/test_circuits.png}
\end{figure}

\begin{table}
	\caption{The differential iterative structure of RECTANGLE when $A=2$}\label{tab:dis-rect}
	\centering
	\begin{tabular}{|c|c|c|c|c|c|}
		\hline
		head No. & head difference & tail No. & tail difference & $d$ & $\Prob$ \\
		\hline
		3 & 8c00000000000000 & 33 & 3000c00000000000 & 0 & $2^{-5}$ \\
        3 & 8c00000000000000 & 34 & 3000800000000000 & 0 & $2^{-5}$ \\
        8 & 800000000000a000 & 19 & 500000000000000a & 0 & $2^{-6}$ \\
        16 & 6000000000000008 & 35 & 3000400000000000 & 15 & $2^{-5}$ \\
        16 & 6000000000000008 & 37 & 3000000000000000 & 15 & $2^{-5}$ \\
        16 & 6000000000000008 & 41 & 2300000000000000 & 14 & $2^{-5}$ \\
        19 & 500000000000000a & 3 & 8c00000000000000 & 2 & $2^{-5}$ \\
        33 & 3000c00000000000 & 8 & 800000000000a000 & 7 & $2^{-5}$ \\
        34 & 3000800000000000 & 19 & 500000000000000a & 4 & $2^{-6}$ \\
        35 & 3000400000000000 & 19 & 500000000000000a & 4 & $2^{-5}$ \\
        37 & 3000000000000000 & 48 & 2000800000000000 & 15 & $2^{-3}$ \\
        41 & 2300000000000000 & 3 & 8c00000000000000 & 3 & $2^{-6}$ \\
        48 & 2000800000000000 & 19 & 500000000000000a & 4 & $2^{-6}$ \\
        52 & 2000060000000000 & 16 & 6000000000000008 & 4 & $2^{-5}$ \\
        52 & 2000060000000000 & 52 & 2000060000000000 & 15 & $2^{-5}$ \\
        52 & 2000060000000000 & 71 & 2000000000000008 & 4 & $2^{-5}$ \\
        71 & 2000000000000008 & 33 & 3000c00000000000 & 15 & $2^{-6}$ \\
        71 & 2000000000000008 & 34 & 3000800000000000 & 15 & $2^{-6}$ \\
		\hline
	\end{tabular}
\end{table}
%\section{Experiments\label{sec:experiment}}

\subsection{Experiments on Searching for Iterative Trails}

We apply our method in Section \ref{subsec:iterative-trails} to PRESENT, GIFT-64, RECTANGLE, 256-bit KNOT permutation and ASCON permutation. The results on iterative trails are shown in Table \ref{tab:iterative-trails}. The visualizations of $G^{IT}_{F,max\_asn}$ are shown in Appendix \ref{app:visualization}.

%The iterative structures are illustrated in Figures \ref{fig:graph_knot256_ddt}-\ref{fig:graph_gift_lat} in Appendix \ref{sec:is}. The differential and linear iterative structures of PRESENT are too big to be illustrated. The quantitive summary results are illustrated in Table \ref{tab:is}. For PRESENT, we constrain the elementary differential iterative trails within 10 rounds, for the genuine iterative structure is too complicated.

\begin{table}
	\caption{Results on iterative trails}\label{tab:iterative-trails}
	\centering
	\tiny
	\begin{tabular}{|c|c|c|c|c|c|c|c|c|c|}
		\hline
		cryptanalysis & cipher ($F$) & rs. & $max\_asn$ &$|V_{F,max\_asn}|$&$\leq n$& $|V^{IT}_{F,max\_asn}|$ & \#ecs. & best w/l. &time\\
		\hline
		\multirow{5}{*}{differential (Prob.)}
		&KNOT-perm-256 & yes & 2 & 8214 & - & 5 & 6 & 5.3 & 0.3s\\
		\cline{2-10}
		&PRESENT & no & 2 & 17256 & 10 & 225 & 463 & 4.5 & 2.1s\\
		\cline{2-10}
		&GIFT-64 & no & 2 & 19344 & - & 32 & 66 & 5 & 1.8s\\
		\cline{2-10}
		& RECTANGLE & yes & 2 & 1450 & - & 6 & 3 & 5 & 0.1s\\
		\cline{2-10}
		& ASCON-perm & yes & 3 & 3939 & - & 0 & 0 & - & 2.7h\\
		\hline
		\multirow{4}{*}{linear (Cor.)}
		& KNOT-perm-256 & yes & 2 & 8229 & - & 8 & 10 & 3 & 0.4s\\
		\cline{2-10}
		& PRESENT & no & 1 & 208 & - & 27 & 114223 & 2 & 3.6s\\
		\cline{2-10}
		& GIFT-64 & no & 2 & 21696 & - & 16 & 4 & 3 & 2.3s\\
		\cline{2-10}
		& RECTANGLE & yes & 2 & 1465 & - & 10 & 16 & 3 & 0.1s\\
		\cline{2-10}
		& ASCON-perm & yes & 3 & 336 & - & 0 & 0 & - & 4.0h\\
		\hline
		\multicolumn{9}{l}{rs.: whether the cipher has the property of rotational symmetry}\\
		\multicolumn{9}{l}{$\leq n$: the length of any elementary circuit is restricted to no more than $n$}\\
		\multicolumn{9}{l}{\#ecs.: number of elementary circuits}\\
		\multicolumn{9}{l}{best w/l.: the smallest weight/length that an elementary circuit can has}\\
	\end{tabular}
\end{table}

We consider the smallest weight per length an elementary iterative trail can has (best w/l. in Table \ref{tab:iterative-trails}) as an index describing the growth of iterative differential and linear propagations. We can see that PRESENT has both the weakest growth of iterative differential and linear propagations. The weakest differential iterative trail is exactly the one found by Wang et al.\cite{W08}. 

%256-bit KNOT permutation, PRESENT, GIFT-64 and RECTANGLE share a common point that each of them uses a bit permutation as its linear layer which is hardware-friendly and an S-box also providing diffusion as its non-linear layer. However, due to the weakness of the linear layer, the best differential and linear weights of them all increase proportionally with the round number, because of the existence of iterative trails. Meanwhile ASCON permutation uses a low-latency S-box and a strong linear layer. From Table \ref{tab:is} we can see that the differential iterative structure of ASCON permutation is empty even when $k=3$.

%A question is raised that how far a cipher with a comparatively weak diffusion layer can go. As long as the iterative structure is not empty, the best weight increases proportionally with the round number. Through the visualization of iterative structure, we hope to inform symmetric-key primitive designers of the detailed structure of iterative trails. 

%\subsection{Results of the Best Weights}
%
%Results for 256-bit KNOT permutation, PRESENT, GIFT-64 and RECTANGLE on the best weights obtained using our methods in section \ref{sec:para2} and \ref{sec:para3} are illustrated in Tables \ref{tab:knot256}-\ref{tab:rect}. They are compared with the weights of the best differential and linear trails. Using the method proposed in section \ref{sec:para2}, we need to determine the parameters $(r^f,r^b)$ first. Using the method proposed in section \ref{sec:para3}, we need to determine the parameters $(r^f,w^f,r^b,w^b,gap)$ first, where $gap=wub-bw[r]$ when searching $r$-round clusters. The larger each parameter is, the more time is consumed and the more accurate the result is. 
%
%The results obtained by our method coincide with the results of genuine best weight when the round number is large, except for results of linear weights for GIFT-64. The best 13-round linear trail of GIFT given in \cite{ZZDX19} has correlation $2^{-34}$ and has no iterative trail in it. Our method is unable to find such a trail. 
%
%%Our method finishes in minutes or even in seconds and is able to obtain an upper bound of the best weight for ciphers have iterative trails.

\subsection{Experiments on Finding Differential Trails and Linear Trails}

We apply our method in Section \ref{subsec:find-trails} and \ref{subsec:find-clusters} to PRESENT, GIFT-64, RECTANGLE and 256-bit KNOT permutation. The algorithm is run on an Intel Core i7-6700 CPU at 3.40GHz with 16GB RAM. The results for differential cryptanalysis are shown in Table \ref{tab:EDP}. The results for linear cryptanalysis are shown in Table \ref{tab:ELP}. 

\begin{table}
	\caption{Results for differential cryptanalysis}\label{tab:EDP}
	\centering
	\begin{tabular}{|c|c|c|c|c|c|c|}
		\hline
		cipher & rounds & EDP1 & Time & EDP2 & Time \\
		\hline
		PRESENT & \\
		\hline 
		RECTANGLE & 13 & 56 & 1.2s & 55.6601 & 12007.5s \\
		\hline
		GIFT-64 \\
		\hline
		KNOT-perm-256 \\
		\hline
	\end{tabular}
\end{table}

\begin{table}
	\caption{Results for linear cryptanalysis}\label{tab:ELP}
	\centering
	\begin{tabular}{|c|c|c|c|c|c|c|}
		\hline
		cipher & rounds & ELP1 & Time & ELP2 & Time \\
		\hline
		PRESENT & \\
		\hline 
		RECTANGLE & 13 & 62 & 0.2s & 59.6377 & 337.195s \\
		\hline
		GIFT-64 \\
		\hline
		KNOT-perm-256 \\
		\hline
	\end{tabular}
\end{table}

\subsection{Experiment on KNOT}

\subsubsection{Attack Model 1}


%\section{Conclusion\label{sec:conclusion}}

In this work, we propose a new automatic tool searching for iterative distinguishers for SPN symmetric-key primitives. We visualize iterative characteristics as a directed graph hoping to provide insignts on cipher designing. Based on the iterative characteristics, we efficiently find exploitable differentials and linear hulls. 
%It's shown that the best differentials and linear approximations are dominated by iterative characteristics. 

We have conducted an initial study on ASCON's inner permutation and GIFT-128. A question raised for designers is that, how to in purpose design a cipher with bit permutation as its linear layer can have no low-weight iterative characteristics. 

Our method is not only suitable for SPN primitives, but also suitable for primitives of Feistel structure with SPN round functions. We leave it in future work. 

%
% ---- Bibliography ----
%
% BibTeX users should specify bibliography style 'splncs04'.
% References will then be sorted and formatted in the correct style.
%
\bibliographystyle{splncs04}
% \bibliography{mybibliography}
%
\begin{thebibliography}{8}
%\bibitem{ref_article1}
%Author, F.: Article title. Journal \textbf{2}(5), 99--110 (2016)

%\bibitem{ref_lncs1}
%Author, F., Author, S.: Title of a proceedings paper. In: Editor, F., Editor, S. (eds.) CONFERENCE 2016, LNCS, vol. 9999, pp. 1--13. Springer, Heidelberg (2016). \doi{10.10007/1234567890}

%\bibitem{ref_book1}
%Author, F., Author, S., Author, T.: Book title. 2nd edn. Publisher, Location (1999)

%\bibitem{ref_proc1}
%Author, A.-B.: Contribution title. In: 9th International Proceedings on Proceedings, pp. 1--2. Publisher, Location (2010)

%\bibitem{ref_url1}
%LNCS Homepage, \url{http://www.springer.com/lncs}. Last accessed 4 Oct 2017

%%%%%%%%%%%%%%%%%%%%%%%%%%%%%%%%%%%%%%%%%%%%%%%%%%%%%%%%%%%
%dedicated search:
%AKM97,BZL14,OMA95

%MILP search:
%MWG12,SHW14-1,SHW14-2,ZZDX19

%differential cryptanalysis:
%BS91,BS92

%linear cryptanalysis:
%M93,M94_1,M94_2

%ciphers
%Keccack:BDPV12,BDPVV14
%PRESENT:BKL07
%GIFT:BPP17
%Ascon:DEMS16
%AES:DR98,DR01,DR02
%RECTANGLE:ZBL15
%KNOT:ZDY19

%iterative trails
%K92,W08

%algorithms:
%J75,V03

%provable security
%R04
%%%%%%%%%%%%%%%%%%%%%%%%%%%%%%%%%%%%%%%%%%%%%%%%%%%%%%%%%%%

\bibitem{AKM97}
Aoki, K., Kobayashi, K.,  Moriai, S. (1997, January). Best differential characteristic search of FEAL. In International Workshop on Fast Software Encryption (pp. 41-53). Springer, Berlin, Heidelberg.

\bibitem{BDPV12}
Bertoni, G., Daemen, J., Peeters, M.,  Van Assche, G. (2012). Permutation-based encryption, authentication and authenticated encryption. Directions in Authenticated Ciphers, 159-170.

\bibitem{BDPVV14}
Bertoni, G., Daemen, J., Peeters, M., Van Assche, G.,  Van Keer, R. (2014). CAESAR submission: Ketje v2. CAESAR First Round Submission, March.

\bibitem{BKL07}
Bogdanov, A., Knudsen, L. R., Leander, G., Paar, C., Poschmann, A., Robshaw, M. J., ...  Vikkelsoe, C. (2007, September). PRESENT: An ultra-lightweight block cipher. In International workshop on cryptographic hardware and embedded systems (pp. 450-466). Springer, Berlin, Heidelberg.

\bibitem{BPP17}
Banik S, Pandey SK, Peyrin T, Sasaki Y, Sim SM, Todo Y. GIFT: a small present. InInternational Conference on Cryptographic Hardware and Embedded Systems 2017 Sep 25 (pp. 321-345). Springer, Cham.

\bibitem{BS91}
Biham E, Shamir A. Differential cryptanalysis of DES-like cryptosystems. Journal of CRYPTOLOGY. 1991 Jan 1;4(1):3-72.

\bibitem{BS92}
Biham E, Shamir A. Differential cryptanalysis of the full 16-round DES. InAnnual International Cryptology Conference 1992 Aug 16 (pp. 487-496). Springer, Berlin, Heidelberg.

\bibitem{BZL14}
Bao Z, Zhang W, Lin D. Speeding up the search algorithm for the best differential and best linear trails. In International Conference on Information Security and Cryptology 2014 Dec 13 (pp. 259-285). Springer, Cham.

\bibitem{DEMS16}
Dobraunig C, Eichlseder M, Mendel F, Schläffer M. Ascon v1.2. Submission to the CAESAR Competition. 2016 Sep 15.

\bibitem{DR98}
Daemen J, Rijmen V. The block cipher Rijndael. InInternational Conference on Smart Card Research and Advanced Applications 1998 Sep 14 (pp. 277-284). Springer, Berlin, Heidelberg.

\bibitem{DR01}
Daemen J, Rijmen V. The wide trail design strategy[C]//IMA International Conference on Cryptography and Coding. Springer, Berlin, Heidelberg, 2001: 222-238.

\bibitem{DR02}
Daemen J, Rijmen V. The design of Rijndael[M]. New York: Springer-verlag, 2002.

\bibitem{J75}
Johnson DB. Finding all the elementary circuits of a directed graph. SIAM Journal on Computing. 1975 Mar;4(1):77-84.

\bibitem{K92}
Knudsen LR. Iterative Characteristics of DES and s 2-DES. InAnnual International Cryptology Conference 1992 Aug 16 (pp. 497-511). Springer, Berlin, Heidelberg.
\bibitem{M93}
Matsui M. Linear cryptanalysis method for DES cipher. InWorkshop on the Theory and Application of of Cryptographic Techniques 1993 May 23 (pp. 386-397). Springer, Berlin, Heidelberg.

\bibitem{M94_1}
Matsui M. The first experimental cryptanalysis of the Data Encryption Standard. InAnnual International Cryptology Conference 1994 Aug 21 (pp. 1-11). Springer, Berlin, Heidelberg.

\bibitem{M94_2}
Matsui M. On correlation between the order of S-boxes and the strength of DES. InWorkshop on the Theory and Application of of Cryptographic Techniques 1994 May 9 (pp. 366-375). Springer, Berlin, Heidelberg.

\bibitem{MWG12}
Mouha N, Wang Q, Gu D, Preneel B. Differential and linear cryptanalysis using mixed-integer linear programming. InInternational Conference on Information Security and Cryptology 2011 Nov 30 (pp. 57-76). Springer, Berlin, Heidelberg.

\bibitem{OMA95}
Ohta K, Moriai S, Aoki K. Improving the search algorithm for the best linear expression. InAnnual International Cryptology Conference 1995 Aug 27 (pp. 157-170). Springer, Berlin, Heidelberg.

\bibitem{R04}
Rogaway P. Nonce-based symmetric encryption. InInternational Workshop on Fast Software Encryption 2004 Feb 5 (pp. 348-358). Springer, Berlin, Heidelberg.

\bibitem{SHW14-1}
Sun S, Hu L, Wang P, Qiao K, Ma X, Song L. Automatic security evaluation and (related-key) differential characteristic search: application to SIMON, PRESENT, LBlock, DES (L) and other bit-oriented block ciphers. InInternational Conference on the Theory and Application of Cryptology and Information Security 2014 Dec 7 (pp. 158-178). Springer, Berlin, Heidelberg.

\bibitem{SHW14-2}
Sun S, Hu L, Wang M, Wang P, Qiao K, Ma X, Shi D, Song L, Fu K. Towards finding the best characteristics of some bit-oriented block ciphers and automatic enumeration of (related-key) differential and linear characteristics with predefined properties. IACRCryptology ePrint Archive. 2014;747:2014.

\bibitem{V03}
Vajnovszki V. A loopless algorithm for generating the permutations of a multiset. Theoretical Computer Science. 2003 Oct 7;307(2):415-31.

\bibitem{W08}
Wang M. Differential cryptanalysis of reduced-round PRESENT. InInternational Conference on Cryptology in Africa 2008 Jun 11 (pp. 40-49). Springer, Berlin, Heidelberg.

\bibitem{ZBL15}
Zhang W, Bao Z, Lin D, Rijmen V, Yang B, Verbauwhede I. RECTANGLE: a bit-slice lightweight block cipher suitable for multiple platforms. Science China Information Sciences. 2015 Dec 1;58(12):1-5.

\bibitem{ZDY19}
Wentao Zhang, Tianyou Ding, Bohan Yang, Zhenzhen Bao, Zejun Xiang, Fulei Ji, Xuefeng Zhao. KNOT: Algorithm Specifications and Supporting Document. \url{https://csrc.nist.gov/CSRC/media/Projects/lightweight-cryptography/documents/round-2/spec-doc-rnd2/knot-spec-round.pdf}

\bibitem{ZZDX19}
Zhou C, Zhang W, Ding T, Xiang Z. Improving the MILP-based Security Evaluation Algorithms against Differential Cryptanalysis Using Divide-and-Conquer Approach. IACR Cryptology ePrint Archive. 2019;2019:19.

%TODO: SAT search references

%TODO: generating graph references



\end{thebibliography}
%\appendix

\section{Algoriths Targetting Primitives with the Rotational Symmetry}\label{sec:equiv}

In Section \ref{sec:def-it}, we've already defined rotational iterative characteristics using the rotational equivalence relation denoted by $=_E$. If two values are rotational equivalent, they belong to the same \textit{rotational equivalence class}. We define the \textit{representative} of a equivalence class as the largest value in it. We denote the function turning a value to its representative by $r_E(\cdot)$. We define The left rotation number from a value to its representative as the distance between them and denote it by $d_E(\cdot)$. For a value $u\in \bbF_2^{s\times m}$, $r_E(u)=\rotl_{d_E(u)}(u)$. 

The algorithms targetting primitives having the rotational symmetry are shown in Algorithm \ref{algo:gen-g-equiv}, \ref{algo:find-ite-c-equiv}, \ref{algo:find-ite-h-equiv} and \ref{algo:msg-equiv}. 

\begin{algorithm}[htbp]
	\caption{Generate $G$ for a rotational symmetric primitive}
	\label{algo:gen-g-equiv}
	\begin{algorithmic}[1]
		\Require upper limit of active S-box number $A$, the rotational equivalence relation $=_E$ and its corresponding functions $r_E$ and $d_E$
		\Procedure {}{}
                \State Create an empty graph $G$
		\For{each $u\in\bbF_2^n$ satisfying $\Asn(u)\leq A$}
		\If{$\Asn(\calL^{-1}(u))\leq A$}
		\For{each $v\in\bbF_2^n$ satisfying $\Asn(v)\leq A$ and the 1-round characteristic $u\xrightarrow{\calL\circ\calS}v$ is valid}
		\State $u'\leftarrow r_E(u),v'\leftarrow r_E(v)$
                \State $d\leftarrow (d_E(v)-d_E(u))\mod m$
		\State Add edge $u'\rightarrow v'$ to $G$
                \State $c(u'\xrightarrow[d]{} v')\leftarrow \bbP(u\xrightarrow{\calL\circ\calS}v)$
		\EndFor
		\EndIf
		\EndFor
                \State Reduce $G$ by removing any vertex if it doesn't have at least one incoming and one outgoing edge until no more vertices can be removed. 
                \Return{$G$}
		\EndProcedure
	\end{algorithmic}
\end{algorithm}

\begin{algorithm}[htbp]
	\caption{Find the best rotational iterative characteristic in $G$}
	\label{algo:find-ite-c-equiv}
	\begin{algorithmic}[1]
        \Require $G$
        \Ensure The best average weight growth of an rotational iterative characteristic
        \Procedure {}{}
        \State Create an empty graph $G'$
        \For{each two vertices $u$ and $v$ linked in $G$}
        \State Add edge $u\rightarrow v$ to $G'$
        \State $c'(u\rightarrow v)\leftarrow \max\limits_d c(u\xrightarrow[d]{} v)$
        \EndFor
        \State $bwpr\leftarrow\infty$
        \For{each elementary circuit $p=(\alpha_0,\cdots,\alpha_r)$ in $G'$}
        \State $bwpr\leftarrow\min\{-\log_2\Big|\prod\limits_{i=0}^{r-1}c'(\alpha_i,\alpha_{i+1})\Big|/r,bwpr\}$        
        \EndFor
        \Return{$bwpr$}
	\EndProcedure
	\end{algorithmic}
\end{algorithm}

\begin{algorithm}[htbp]
	\caption{Find the best rotatianal iterative differential or linear hull in $G$}
	\label{algo:find-ite-h-equiv}
	\begin{algorithmic}[1]
        \Require $G$, upper limits of the round number $rd$
        \Ensure the best average weight growth of an rotational iterative distinguisher considering the clustering effect with the number of rounds no more than $rd$
        \Procedure {}{}
        \State $bwpr\leftarrow\infty$
        \For{$r=1:rd$}
        \For{each distinct strong component of $G$ with vertex set $V_{sc}$}
        \State Let $\calH(\cdot,\cdot,\cdot)$ be an empty hash table. 
        \State $\calH(u,v,d)\leftarrow c(u\xrightarrow[d]{} v),\forall u,v\in V_{sc},\forall$ existent $d$
        \For{$i\leftarrow 2:r$}
        \State Create an empty hash table $\calH'(\cdot,\cdot,\cdot)$.
        \For{each key $(u,z,d_1)$ of $\calH(\cdot,\cdot,\cdot)$ and each edge $z\xrightarrow[d_2]{} v$}
        \State $d\leftarrow (d_1+d_2)\mod m$
        \If{$\calH'(u,v,d)$ exists}
        \State $\calH'(u,v,d)\leftarrow\calH'(u,v,d)+\calH(u,z,d_1)\cdot c(z\xrightarrow[d_2]{} v)$
        \Else
        \State $\calH'(u,v,d)\leftarrow\calH(u,z,d_1)\cdot c(z\xrightarrow[d_2]{} v)$
        \EndIf
        \EndFor
        \State $\calH\leftarrow \calH'$
        \EndFor
        \State $bwpr\leftarrow \min\{-\log_2\max\limits_{u,d}|\calH(u,u,d)|/r,bwpr\}$
        \EndFor
        \EndFor
        \Return{$bwpr$}
        \EndProcedure
	\end{algorithmic}
\end{algorithm}

\begin{algorithm}[htbp]
	\caption{Find the best differential or linear hull containing rotational iterative sub-characteristics}
	\label{algo:msg-equiv}
	\begin{algorithmic}[1]
        \Require $G$, round number $r$, maximum round number to be extended $rb$, maximum weight a extended characteristic can has $wb$
        \Ensure The weight of the best differential or linear hull containing rotational iterative sub-characteristics
        \Procedure {}{}
        \State $bw\leftarrow\infty$
        \State Create an empty multistage graph $MSG$ with $r+1$ stages $S_0,\cdots,S_r$.
        \State Reduce $G$ to its iterative structure $G_{IS}$. 
        \State Add each edge in $G$ between $S_i$ and $S_{i+1}$ for all $i\in[0,r-1]$
        \For{each characteristic $p=(u=u_0,u_1,\cdots,u_k)$ extended forward from $u\in G$ satisfying $k\leq rb$ and $c(p)\geq 2^{-wb}$}
        \State $u_j'\leftarrow r_E(u_j),\forall 0\leq j\leq k$
        \State $d_j\leftarrow d_E(u_{j+1})-d_E(u_j),\forall 0\leq j<k$.
        \State Add edge $u_j'\xrightarrow[d_j]{} u_{j+1}'$ between $S_{r-k+j}$ and $S_{r-k+j+1}$
        \EndFor
        \For{each characteristic $p=(u_0,u_1,\cdots,u_k=u)$ extended backward from $u\in G$ satisfying $k\leq rb$ and $c(p)\geq 2^{-wb}$}
        \State $u_j'\leftarrow r_E(u_j),\forall 0\leq j\leq k$
        \State $d_j\leftarrow (d_E(u_{j+1})-d_E(u_j))\mod m,\forall 0\leq j<k$.
        \State Add edge $u_j'\xrightarrow[d_j]{} u_{j+1}'$ between $S_{k-1+j}$ and $S_{k+j}$
        \EndFor

        \State $bw\leftarrow\infty$
        \For{each $u\in S_0$}
        \State Let $\calH(\cdot,\cdot)$ be an empty hash table. 
        \State $\calH(v,d)\leftarrow c(u\xrightarrow[d]{} v),\forall v\in S_1,\forall \text{ existent } d$
        \For{$i\leftarrow 2:r$}
        \State Create an empty hash table $\calH'(\cdot,\cdot)$.
        \For{each key $(z,d)$ of $\calH(\cdot,\cdot)$ and each edge $z\xrightarrow[d']{} v,z\in S_{i-1},v\in S_i$}
        \State $d''\leftarrow d+d'\mod m$
        \If{$\calH'(v,d'')$ exists}
        \State $\calH'(v,d'')\leftarrow\calH'(v,d'')+\calH(z,d)\cdot c(z\xrightarrow[d']{} v)$
        \Else
        \State $\calH'(v,d'')\leftarrow\calH(z,d)\cdot c(z\xrightarrow[d']{} v)$
        \EndIf
        \EndFor
        \State $\calH\leftarrow \calH'$
        \EndFor
        \State $bw\leftarrow\min\{-\log_2\max\limits_{v,d}\calH(v,d),bw\}$
        \EndFor

        %\State $bw\leftarrow-\log_2\max\limits_{\alpha\in S_0,\beta\in S_r}c(h^r_{\alpha,\beta})$
        \State \Return{$bw$}
        \EndProcedure
	\end{algorithmic}
\end{algorithm}

\section{Details of the Generated Graphs}

The detailed edges of the graphs given in Figure \ref{fig:graph-rect-ddt}, \ref{fig:graph-rect-lat2}, \ref{fig:graph-knot256-ddt}, \ref{fig:graph-knot256-lat2} and \ref{fig:graph-gift64-lat2} are respectively shown in Table \ref{tab:dis-rect-ddt}, \ref{tab:dis-rect-lat2}, \ref{tab:dis-knot256-ddt}, \ref{tab:dis-knot256-lat2} and \ref{tab:dis-gift64-lat2}. 

\begin{table}
	\caption{The edges of the generated graph for RECTANGLE containing differential rotational iterative characteristics when $A=2$ (corresponding to Figure \ref{fig:graph-rect-ddt})}\label{tab:dis-rect-ddt}
	\centering
	\begin{tabular}{|c|c|c|c|c|c|}
		\hline
		head No. & head difference & tail No. & tail difference & $d$ & $\Prob$ \\
		\hline
		3 & 8c00000000000000 & 33 & 3000c00000000000 & 0 & $2^{-5}$ \\
        3 & 8c00000000000000 & 34 & 3000800000000000 & 0 & $2^{-5}$ \\
        8 & 800000000000a000 & 19 & 500000000000000a & 0 & $2^{-6}$ \\
        16 & 6000000000000008 & 35 & 3000400000000000 & 15 & $2^{-5}$ \\
        16 & 6000000000000008 & 37 & 3000000000000000 & 15 & $2^{-5}$ \\
        16 & 6000000000000008 & 41 & 2300000000000000 & 14 & $2^{-5}$ \\
        19 & 500000000000000a & 3 & 8c00000000000000 & 2 & $2^{-5}$ \\
        33 & 3000c00000000000 & 8 & 800000000000a000 & 7 & $2^{-5}$ \\
        34 & 3000800000000000 & 19 & 500000000000000a & 4 & $2^{-6}$ \\
        35 & 3000400000000000 & 19 & 500000000000000a & 4 & $2^{-5}$ \\
        37 & 3000000000000000 & 48 & 2000800000000000 & 15 & $2^{-3}$ \\
        41 & 2300000000000000 & 3 & 8c00000000000000 & 3 & $2^{-6}$ \\
        48 & 2000800000000000 & 19 & 500000000000000a & 4 & $2^{-6}$ \\
        52 & 2000060000000000 & 16 & 6000000000000008 & 4 & $2^{-5}$ \\
        52 & 2000060000000000 & 52 & 2000060000000000 & 15 & $2^{-5}$ \\
        52 & 2000060000000000 & 71 & 2000000000000008 & 4 & $2^{-5}$ \\
        71 & 2000000000000008 & 33 & 3000c00000000000 & 15 & $2^{-6}$ \\
        71 & 2000000000000008 & 34 & 3000800000000000 & 15 & $2^{-6}$ \\
		\hline
	\end{tabular}
\end{table}

\begin{table}
	\caption{The edges of the generated graph for RECTANGLE containing linear rotational iterative characteristics when $A=2$ (corresponding to Figure \ref{fig:graph-rect-lat2})}\label{tab:dis-rect-lat2}
	\centering
	\begin{tabular}{|c|c|c|c|c|c|}
		\hline
		head No. & head mask & tail No. & tail mask & $d$ & $\Cor^2$ \\
		\hline
        0 & c000000000000000 & 114 & 1000000000000000 & 0 & $2^{-2}$ \\
        1 & a000000000000000 & 14 & 8000000000000000 & 3 & $2^{-2}$ \\
        2 & 9000000000000000 & 14 & 8000000000000000 & 3 & $2^{-4}$ \\
        2 & 9000000000000000 & 82 & 1008000000000000 & 0 & $2^{-4}$ \\
        8 & 800000000000a000 & 18 & 5008000000000000 & 0 & $2^{-6}$ \\
        8 & 800000000000a000 & 19 & 500000000000000a & 0 & $2^{-6}$ \\
        8 & 800000000000a000 & 107 & 100000000000000a & 0 & $2^{-4}$ \\
        9 & 8000000000008000 & 18 & 5008000000000000 & 0 & $2^{-6}$ \\
        9 & 8000000000008000 & 19 & 500000000000000a & 0 & $2^{-6}$ \\
        10 & 8000000000000900 & 10 & 8000000000000900 & 3 & $2^{-6}$ \\
        13 & 8000000000000008 & 79 & 100c000000000000 & 0 & $2^{-6}$ \\
        14 & 8000000000000000 & 82 & 1008000000000000 & 0 & $2^{-2}$ \\
        18 & 5008000000000000 & 10 & 8000000000000900 & 6 & $2^{-6}$ \\
        19 & 500000000000000a & 0 & c000000000000000 & 3 & $2^{-8}$ \\
        19 & 500000000000000a & 13 & 8000000000000008 & 3 & $2^{-6}$ \\
        19 & 500000000000000a & 54 & 3000c00000000000 & 15 & $2^{-8}$ \\
        19 & 500000000000000a & 79 & 100c000000000000 & 0 & $2^{-8}$ \\
        19 & 500000000000000a & 84 & 1000c00000000000 & 15 & $2^{-8}$ \\
        20 & 5000000000000008 & 0 & c000000000000000 & 3 & $2^{-8}$ \\
        20 & 5000000000000008 & 53 & 3008000000000000 & 15 & $2^{-4}$ \\
        20 & 5000000000000008 & 54 & 3000c00000000000 & 15 & $2^{-8}$ \\
        20 & 5000000000000008 & 79 & 100c000000000000 & 0 & $2^{-8}$ \\
        20 & 5000000000000008 & 84 & 1000c00000000000 & 15 & $2^{-8}$ \\
        21 & 5000000000000000 & 14 & 8000000000000000 & 3 & $2^{-4}$ \\
        21 & 5000000000000000 & 82 & 1008000000000000 & 0 & $2^{-4}$ \\
        53 & 3008000000000000 & 10 & 8000000000000900 & 6 & $2^{-6}$ \\
        54 & 3000c00000000000 & 20 & 5000000000000008 & 4 & $2^{-6}$ \\
        54 & 3000c00000000000 & 21 & 5000000000000000 & 4 & $2^{-4}$ \\
        54 & 3000c00000000000 & 108 & 1000000000000008 & 4 & $2^{-6}$ \\
        79 & 100c000000000000 & 2 & 9000000000000000 & 3 & $2^{-6}$ \\
        79 & 100c000000000000 & 81 & 1009000000000000 & 0 & $2^{-6}$ \\
        81 & 1009000000000000 & 10 & 8000000000000900 & 6 & $2^{-8}$ \\
        82 & 1008000000000000 & 10 & 8000000000000900 & 6 & $2^{-6}$ \\
        84 & 1000c00000000000 & 20 & 5000000000000008 & 4 & $2^{-6}$ \\
        84 & 1000c00000000000 & 87 & 1000500000000000 & 0 & $2^{-4}$ \\
        84 & 1000c00000000000 & 108 & 1000000000000008 & 4 & $2^{-6}$ \\
        87 & 1000500000000000 & 1 & a000000000000000 & 3 & $2^{-6}$ \\
        87 & 1000500000000000 & 8 & 800000000000a000 & 7 & $2^{-8}$ \\
        87 & 1000500000000000 & 9 & 8000000000008000 & 7 & $2^{-8}$ \\
        107 & 100000000000000a & 0 & c000000000000000 & 3 & $2^{-8}$ \\
        107 & 100000000000000a & 13 & 8000000000000008 & 3 & $2^{-6}$ \\
        107 & 100000000000000a & 54 & 3000c00000000000 & 15 & $2^{-8}$ \\
        107 & 100000000000000a & 79 & 100c000000000000 & 0 & $2^{-8}$ \\
        107 & 100000000000000a & 84 & 1000c00000000000 & 15 & $2^{-8}$ \\
        108 & 1000000000000008 & 0 & c000000000000000 & 3 & $2^{-8}$ \\
        108 & 1000000000000008 & 54 & 3000c00000000000 & 15 & $2^{-8}$ \\
        108 & 1000000000000008 & 79 & 100c000000000000 & 0 & $2^{-8}$ \\
        108 & 1000000000000008 & 84 & 1000c00000000000 & 15 & $2^{-8}$ \\
        114 & 1000000000000000 & 14 & 8000000000000000 & 3 & $2^{-4}$ \\
        114 & 1000000000000000 & 82 & 1008000000000000 & 0 & $2^{-4}$ \\        
		\hline
	\end{tabular}
\end{table}

\begin{table}
	\caption{The edges of the generated graph for KNOT-256 containing differential rotational iterative characteristics when $A=2$ (corresponding to Figure \ref{fig:graph-knot256-ddt})}\label{tab:dis-knot256-ddt}
	\centering
	\begin{tabular}{|c|c|c|c|c|c|}
		\hline
		head No. & head difference & tail No. & tail difference & $d$ & $\Prob$ \\
		\hline
		29 & X[0]=0x4,X[47]=0xc & 29 & X[0]=0x4,X[47]=0xc & 56 & $2^{-6}$ \\
        29 & X[0]=0x4,X[47]=0xc & 72 & X[0]=0x2,X[40]=0xc & 63 & $2^{-5}$ \\
        72 & X[0]=0x2,X[40]=0xc & 73 & X[0]=0x2,X[40]=0xa & 63 & $2^{-5}$ \\
        72 & X[0]=0x2,X[40]=0xc & 196 & X[0]=0x1,X[39]=0x2 & 0 & $2^{-5}$ \\
        73 & X[0]=0x2,X[40]=0xa & 73 & X[0]=0x2,X[40]=0xa & 63 & $2^{-6}$ \\
        73 & X[0]=0x2,X[40]=0xa & 196 & X[0]=0x1,X[39]=0x2 & 0 & $2^{-6}$ \\
        140 & X[0]=0x1,X[17]=0x4 & 29 & X[0]=0x4,X[47]=0xc & 9 & $2^{-6}$ \\
        140 & X[0]=0x1,X[17]=0x4 & 72 & X[0]=0x2,X[40]=0xc & 16 & $2^{-5}$ \\
        196 & X[0]=0x1,X[39]=0x2 & 140 & X[0]=0x1,X[17]=0x4 & 39 & $2^{-6}$ \\
		\hline
	\end{tabular}
\end{table}

\begin{table}
	\caption{The edges of the generated graph for KNOT-256 containing linear rotational iterative characteristics when $A=2$ (corresponding to Figure \ref{fig:graph-knot256-lat2})}\label{tab:dis-knot256-lat2}
	\centering
	\begin{tabular}{|c|c|c|c|c|c|}
		\hline
		head No. & head mask & tail No. & tail mask & $d$ & $\Cor^2$ \\
		\hline
		32 & X[0]=0x4,X[47]=0x9 & 189 & X[0]=0x1,X[39]=0xc & 0 & $2^{-6}$ \\
        32 & X[0]=0x4,X[47]=0x9 & 193 & X[0]=0x1,X[39]=0x4 & 0 & $2^{-6}$ \\
        55 & X[0]=0x3,X[57]=0x4 & 252 & X[0]=0x1,X[57]=0x1 & 0 & $2^{-6}$ \\
        56 & X[0]=0x3 & 274 & X[0]=0x1 & 0 & $2^{-2}$ \\
        73 & X[0]=0x2,X[40]=0x9 & 190 & X[0]=0x1,X[39]=0xa & 0 & $2^{-6}$ \\
        153 & X[0]=0x1,X[24]=0x2 & 190 & X[0]=0x1,X[39]=0xa & 24 & $2^{-6}$ \\
        189 & X[0]=0x1,X[39]=0xc & 32 & X[0]=0x4,X[47]=0x9 & 56 & $2^{-6}$ \\
        189 & X[0]=0x1,X[39]=0xc & 73 & X[0]=0x2,X[40]=0x9 & 63 & $2^{-8}$ \\
        189 & X[0]=0x1,X[39]=0xc & 153 & X[0]=0x1,X[24]=0x2 & 39 & $2^{-8}$ \\
        190 & X[0]=0x1,X[39]=0xa & 32 & X[0]=0x4,X[47]=0x9 & 56 & $2^{-4}$ \\
        190 & X[0]=0x1,X[39]=0xa & 73 & X[0]=0x2,X[40]=0x9 & 63 & $2^{-6}$ \\
        190 & X[0]=0x1,X[39]=0xa & 153 & X[0]=0x1,X[24]=0x2 & 39 & $2^{-6}$ \\
        193 & X[0]=0x1,X[39]=0x4 & 32 & X[0]=0x4,X[47]=0x9 & 56 & $2^{-6}$ \\
        193 & X[0]=0x1,X[39]=0x4 & 73 & X[0]=0x2,X[40]=0x9 & 63 & $2^{-8}$ \\
        193 & X[0]=0x1,X[39]=0x4 & 153 & X[0]=0x1,X[24]=0x2 & 39 & $2^{-8}$ \\
        245 & X[0]=0x1,X[56]=0x6 & 246 & X[0]=0x1,X[56]=0x5 & 0 & $2^{-6}$ \\
        246 & X[0]=0x1,X[56]=0x5 & 246 & X[0]=0x1,X[56]=0x5 & 0 & $2^{-6}$ \\
        247 & X[0]=0x1,X[56]=0x4 & 246 & X[0]=0x1,X[56]=0x5 & 0 & $2^{-6}$ \\
        252 & X[0]=0x1,X[57]=0x1 & 245 & X[0]=0x1,X[56]=0x6 & 0 & $2^{-6}$ \\
        271 & X[0]=0x1,X[63]=0x3 & 55 & X[0]=0x3,X[57]=0x4 & 63 & $2^{-6}$ \\
        271 & X[0]=0x1,X[63]=0x3 & 56 & X[0]=0x3 & 63 & $2^{-6}$ \\
        271 & X[0]=0x1,X[63]=0x3 & 271 & X[0]=0x1,X[63]=0x3 & 0 & $2^{-6}$ \\
        274 & X[0]=0x1 & 247 & X[0]=0x1,X[56]=0x4 & 0 & $2^{-2}$ \\
		\hline
	\end{tabular}
\end{table}

\begin{table}
	\caption{The edges of the generated graph for GIFT-64 containing linear iterative characteristics when $A=2$ (corresponding to Figure \ref{fig:graph-gift64-lat2})}\label{tab:dis-gift64-lat2}
	\centering
	\begin{tabular}{|c|c|c|c|c|}
		\hline
		head No. & head mask & tail No. & tail mask & $\Cor^2$ \\
		\hline
		12 & a0000000a0000000 & 1059 & 0000202000000000 & $2^{-4}$ \\
        157 & 2020000000000000 & 949 & 0000a0000000a000 & $2^{-8}$ \\
        270 & 0a0000000a000000 & 157 & 2020000000000000 & $2^{-4}$ \\
        407 & 0202000000000000 & 12 & a0000000a0000000 & $2^{-8}$ \\
        513 & 00a0000000a00000 & 2049 & 0000000000002020 & $2^{-4}$ \\
        740 & 000a0000000a0000 & 1699 & 0000000020200000 & $2^{-4}$ \\
        949 & 0000a0000000a000 & 1245 & 0000020200000000 & $2^{-4}$ \\
        1059 & 0000202000000000 & 1143 & 00000a0000000a00 & $2^{-8}$ \\
        1143 & 00000a0000000a00 & 407 & 0202000000000000 & $2^{-4}$ \\
        1245 & 0000020200000000 & 270 & 0a0000000a000000 & $2^{-8}$ \\
        1322 & 000000a0000000a0 & 2097 & 0000000000000202 & $2^{-4}$ \\
        1485 & 0000000a0000000a & 1811 & 0000000002020000 & $2^{-4}$ \\
        1699 & 0000000020200000 & 1322 & 000000a0000000a0 & $2^{-8}$ \\
        1811 & 0000000002020000 & 513 & 00a0000000a00000 & $2^{-8}$ \\
        2049 & 0000000000002020 & 1485 & 0000000a0000000a & $2^{-8}$ \\
        2097 & 0000000000000202 & 740 & 000a0000000a0000 & $2^{-8}$ \\
		\hline
	\end{tabular}
\end{table}

\end{document}
