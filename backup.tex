\subsection{Differential Trails, Differentials and Truncated Differentials \cite{DR02}}

Let $\beta$ be an iterative Boolean transformation from $\mathbb{F}_2^n$ to $\mathbb{F}_2^n$: 
\[
    \beta=\rho^{(r)}\circ\rho^{(r-1)}\circ\cdots\circ\rho^{(2)}\circ\rho^{(1)}.
\]
A \textit{diffrential trail} $Q$ over $\beta$ consists of a sequence of $r+1$ differences:
\[
    Q=(q^{(0)},q^{(1)},q^{(2)},\cdots,q^{(r-1)},q^{(r)}).
\]
The probability of a differential step $(q^{(i-1)},q^{(i)})$ is defined as:
\begin{align*}
    \Prob^{\rho^{(i)}}(q^{(i-1)},q^{(i)})&=\Prob_x[\rho^{(i)}(x)\oplus\rho^{(i)}(x\oplus q^{(i-1)})=q^{(i)}]\\
    &=2^{-n}\times\#\{x\in \mathbb{F}_2^n|\rho^{(i)}(x)\oplus\rho^{(i)}(x\oplus q^{(i-1)})=q^{(i)}\}
\end{align*}
Assuming the independence of the differential steps, the probability of $Q$ is:
\[
    \Prob^{\beta}(Q)=\prod\limits_i\Prob^{\rho^{(i)}}(q^{(i-1)},q^{(i)}).
\]
A \textit{differential} of $\beta$ is composed of $r$-round differential trails with the same initial and final differences. The probability of a differential $(a,b)$ is the sum of the probabilities of all these differential trails:
\[
    \Prob^{\beta}(a,b)=\sum\limits_{q^{(0)}=a,q^{(r)}=b}\Prob^{\beta}(Q).
\]
Let $\lambda$ be a linear function corresponding to an $n\times l$ binary matrix $M$. The probabilities of \textit{truncated} differentials of $\lambda\circ\beta$ are given by:
\[
    \Prob^{\lambda\circ\beta}(a,b)=\sum\limits_{\omega|b=M\omega}\Prob^{\lambda\circ\beta}(a,\omega).
\]

\subsection{Linear Trails and Linear Hulls \cite{DR02}}

A \textit{linear trail} $U$ over $\beta$ consists of a sequence of $r+1$ masks:
\[
    U=(u^{(0)},u^{(1)},u^{(2)},\cdots,u^{(r-1)},u^{(r)}).
\]
The correlation of a linear step $(u^{(i-1)},u^{(i)})$ is defined as:
\begin{align*}
    \Cor^{\rho^{(i)}}(u^{(i-1)},u^{(i)})&=2\times(\Prob_x[u^{(i-1)}\cdot x=u^{(i)}\cdot \rho^{(i)}(x)]-\frac{1}{2})\\
    &=2^{-n+1}\times\#\{x\in \mathbb{F}_2^n|u^{(i-1)}\cdot x=u^{(i)}\cdot \rho^{(i)}(x)\}-1.
\end{align*}
The correlation of $U$ is:
\[
    \Cor^{\beta}(U)=\prod\limits_i\Cor(u^{(i-1)},u^{(i)}).
\]
A \textit{linear hull} of $\beta$ is composed of $r$-round linear trails with the same initial and final masks. The correlation of a linear hull $(a,b)$ is the sum of the correlations of all these linear trails:
\[
    \Cor^{\beta}(a,b)=\sum\limits_{u^{(0)}=a,u^{(r)}=b}\Cor(U).
\]
A key-alternating cipher $\beta'$ consists of key-independent round transformations $\rho^{(i)}$ and simple key addition by means of XOR denoted as $\sigma[k]$:
\[
    \beta'=\sigma[k^{(r)}]\circ\rho^{(r)}\circ\sigma[k^{(r-1)}]\circ\cdots\circ\sigma[k^{(1)}]\circ\rho^{(1)}\circ\sigma[k^{(0)}].
\]
For a key-alternating cipher, The amplitude of the correlation of a linear trail is independent of the round keys:
\begin{align*}
    \Cor^{\beta'}(U)&=(-1)^{u^{(0)}\cdot k^{(0)}}\prod\limits_i(-1)^{u^{(i)}\cdot k^{(i)}}\Cor^{\rho^{(i)}}(u^{(i-1)},u^{(i)}).\\
    &=(-1)^{U\cdot K}\cdot(-1)^{d_U}\bigg\lvert\prod\limits_i\Cor^{\rho^{(i)}}(u^{(i-1)},u^{(i)})\bigg\rvert\\
    &=(-1)^{d_U\oplus U\cdot K}\Big\lvert\Cor^{\beta'}(U)\Big\rvert,
\end{align*}
where $K=(k^{(0)},k^{(1)},k^{(2)},\cdots,k^{(r-1)},k^{(r)})$, $d_U=1$ if $\prod\limits_i\Cor^{\rho^{(i)}}(u^{(i-1)},u^{(i)})<0$ and $d_U=0$ otherwise. The correlation of a linear hull $(a,b)$ for a key-alternating cipher is:
\[
    \Cor^{\beta'}(a,b)=\sum\limits_{u^{(0)}=a,u^{(r)}=b}(-1)^{d_U\oplus U\cdot K}\lvert\Cor^{\beta'}(U)\rvert.
\]
We denote the square of a correlation by correlation potential. The average correlation potential between an input and an output mask is the sum of the correlation potentials of all linear trails between the input and output masks:
\[
    \text{Exp}_{K}[(\Cor^{\beta'}(a,b))^2]=\sum\limits_{u^{(0)}=a,u^{(r)}=b}(\Cor^{\beta'}(U))^2.
\]

\subsection{EDP and ELP \cite{DR02}}
The differential probabilities and linear correlations of a cipher $\calE_K$ both depend on the specific key used $K$. In the case of differential cryptanalysis, EDP (expected differential probability) is defined as:
\[
    \EDP(a,b)=\Exp_{K}[\Prob^{\calE_K}(a,b)].
\]
It is often assumed that
\[
    \Prob^{\calE_K}(a,b)\approx\EDP(a,b)
\]
for most keys. In the case of linear cryptanalysis, ELP (expected linear potential) is defined as:
\begin{align*}
    \ELP(a,b)&=\Exp_{K}[(\Cor^{\calE_K}(a,b))^2]\\
    &=\sum\limits_{u^{(0)}=a,u^{(r)}=b}(\Cor^{\calE_K}(U))^2.
\end{align*}
    

If the cipher is a key-alternating one, $\calE_K=\sigma[k^{(r)}]\circ\rho^{(r)}\circ\sigma[k^{(r-1)}]\circ\cdots\circ\sigma[k^{(1)}]\circ\rho^{(1)}\circ\sigma[k^{(0)}]$. Let $\calE$ be $\rho^{(r)}\circ\cdots\circ\rho^{(1)}$ without key addition, then we define:
\[
    \ELP(a,b)=\Big(\sum\limits_{u^{(0)}=a,u^{(r)}=b}\Cor^{\calE}(U)\Big)^2.
\]

%For the sponge construction and its variant, no key is involved in the inner permutations. Thus the correlation of a linear hull is directly the signed sum of the correlations of the linear trails.


\bibitem{AKM97}
Aoki, K., Kobayashi, K.,  Moriai, S. (1997, January). Best differential characteristic search of FEAL. In International Workshop on Fast Software Encryption (pp. 41-53). Springer, Berlin, Heidelberg.

\bibitem{BDPV12}
Bertoni, G., Daemen, J., Peeters, M.,  Van Assche, G. (2012). Permutation-based encryption, authentication and authenticated encryption. Directions in Authenticated Ciphers, 159-170.

\bibitem{BDPVV14}
Bertoni, G., Daemen, J., Peeters, M., Van Assche, G.,  Van Keer, R. (2014). CAESAR submission: Ketje v2. CAESAR First Round Submission, March.

\bibitem{BKL07}
Bogdanov, A., Knudsen, L. R., Leander, G., Paar, C., Poschmann, A., Robshaw, M. J., ...  Vikkelsoe, C. (2007, September). PRESENT: An ultra-lightweight block cipher. In International workshop on cryptographic hardware and embedded systems (pp. 450-466). Springer, Berlin, Heidelberg.

\bibitem{BPP17}
Banik S, Pandey SK, Peyrin T, Sasaki Y, Sim SM, Todo Y. GIFT: a small present. InInternational Conference on Cryptographic Hardware and Embedded Systems 2017 Sep 25 (pp. 321-345). Springer, Cham.

\bibitem{BS91}
Biham E, Shamir A. Differential cryptanalysis of DES-like cryptosystems. Journal of CRYPTOLOGY. 1991 Jan 1;4(1):3-72.

\bibitem{BS92}
Biham E, Shamir A. Differential cryptanalysis of the full 16-round DES. InAnnual International Cryptology Conference 1992 Aug 16 (pp. 487-496). Springer, Berlin, Heidelberg.

\bibitem{BZL14}
Bao Z, Zhang W, Lin D. Speeding up the search algorithm for the best differential and best linear trails. In International Conference on Information Security and Cryptology 2014 Dec 13 (pp. 259-285). Springer, Cham.

\bibitem{DEM15}
Dobraunig C, Eichlseder M, Mendel F. Heuristic tool for linear cryptanalysis with applications to CAESAR candidates[C]//International Conference on the Theory and Application of Cryptology and Information Security. Springer, Berlin, Heidelberg, 2015: 490-509.

\bibitem{DEMS16}
Dobraunig C, Eichlseder M, Mendel F, Schläffer M. Ascon v1.2. Submission to the CAESAR Competition. 2016 Sep 15.

\bibitem{DR98}
Daemen J, Rijmen V. The block cipher Rijndael. InInternational Conference on Smart Card Research and Advanced Applications 1998 Sep 14 (pp. 277-284). Springer, Berlin, Heidelberg.

\bibitem{DR01}
Daemen J, Rijmen V. The wide trail design strategy[C]//IMA International Conference on Cryptography and Coding. Springer, Berlin, Heidelberg, 2001: 222-238.

\bibitem{DR02}
Daemen J, Rijmen V. The design of Rijndael[M]. New York: Springer-verlag, 2002.

\bibitem{HV18}
Hall-Andersen, M., \& Vejre, P. S. (2018). Generating Graphs Packed with Paths Estimation of Linear Approximations and Differentials. IACR Transactions on Symmetric Cryptology, 2018(3), 265-289.

\bibitem{J75}
Johnson DB. Finding all the elementary circuits of a directed graph. SIAM Journal on Computing. 1975 Mar;4(1):77-84.

\bibitem{K92}
Knudsen LR. Iterative Characteristics of DES and s 2-DES. InAnnual International Cryptology Conference 1992 Aug 16 (pp. 497-511). Springer, Berlin, Heidelberg.
\bibitem{M93}
Matsui M. Linear cryptanalysis method for DES cipher. InWorkshop on the Theory and Application of of Cryptographic Techniques 1993 May 23 (pp. 386-397). Springer, Berlin, Heidelberg.

\bibitem{M94_1}
Matsui M. The first experimental cryptanalysis of the Data Encryption Standard. InAnnual International Cryptology Conference 1994 Aug 21 (pp. 1-11). Springer, Berlin, Heidelberg.

\bibitem{M94_2}
Matsui M. On correlation between the order of S-boxes and the strength of DES. InWorkshop on the Theory and Application of of Cryptographic Techniques 1994 May 9 (pp. 366-375). Springer, Berlin, Heidelberg.

\bibitem{MWG12}
Mouha N, Wang Q, Gu D, Preneel B. Differential and linear cryptanalysis using mixed-integer linear programming. InInternational Conference on Information Security and Cryptology 2011 Nov 30 (pp. 57-76). Springer, Berlin, Heidelberg.

\bibitem{OMA95}
Ohta K, Moriai S, Aoki K. Improving the search algorithm for the best linear expression. InAnnual International Cryptology Conference 1995 Aug 27 (pp. 157-170). Springer, Berlin, Heidelberg.

\bibitem{R04}
Rogaway P. Nonce-based symmetric encryption. InInternational Workshop on Fast Software Encryption 2004 Feb 5 (pp. 348-358). Springer, Berlin, Heidelberg.

\bibitem{SHW14-1}
Sun S, Hu L, Wang P, Qiao K, Ma X, Song L. Automatic security evaluation and (related-key) differential characteristic search: application to SIMON, PRESENT, LBlock, DES (L) and other bit-oriented block ciphers. InInternational Conference on the Theory and Application of Cryptology and Information Security 2014 Dec 7 (pp. 158-178). Springer, Berlin, Heidelberg.

\bibitem{SHW14-2}
Sun S, Hu L, Wang M, Wang P, Qiao K, Ma X, Shi D, Song L, Fu K. Towards finding the best characteristics of some bit-oriented block ciphers and automatic enumeration of (related-key) differential and linear characteristics with predefined properties. IACRCryptology ePrint Archive. 2014;747:2014.

\bibitem{V03}
Vajnovszki V. A loopless algorithm for generating the permutations of a multiset. Theoretical Computer Science. 2003 Oct 7;307(2):415-31.

\bibitem{W08}
Wang M. Differential cryptanalysis of reduced-round PRESENT. InInternational Conference on Cryptology in Africa 2008 Jun 11 (pp. 40-49). Springer, Berlin, Heidelberg.

\bibitem{ZDY19}
Wentao Zhang, Tianyou Ding, Bohan Yang, Zhenzhen Bao, Zejun Xiang, Fulei Ji, Xuefeng Zhao. KNOT: Algorithm Specifications and Supporting Document. \url{https://csrc.nist.gov/CSRC/media/Projects/lightweight-cryptography/documents/round-2/spec-doc-rnd2/knot-spec-round.pdf}

\bibitem{ZBL15}
Zhang W, Bao Z, Lin D, Rijmen V, Yang B, Verbauwhede I. RECTANGLE: a bit-slice lightweight block cipher suitable for multiple platforms. Science China Information Sciences. 2015 Dec 1;58(12):1-5.

\bibitem{ZZDX19}
Zhou C, Zhang W, Ding T, Xiang Z. Improving the MILP-based Security Evaluation Algorithms against Differential Cryptanalysis Using Divide-and-Conquer Approach. IACR Cryptology ePrint Archive. 2019;2019:19.

%TODO: SAT search references

%TODO: generating graph references
