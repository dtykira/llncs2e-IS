\section{Conclusion\label{sec:conclusion}}

In this work, we propose a new automatic tool searching for iterative trails for symmetric-key primitives based on S-boxes. We visualize the graph representation of iterative trails hoping to provide additional insignts. Based on the iterative trails, we efficiently estimate the probabilities of differentials and correlations of linear hulls. Our results show that the combination of iterative trails is very likely to enhance differential and linear propagations. What's more, our results show that, for ciphers we conduct experiments on with bit permutations as their linear layers, the good differentials and linear hulls are dominated by iterative trails. 

We have conducted an initial study on ASCON permutation. For its comparatively strong diffusion layer, iterative trails are difficult to be found. A question raised for designers is that, whether a cipher with bit permutation as its linear layer can have no iterative trails. And we plan to continue experimenting on ciphers with bit permutations as linear layers like ICEBERG, PUFFIN, etc. Ciphers with Feistel structure, like DES, are also going to be considered. 

In Algorithm \ref{algo:1}, we generate a graph describing the 1-round differential or linear propagations and then shrink it to a subgraph named as an iterative structure. The number of active S-boxes are limited. By using the technique in Section \ref{sec:expansion}, we can take more 1-round differential or linear propagations into consideration. The collection of interesting 1-round differential or linear propagations is heuristic. One may heuristically alter the way of the collection. 

In Algorithm \ref{algo:3}, we build a multistage graph by further considering trails extended from the vertices in the iterative structure. The weight of the trails are bounded in certain heuristic way. One can loose the bounds to obtain more accurate results but costing more time and memory. One can also heuristically alter the way how the bounds restrict the extension trails. 

