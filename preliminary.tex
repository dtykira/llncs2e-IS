\section{Preliminaries\label{sec:pre}}

A block cipher is a function $\calE:\bbF_2^k \times \bbF_2^n \rightarrow \bbF_2^n$ with $C=\calE(K,P)$ where $K$, $P$ and $C$ are the $k$-bit master key, $n$-bit plaintext and $n$-bit ciphertext. $k$ is the key size and $n$ is the block size. Embedded into an operation mode, a block cipher can be used to encrypt a message with arbitrary length. 

A permutation is a function $\calP:\bbF_2^n \rightarrow \bbF_2^n$ with $SO=\calP(SI)$ where $SI$ and $SO$ are the $n$-bit input and output state. The sponge/duplex construction where a permutation is embedded can build various primitives such as a hash function, a stream cipher, a MAC or an authenticated encryption scheme \cite{bertoni2007sponge}. Note that a block cipher with key fixed $\calE_K=\calE(K,\cdot)$ can be seen as a permutation.

In this paper, we focus on symmetric-key primitives including iterated key-alternating block ciphers and permutations of SPN structures. The state of such a primitive can be seperated into $m$ words of $s$ bits and it holds that $n=s\times m$. The round function of the $i$-th round consists of three layers and is denoted by $\calR_i=\calL\circ\calS\circ\calA_{W_i}$ where the three layers are:
\begin{itemize}
    \item Addition layer $\calA_{W_i}$: xor the $i$-th $n$-bit round key or constant $W_i$ to the state;
    \item Non-linear layer $\calS$: apply $m$ parallel $s$-bit S-boxes to the words, i.e.
    \begin{align*}
        \calS=\calS_0||\cdots||\calS_{m-1};
    \end{align*}
    \item Linear layer $\calL$: multiply an $n\times n$ bijective binary matrix to the state. 
\end{itemize}
\begin{figure}[H]
    \centering
\begin{tikzpicture}
    \tikzstyle{every node}=[transform shape];
    \tikzstyle{every node}=[node distance=1.2cm];
    \tikzstyle{every node}=[font=\footnotesize,scale=0.9]
    \node (P) [] {$P(SI)$};
    \node (XOR-0) [right of=P,XOR] {};
    \node (S-0) [right of=XOR-0,draw,rectangle] {$\calS$};
    \node (L-0) [right of=S-0,draw,rectangle] {$\calL$};
    \node (XOR-1) [right of=L-0,XOR] {};
    \node (S-1) [right of=XOR-1,draw,rectangle] {$\calS$};
    \node (L-1) [right of=S-1,draw,rectangle] {$\calL$};
    \node (dots) [right of=L-1,] {$\dots$};
    \node (XOR-r-1) [right of=dots,XOR] {};
    \node (S-r-1) [right of=XOR-r-1,draw,rectangle] {$\calS$};
    \node (L-r-1) [right of=S-r-1,draw,rectangle] {$\calL$};
    \node (XOR-r) [right of=L-r-1,XOR] {};
    \node (C) [right of=XOR-r] {$C(SO)$};
    
    \path[line] (P) edge (XOR-0);
    \path[line] (XOR-0) edge node[above] {$X_0$} (S-0);
    \path[line] (S-0) edge node[above] {$Y_0$} (L-0);
    \path[line] (L-0) edge node[above] {$Z_0$} (XOR-1);
    \path[line] (XOR-1) edge node[above] {$X_1$} (S-1);
    \path[line] (S-1) edge node[above] {$Y_1$} (L-1);
    \path[line] (L-1) edge node[above] {$Z_1$} (dots);
    \path[line] (dots) edge node[above] {$Z_{r-2}$} (XOR-r-1);
    \path[line] (XOR-r-1) edge node[above] {$X_{r-1}$} (S-r-1);
    \path[line] (S-r-1) edge node[above] {$Y_{r-1}$} (L-r-1);
    \path[line] (L-r-1) edge node[above] {$Z_{r-1}$} (XOR-r);
    \path[line] (XOR-r) edge (C);

    \node (W-0) [above of=XOR-0] {$W_0$};
    \node (W-1) [above of=XOR-1] {$W_1$};
    \node (W-r-1) [above of=XOR-r-1] {$W_{r-1}$};
    \node (W-r) [above of=XOR-r] {$W_r$};

    \path[line] (W-0) edge (XOR-0);
    \path[line] (W-1) edge (XOR-1);
    \path[line] (W-r-1) edge (XOR-r-1);
    \path[line] (W-r) edge (XOR-r);

    \draw [decorate,decoration={brace,amplitude=5pt,mirror},xshift=-0.5cm,yshift=0pt] (1.2,-0.5) -- ++(2.3,0) node [black,midway,yshift=-0.5cm] {$\calR_0$};
    \draw [decorate,decoration={brace,amplitude=5pt,mirror},xshift=-0.5cm,yshift=0pt] (3.9,-0.5) -- ++(2.3,0) node [black,midway,yshift=-0.5cm] {$\calR_1$};
    \draw [decorate,decoration={brace,amplitude=5pt,mirror},xshift=-0.5cm,yshift=0pt] (7.5,-0.5) -- ++(2.3,0) node [black,midway,yshift=-0.5cm] {$\calR_{r-1}$};

    \draw [decorate,decoration={brace,amplitude=5pt},xshift=-0.5cm,yshift=0pt] (2.1,0.5) -- ++(1.3,0) node [black,midway,yshift=0.5cm] {$\calR_0^*$};
    \draw [decorate,decoration={brace,amplitude=5pt},xshift=-0.5cm,yshift=0pt] (4.8,0.5) -- ++(1.3,0) node [black,midway,yshift=0.5cm] {$\calR_1^*$};
    \draw [decorate,decoration={brace,amplitude=5pt},xshift=-0.5cm,yshift=0pt] (8.4,0.5) -- ++(1.3,0) node [black,midway,yshift=0.5cm] {$\calR_{r-1}^*$};

    %\draw [decorate,decoration={brace,amplitude=10pt,mirror},xshift=-0.5cm,yshift=0pt] (XOR-0.south west) -- node [black,midway,yshift=-0.7cm] {$\calR_0$} (L-0.south east);
    %\draw [decorate,decoration={brace,amplitude=10pt,mirror},xshift=-0.5cm,yshift=0pt] (XOR-1.south west) -- node [black,midway,yshift=-0.7cm] {$\calR_1$} (L-1.south east);
    %\draw [decorate,decoration={brace,amplitude=10pt,mirror},xshift=-0.5cm,yshift=0pt] (XOR-r-1.south west) -- node [black,midway,yshift=-0.7cm] {$\calR_{r-1}$} (L-r-1.south east);

    %\draw [decorate,decoration={brace,amplitude=10pt},xshift=-0.5cm,yshift=0pt] (S-0.north west) -- node [black,midway,yshift=0.7cm] {$\calR_0^*$} (L-0.north east);
    %\draw [decorate,decoration={brace,amplitude=10pt},xshift=-0.5cm,yshift=0pt] (S-1.north west) -- node [black,midway,yshift=0.7cm] {$\calR_1^*$} (L-1.north east);
    %\draw [decorate,decoration={brace,amplitude=10pt},xshift=-0.5cm,yshift=0pt] (S-r-1.north west) -- node [black,midway,yshift=0.7cm] {$\calR_{r-1}^*$} (L-r-1.north east);
\end{tikzpicture}
\caption{Structure of an SPN block cipher or permutation}
\label{fig:SPN}
\end{figure}
We use $W_i$ to denote the $i$-th round key for a block cipher or round constant for a permutation. The primitive iterates the round function several times (See Fig. \ref{fig:SPN}). We denote the $i$-th round function excluding the addition layer by $\calR_i^*=\calL\circ\calS$. We denote the states before the non-linear layer, before the linear layer and after the linear layer of the $i$-th round function by $X_i$, $Y_i$ and $Z_i$. $X_i[j]$ denotes the $j$-th word of $X_i$, i.e. the input value of the $j$-th S-box. For an $X_i$ with $k$ active S-boxes whose index set is $\mathfrak{K}=\{j_0,\cdots,j_{k-1}\}$, we denote it by $X_i[j_0]=x_0,\cdots,X_i[j_{k-1}]=x_{k-1}$ where $x_t\neq 0,\forall 0\leq t<k$ and $X_i[t]=0$ if $t\notin \mathfrak{K}$.

\subsection{Differential Cryptanalysis}

In differential cryptanalysis, the attacker tries to find an exploitable \textit{differential} which is a difference pair $(\Delta P=P\oplus P',\Delta C=C\oplus C')$ with high probability to distinguish the target block cipher or permutation from a random permutation. 

\begin{definition}[Differential Probability of $\calP$]
    For a permutation $\calP$, given $\alpha,\beta\in \bbF_2^n$, the differential probability of the differential $(\alpha,\beta)$ is
    \begin{align*}
        \bbP(\alpha\xrightarrow{\calP}\beta)=2^{-n}\cdot\Big|\{x\in\bbF_2^n|\calP(x)\oplus\calP(x\oplus\alpha)=\beta\}\Big|.
    \end{align*}
\end{definition}

Since the fixed-key block cipher $\calE_K$ is a permutation, its differential probability is also defined as above. However, the key $K$ of $\calE_K$ is unknown for the attacker. Thus the \textit{expected differential probability} (EDP) over all keys is defined.

\begin{definition}[EDP of $\calE$]
    For a block cipher $\calE$, given $\alpha,\beta\in \bbF_2^n$, the EDP of the differential $(\alpha,\beta)$ over a uniformly distributed random key $K\in \bbF_2^k$ is 
    \begin{align*}
        \EDP(\alpha\xrightarrow{\calE}\beta):=2^{-k}\cdot\sum\limits_{K\in \bbF_2^k}\bbP(\alpha\xrightarrow{\calE_K}\beta)
    \end{align*}
\end{definition}

\begin{definition}[Differential Characteristics]
    An $r$-round differential characteristic is a sequence of $r+1$ differences $(\alpha_0,\cdots,\alpha_r)\in(\bbF_2^n)^{r+1}$. Its probability is
    \begin{align*}
        &\bbP(\alpha_0\xrightarrow{\calR_0}\cdots\xrightarrow{\calR_{r-1}}\alpha_r)\\
        =&2^{-n}\cdot\Big|\{x\in\bbF_2^n|\forall i, \calR_i\circ\cdots\circ\calR_0(x)\oplus\calR_i\circ\cdots\circ\calR_0(x\oplus\alpha_0)=\alpha_{i+1}\}\Big|\\
    \end{align*}
\end{definition}

Under the assumption of independent random round keys and the hypothesis of stochastic equivalence \cite{lai1991markov}, the probability of a differential characteristic is calculated round by round and S-box by S-box:
\begin{align*}
    \bbP(\alpha_0\xrightarrow{\calR_0}\cdots\xrightarrow{\calR_{r-1}}\alpha_r)
    =&\prod\limits_{i=0}^{r-1}\bbP(\alpha_i\xrightarrow{\calR_i^*}\alpha_{i+1})\\
    =&\prod\limits_{i=0}^{r-1}\prod\limits_{j=0}^{m-1}\bbP(\alpha_i[j]\xrightarrow{\calS_i}\beta_i[j])
\end{align*}
where $\alpha_{i+1}=\calL(\beta_i),0\leq i<r$. 

The probability of a differential is calculated by summing the probabilities of all differential characteristics sharing the same input and output differences:
\begin{align*}
    \bbP(\alpha\xrightarrow{\calP}\beta) \text{ or } \EDP(\alpha\xrightarrow{\calE}\beta)=&\sum\limits_{\alpha_1,\cdots,\alpha_{r-1}}\bbP(\alpha_0\xrightarrow{\calR_0}\cdots\xrightarrow{\calR_{r-1}}\alpha_r)\\
    =&\sum\limits_{\alpha_1,\cdots,\alpha_{r-1}}\prod\limits_{i=0}^{r-1}\prod\limits_{j=0}^{m-1}\bbP(\alpha_i[j]\xrightarrow{\calS_i}\beta_i[j])
\end{align*}

\subsubsection{Truncated Differential}
Let $\lambda$ be a linear function corresponding to an $n\times l$ binary matrix $M$. The probability of a \textit{truncated} differential of $\lambda\circ\calP$ is given by \cite{daemen2002design}:
\[
    \bbP(\alpha\xrightarrow{\lambda\circ\calP}\beta)=\sum\limits_{\omega|\beta=M\omega}\bbP(\alpha\xrightarrow{\calP}\omega).
\]

\subsection{Linear Cryptanalysis}

In linear cryptanalysis, the attacker tries to find an exploitable \textit{linear approximation}  $\Gamma P\cdot P\oplus \Gamma C\cdot C$ determined by a mask pair $(\Gamma P,\Gamma C)$ revealing an approximate linear relation between $P$ and $C$ and thus distinguishing the target block cipher or permutation from a random permutation. 

\begin{definition}[Correlation]
    The correlation of a Boolean function $f:\bbF_2^n\rightarrow\bbF_2$ is
    \begin{align*}
        c_f=2^{-n}\cdot\Big(\Big| \{x\in\bbF_2^n|f(x)=0\} \Big|-\Big| \{x\in\bbF_2^n|f(x)=1\} \Big|\Big)
    \end{align*}
\end{definition}

\begin{definition}[Linear Approximation]
    For a permutation $\calP$, given $\alpha,\beta\in \bbF_2^n$, $\alpha\cdot x\oplus \beta\cdot\calP(x)$ is a linear approximation of $\calP$ and we denote $c_{\alpha\cdot x\oplus \beta\cdot\calP(x)}$ by $\Cor(\alpha\xrightarrow{\calP}\beta)$.
\end{definition}

\begin{definition}[Linear Characteristics]
    An $r$-round linear characteristic is a sequence of $r+1$ masks $(\alpha_0,\cdots,\alpha_r)\in(\bbF_2^n)^{r+1}$. Its correlation is calculated by
    \begin{align*}
        \Cor(\alpha_0\xrightarrow{\calR_0}\cdots\xrightarrow{\calR_{r-1}}\alpha_r)
        =(-1)^{\oplus_{i=0}^r \alpha_i\cdot W_i}\cdot\prod\limits_{i=0}^{r-1}\Cor(\alpha_i\xrightarrow{\calR_i^*}\alpha_{i+1})
    \end{align*}
\end{definition}

In the case of key-alternating block ciphers, the \textit{expected linear potential} (ELP) of a linear approximation is calculated by summing the correlation squares of all linear characteristics sharing the same input and output masks according to Theorem 7.9.1 in \cite{daemen2002design}:
\begin{align*}
    \ELP(\alpha\xrightarrow{\calE}\beta)=&\sum\limits_{\alpha_1,\cdots,\alpha_{r-1}}\Cor^2(\alpha_0\xrightarrow{\calR_0}\cdots\xrightarrow{\calR_{r-1}}\alpha_r)\\
    =&\sum\limits_{\alpha_1,\cdots,\alpha_{r-1}}\Cor^2(\alpha_0\xrightarrow{\calR_0^*}\cdots\xrightarrow{\calR_{r-1}^*}\alpha_r)\\
    =&\sum\limits_{\alpha_1,\cdots,\alpha_{r-1}}\prod\limits_{i=0}^{r-1}\prod\limits_{j=0}^{m-1}\Cor^2(\alpha_i[j]\xrightarrow{\calS_i}\beta_i[j])
\end{align*}
where $\beta_i=\calL^T(\alpha_{i+1}),0\leq i<r$.

In the case of permutations, the correlation of a linear approximation is calculated by the signed sum of correlations all linear characteristics sharing the same input and output masks:
\begin{align*}
    \Cor(\alpha\xrightarrow{\calP}\beta)=&\sum\limits_{\alpha_1,\cdots,\alpha_{r-1}}\Cor(\alpha_0\xrightarrow{\calR_0}\cdots\xrightarrow{\calR_{r-1}}\alpha_r)\\
    =&\sum\limits_{\alpha_1,\cdots,\alpha_{r-1}}\prod\limits_{i=0}^{r-1} (-1)^{\oplus_{i=0}^r \alpha_i\cdot W_i} \prod\limits_{j=0}^{m-1}\Cor(\alpha_i[j]\xrightarrow{\calS_i}\beta_i[j]).
\end{align*}

\subsection{Concepts in Graph Theory}

A \textit{directed graph} $G(V, E)$ consists of a nonempty and finite set of \textit{vertices} $V$ and a set $E$ of ordered pairs of distinct vertices called \textit{edges}. We denote a directed edge from a vertex $u\in V$ to a vertex $v\in V$ by $u\rightarrow v$. $u$ is called the head of the edge and $v$ is called the tail of it. Each edge $u\rightarrow v$ can be associated with a \textit{cost} and it's denoted by $c(u\rightarrow v)$. A \textit{path} $p_{u,v}$ is a sequence of vertices $(u=v_0,v_1,\cdots,v_{k-1},v=v_k)$ such that $v_i\rightarrow v_{i+1}\in E, 0\leq i<k$. The \textit{length} of the path is
\[
    l(p_{u,v})=k,
\]
the \textit{cost} of the path is
\[
    c(p_{u,v})=\prod\limits_{i=1}^{k}c(v_{i-1}\rightarrow v_i).
\]
The \textit{hull} of $(u,v)$ is defined as the set of all paths $p_{u,v}$ leading from $u$ to $v$. More specifically, we define the $k$-length hull of $(u,v)$, denoted by $h^k(u,v)$, as the set of all paths $p_{u,v}$ satisfying $l(p_{u,v})=k$. The cost of $h^k_{u,v}$ is
\[
    c(h^k_{u,v})=\sum\limits_{l(p_{u,v})=k} c(p_{u,v}).
\]

A path $p_{u,u}$ is called a \textit{circuit}. A circuit is \textit{elementary} if no vertex but the first and last appears twice. Two circuits are distinct if one is not a cyclic permutation of the other. An induced subgraph $G'=(V',E')$ is a (maximal) \textit{strong component} of $G$ if for all $u, v\in V'$ there exist paths $p_{uv}$ and $p_{vu}$ and this property holds for no subgraph of $G$ induced by a vertex set $\overline{V'}$ such that $V' \subset \overline{V'} \subseteq V$.

Let $G$ be a directed graph with vertices $V$ and edges $E$. If the vertices in $V$ are partitioned into $l$ subsets $S_0,\cdots,S_{l-1}$, called \textit{stages}, such that any edge in $E$ has the form $u\rightarrow v$ with $u\in S_i$ and $v\in S_{i+1}$, $0\leq i<l$. We call the graph a \textit{multistage graph}.

%\subsubsection{Notation} In this paper, we will not distinguish a vertex from a difference or mask value, an edge from a 1-round trail, a path from a trail, a weight from a differential probability or linear correlation. Note that the term "weight" has several meanings in other literatures, like the hamming weight of a difference or mask value, the negative logarithm of a differential probability or linear correlation. 

\subsection{Estimating EDP and ELP using a Graph Approach}

In \cite{EPRINT:HalVej18}, an algorithm for differential or linear characteristic search is proposed using a multistage graph approach. To summarize, once the onr-round characteristics to be considered are determined, the best differential or linear approximation within the search space can be found. We present its main framework in three steps as follows.

\subsubsection{Generating a multistage graph}
An edge is equivalent to a one-round characteristic. The cost of the edge is determined by the differential probability or correlation of the one-round characteristic. By selecting the interesting one-round characteristics, the multistage graph $G$ is generated by adding corresponding edges between stages $S_i$ and $S_{i+1}, \forall i\in[0,r-1]$. 

\subsubsection{Graph Pruning}
\begin{enumerate}
    \item Remove any vertex in $S_0$ with no outgoing edges.
    \item Remove any vertex in $S_1$ to $S_{r-1}$, if it does not have at least one incoming and one outgong edge, remove it.
    \item Remove any vertex in $S_r$ with no incoming edges.
\end{enumerate}
Repeat the above procedure until no more vertices can be removed. 

\subsubsection{Finding the best differential or linear approximation}
\begin{enumerate}
    \item Let $\calH$ be an empty hash table. Choose an $\alpha\in S_0$ and let $\calH(\alpha)=1$.
    \item For each stage $S_0$ to $S_{r-1}$ of $G$, do:
    \begin{enumerate}
        \item Create an empty hash table $\calH'$.
        \item For each key of $\calH$, let $u$ be the corresponding vertex in $G$. Let $t=\calH(u)$. Then, for each edge $u\rightarrow v$. if $\calH'(v)$ doesn't exists, let $\calH'(v)=t\cdot c(u\rightarrow v)$. Otherwise, let $\calH'(v)=\calH'(v)+t\cdot c(u\rightarrow v)$.
        \item Let $\calH=\calH'$.
    \end{enumerate}
    \item $\calH(\beta)$ now contains $c(h^r_{\alpha,\beta})$.
    \item Repeat for a new value of $\alpha$.
\end{enumerate}
The EDP or ELP for the best differential or linear approximation is estimated by $\max\limits_{\alpha,\beta}c(h^r_{\alpha,\beta})$.